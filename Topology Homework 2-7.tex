\documentclass[12pt]{article}
\usepackage[utf8]{inputenc}
\usepackage{amsmath}
\usepackage{amsfonts}
\usepackage{amssymb}
\usepackage{empheq}
\usepackage{tikz}
\usepackage{enumitem}
\usepackage{eucal}
\usepackage{changepage}
\usetikzlibrary{automata, positioning, shapes}
\addtolength{\topmargin}{-0.75in}
\addtolength{\textheight}{1.75in}
\addtolength{\evensidemargin}{-0.5in}

\title{Topology Homework 03}
\author{Ethan Jensen, Luke Lemaitre, Kasandra Lassagne}
\date{February 7, 2020}

\begin{document}
	\maketitle
	\noindent
  \textbf{EXERCISE 1.16.} \textit{On the plane} \(\mathbb{R}^2\) \textit{, let}
	\[\mathbb{B}=\{(a,b)\times (c,d) \subseteq \mathbb{R}^2 | a<b,\ c<d\}\]
	\textbf{(a)}\ \textit{Show that} \(\mathbb{B}\) \textit{is a basis for a topology on} \(\mathbb{R}^2\). \newline
	\textbf{(b)}\ \textit{Show that the topology} \(\mathcal{T}'\) \textit{generated by} \(\mathbb{B}\) \textit{is the standard topology on} \(\mathbb{R}^2\).
	\newline \newline
	\textbf{(a)}\newline (1)\ \ Consider some \((x,y) \in \mathbb{R}^2\). \newline
	Let \(B'=(x-1 , x+1) \times (y-1, y+1)\). \newline
	Since \(x \in (x-1) , x+1)\) and \(y \in (y-1, y+1),\ (x,y) \in B'\) \newline
	So \(\exists B' \in \mathbb{B} \ni (x,y) \in B' \forall (x,y) \in \mathbb{R}^2\)
	\newline \(\square\) \newline
	(2)\ \ Let \(B_1,\ B_2 \subseteq \mathbb{B}\) \newline
	\(B_1 = (a_1, b_1) \times (a_2, b_2),\ a_1 < b_1,\ a_2 < b_2\) and \(B_2 = (c_1, d_1) \times (c_2, d_2),\ c_1 < c_1,\ c_2 < d_2\)
	\newline
	Let \((x,y) \in B_1 \cap B_2\) \newline
	Using the laws of algebra, we know that \newline
	\(\textup{max}(a_1, c_1)< x < \textup{min}(b_1, d_1)\)\ \  and\ \  \(\textup{max}(a_2, c_2) < y < \textup{min}(b_2, d_2)\)
	\newline \newline
	Let \(B'= (\textup{max}(a_1, c_1), \textup{min}(b_1, d_1)) \times (\textup{max}(a_2, c_2), \textup{min}(b_2, d_2))\)
	\newline
	Thus, \((x,y) \in B' \subseteq B_1 \cap B_2\) \newline
	\(\therefore (x,y) \in B_1 \cap B_2 \implies \exists B' \subseteq B_1 \cap B_2 \ni (x,y) \in B'\ \ \forall\  B_1,B_2\in\mathbb{B}\)
	\newline \(\square\) \newline
	\(\therefore \mathbb{B}\) is a basis for a topology.
	\newline \(\blacksquare\) \newline
	\newpage
	\noindent
	\textbf{(b)}\newline
	Let \(\mathbb{B}_1 = \{(a,b)\times (c,d) |\ a<b, c<d\}\) \newline
	Let \(\mathbb{B}_2 = \{B(p,r)|\ r > 0\}\)
	\newline
	Let \(\mathcal{T}_1\) be the topology generated by \(\mathbb{B}_1 = \mathbb{B}\)
	\newline
	Let \(\mathcal{T}_2\) be the topology generated by \(\mathbb{B}_2 \) (the standard topology).
	\newline
	\begin{figure}[ht]
	\centering
	\begin{tikzpicture}
		\draw
		(-1,0) edge (9,0)
		(0,-1) edge (0,6);
	\begin{scope}[very thick,dashed]
		\draw
		(1,1.2) edge (7, 1.2)
		(1,1.2) edge (1, 4.8)
		(1,4.8) edge (7, 4.8)
		(7,1.2) edge (7, 4.8)
		(3,2.2) circle (1cm);
		\node[] (x) at (1,-0.3) {$a$};
		\node[] (x) at (7,-0.3) {$b$};
		\node[] (x) at (-0.3,1.2) {$c$};
		\node[] (x) at (-0.3,4.8) {$d$};

		\node[] (x) at (7.2, 1) {$B_1$};
		\node[] (x) at (4.5, 1.7) {$B(p,r)$};
		\node[] (x) at (3,2) {$(x,y)$};
		\node[] (x) at (3.5,2.4) {$r$};

		\fill (3,2.2) circle[radius=2pt];
	\end{scope}
	\draw (3,2.2) edge (3.7071,2.9071);
	\end{tikzpicture}
	\caption{\(\rightarrow\)}
	\end{figure}
	\newline
	\begin{figure}[ht]
	\centering
	\begin{tikzpicture}
		\draw
		(-1,0) edge (9,0)
		(0,-1) edge (0,6);
	\begin{scope}[very thick,dashed]
		\draw
		(5.43, 3.73) rectangle (6.57, 4.87)
		(4,3.2) circle (3cm);

		\node[] (x) at (5.8, 3.3) {$B_q$};
		\node[] (x) at (7.5, 1.7) {$B(p,r)$};
		\node[] (x) at (4,2.9) {$p$};
		\node[] (x) at (6,4) {$q$};
		\node[] (x) at (5.6, 4.1) {$s$};

		\fill (4,3.2) circle[radius=2pt];
		\fill (6,4.3) circle[radius=2pt];
		\fill (5.6,4.4) circle[radius=2pt];

		\node[] (x) at (2.25,3.5) {$r$};

	\end{scope}
	\draw (4,3.2) edge (1,3.2);
	\end{tikzpicture}
	\caption{\(\leftarrow\)}
	\end{figure}
	\newline
	\(\rightarrow\) Let \(B_1 \in \mathbb{B}_1\)
	\newline
	\(\exists a,b,c,d \ni B_1 = (a,b) \times (c,d)\)
	\newline
	Let \(p = (x,y) \in B_1\), so \(a<x<b\) and \(c<y<d\).
	\newline
	Let \(r=min(x-a,b-x,y-c,d-y)\), so \(r > 0\).
	\newline \newline
	Consider \(B(p,r) \in \mathbb{B}_2\)
	\newline
	\(B(p,r)\in \mathcal{T}_2\)
	\newline
	Let \(q \in B(p,r)\) \newline
	\(\exists m,\ \theta\  \ni 0< m <r\) where \(q=(x+m\cos \theta,\  y+m\sin \theta)\)
	\newline
	Since \(-1 \leq \sin\theta \leq 1\),\ \ \(-r \leq m \sin\theta \leq r\)
	\newline
	Since \(-1 \leq \cos\theta \leq 1\),\ \ \(-r \leq m \cos\theta \leq r\)
	\newline
	\begin{adjustwidth}{2cm}{0cm}
		\begin{flushleft}
			\(a < x-r\)\ \ \ \ \ \ \ \ \ \ \ \ \ \ \(x+r<b\) \newline
			\(a<x-m\)\ \ \ \ \ \ \ \ \ \ \ \ \ \(x+m<b\) \newline
			\boxed{a < x+m\cos \theta}\ \ \ \ \ \ \boxed{x+m\cos \theta < b} \newline
			\newline
			\(c < y-r\)\ \ \ \ \ \ \ \ \ \ \ \ \ \ \(y+r<d\) \newline
			\(c<y-m\)\ \ \ \ \ \ \ \ \ \ \ \ \ \(y+m<d\) \newline
			\boxed{c < y+m\sin \theta}\ \ \ \ \ \ \boxed{y+m\cos \theta < d}
	\end{flushleft}
	\end{adjustwidth}
	\(\ \)
	\newline
	So \(q \in B_1\) \newline
	Thus, \(B(p,r) \subseteq B_1\) \newline
	By the Union Lemma, \(B_1 = \bigcup_{p \in B_1}B(p,r)\). \newline \newline
	Let \(U \in \mathcal{T}_1.
	\newline
	U = \bigcup B_k\) where \(B_k\in
	\mathbb{B}_1\) since \(\mathbb{B}_1\) generates \(\mathcal{T}_1\)
	\newline
	Thus, \(U = \bigcup\bigcup_{p \in B_k}B(p,r)\), which is a union of basis elements from \(\mathcal{T}_2\). \newline
	So \(U \in \mathcal{T}_2\)
	\newline
	\(\therefore \mathcal{T}_1 \subseteq \mathcal{T}_2\)
	\newline \(\square\) \newline \newline
	\(\leftarrow\) Let \(B(p,r) \in \mathbb{B}_2\) \newline
	Let \(q = (x_q, y_q) \in B(p,r)\). \newline
	\(\exists m \ni 0 < m < r\) and \(d(p,q) = m\) \newline
	Let \(B_q = \{(x_q + \frac{m-r}{\sqrt{2}}, x_q +\frac{r-m}{\sqrt{2}}) \times (y_q + \frac{m-r}{\sqrt{2}}, y_q +\frac{r-m}{\sqrt{2}})\}\) \newline
	\(B_q \in \mathbb{B}_1\)
	\newline
	Let \(s \in B_q\). \newline
	By the formula for Euclidian distance
	\[d(q,s)<\sqrt{\left(x_q-(x_q \pm \frac{m-r}{\sqrt{2}})\right)^2+\left(y_q-(y_q \pm \frac{m-r}{\sqrt{2}})\right)^2}\]
	\[d(q,s)<\sqrt{\left(\frac{m-r}{\sqrt{2}}\right)^2 + \left(\frac{m-r}{\sqrt{2}}\right)^2}\]
	\[d(q,s)<\sqrt{(r-m)^2}\]
	Since \(r>m\)
	\[d(q,s)<r-m\]
	By the triangular property \newline
	\(d(p,s)<d(p,q)+d(q,s)\) \newline
	\(d(p,s)<m+(r-m)\)\ \ \ so \(d(p,s)<r\) \newline
	\newline
	Thus, \(s \in B(p,r)\).
	\newline
	So \(B_q \subseteq B(p,r)\) \newline
	By the Union lemma \(\bigcup_{q \in B(p,r)}B_q=B(p,r)\)
	\newline \newline
	Let \(U \in \mathcal{T}_2\).
	\newline \(U = \bigcup B_k\) where \(B_k \in \mathbb{B}_2\) since \(\mathbb{B}_2\) generates \(\mathcal{T}_2\). \newline
	Thus, \(U = \bigcup \bigcup_{q \in B(p,r)}B_q\), which is a union of basis elements from \(\mathcal{T}_1\). \newline
	So \(U \in \mathcal{T}_1\) \newline
	\(\therefore \mathcal{T}_2 \in \mathcal{T}_1\)
	\newline \(\square\) \newline
	\(\therefore \mathcal{T}_1 = \mathcal{T}_2\) \newline
	The topology \(\mathcal{T}'\) generated by \(\mathbb{B}\) is the standard topology on \(\mathbb{R}^2\).
	\newline \(\blacksquare\) \newline

\end{document}
