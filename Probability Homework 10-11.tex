\documentclass[12pt]{article}
\usepackage[utf8]{inputenc}
\usepackage{amsmath}
\usepackage{empheq}
\addtolength{\topmargin}{-0.875in}
\addtolength{\textheight}{1.75in}

\title{Math 331 A - Probability}
\author{Ethan Jensen }
\date{October 11th, 2019}

\begin{document}

\maketitle HW p.129 \#24,31,34,35 \ p.142 \#62,72,76

\section[20pt]{p. 129 \#24}

If the probability density of \(X\) is given by \newline
\[f(x) = \left\{\begin{array}{ccc}
2x^{-3} \  \textup{for} \ x>1 \\
0 \ \ \ \ \ \textup{elsewhere}
\end{array}\right.\]
Check whether its mean and its variance exist. \newline
\newline
By Def. 4.3
\[\mu = \int_{-\infty}^{\infty}xf(x)dx \]
\[\mu = \int_{-\infty}^{1}0dx + \int_{1}^{\infty}2x^{-2}dx\]
\[\mu = \frac{2}{-1}x^{-1}|^{\infty}_1\]
\[\mu = 2\]
\boxed{\textup{The mean of X exists.}} \newline
By Def. 4.4
\[\mu_2^{'}=\int_{-\infty}^{\infty}x^2f(x)dx\]
\[\mu_2^{'} = \int_{-\infty}^{1}0dx + \int_{1}^{\infty}2x^{-1}dx\]
\[\mu_2^{'} = 2ln(x)|_1^\infty=\infty\]
By Thm. 4.6
\[\sigma^2=\mu_2^{'}-\mu^2\]
\[\sigma^2=\infty\]
\boxed{\textup{The variance of X does not exist.}}
\newpage
\maketitle HW p.129 \#24,31,34,35 \ p.142 \#62,72,76

\section[20pt]{p. 129 \#31}
What is the smallest value k in Chebychev's theorem for which the probability that a random variable will take on a value between \(\mu-k\sigma\) and \(\mu+k\sigma\) is \newline
\textbf{(a)} at least 0.95; \newline
\textbf{(b)} at least 0.99; \newline
\newline
\textbf{(a)} Let \(0.95 \leq P(|x-\mu|<k\sigma)\) and \(k \geq 0\) \newline
By Chebychev's Theorem
\[ 0.95 \leq 1 - \frac{1}{k^2}\]
\[ k^2 \geq 20\]
\[ k \geq \sqrt{20}\]
\boxed{\textup{Smallest value: }k = \sqrt{20}} \newline
\textbf{(b)} Let \(0.99 \leq P(|x-\mu|<k\sigma)\) and \(k \geq 0\) \newline
By Chebychev's Theorem
\[ 0.99 \leq 1 - \frac{1}{k^2}\]
\[ k^2 \geq 100\]
\[ k \geq \sqrt{100}\]
\boxed{\textup{Smallest value: }k = 10} \newline
\newpage
\maketitle HW p.129 \#24,31,34,35 \ p.142 \#62,72,76

\section[20pt]{p. 129 \#34}
Find the moment-generating function of the discrete random variable \(X\) whose probability density is given by
\[f(x)=2\left(\frac{1}{3}\right)^x \textup{for }x = 1,2,3...\]
By Def. 4.6
\[M_X(t)=E(e^{tX})=\sum_{x}e^{tx}f(x)\]
\[M_X(t)=\sum_{x\geq0}e^{tx}\left(\frac{1}{3}\right)^x\]
\[M_X(t)=\sum_{x\geq0}\left(\frac{1}{3}e^t\right)^x\]
By the law of convergence of geometric series
\[M_X(t)=\frac{1}{1-\frac{1}{3}e^t}\]
\boxed{M_X(t)=\frac{3}{3-e^t}}
\newpage
\maketitle HW p.129 \#24,31,34,35 \ p.142 \#62,72,76

\section[20pt]{p. 129 \#35}
If we let \(R_X(t)=\textup{ln}M_x(t)\), show that \(R'_X(0)=\mu\) and \(R''_x(0)=\sigma^2\). Also, use these results to find the mean and the variance of a random variable \(X\) having the moment-generating function
\[M_X(t) = e^{4(e^t-1)}\] \newline
By our definition of \(R_X(t)\) we have
\[R_X'(t)=\frac{M_X'(t)}{M_X(t)}\]
By the quotient rule
\[R_X''(t)=\frac{M_X''(t)M_X(t)-M_X'(t)M_X'(t)}{(M_X(t))^2}\]
\[R_X'(0)=\frac{M_X'(0)}{M_X(0)}\]
\[R_X''(0)=\frac{M_X''(0)M_X(0)-M_X'(0)M_X'(0)}{(M_X(0))^2}\]
By Thm. 4.9
\[\frac{M_X'(0)}{M_X(0)}=\mu/1\]
\[\frac{M_X''(0)M_X(0)-M_X'(0)M_X'(0)}{(M_X(0))^2}=\frac{1*\mu_2'-\mu^2}{1^2}=\mu_2'-\mu^2\]
By Thm. 4.6
\[\frac{M_X''(0)M_X(0)-M_X'(0)M_X'(0)}{(M_X(0))^2}=\sigma^2\]
\boxed{R_X'(0)=\mu}
\boxed{R_X''(0)=\sigma^2}\newline
If we let \(M_X(t) = e^{4(e^t-1)}\)
\[R_X(t)=\textup{ln}M_x(t)=4(e^t-1)\]
\[R_X'(0)=4(e^0)=4\]
\[R_X''(0)=4(e^0)=4\]
Therefore, using our results,
\boxed{\mu=4,\ \sigma^2=4}

\newpage
\maketitle HW p.129 \#24,31,34,35 \ p.142 \#62,72,76
If we let \(R_X(t)=\textup{ln}M_X(t)\)
\section[20pt]{p. 142 \#62}
The probability that Ms. Brown will sell a piece of property at a profit of \$3000 is \(\frac{3}{20}\), the probability that she will sell it at a profit of \$1,500 is \(\frac{7}{20}\), the probability that she will break even is \(\frac{7}{20}\), and the probability that she will lose \$1500 is \(\frac{3}{20}\). What is her expected profit? \newline
\newline
\[f(x)=\left\{\begin{array}{ccc}
\frac{3}{20} \ \ x = 3000,\ x = -1500\\
\frac{7}{20} \ \ x = 1500,\ x = 0\\

\end{array}\right.\]
\[E(x) = \sum_x{xf(x)}\]
\[E(x) = 3000\frac{3}{20}+1500\frac{7}{20}+0\frac{7}{20}+(-1500)\frac{3}{20}\]
\[E(x) = 750\]
\boxed{\textup{Her expected profit is \$750.}}
\newpage
\maketitle HW p.129 \#24,31,34,35 \ p.142 \#62,72,76

\section[20pt]{p. 142 \#72}
With reference to Exercise 3.92 on page 107, find the mean and the variance of the random variable in question.
\[f(x)=\left\{\begin{array}{ccc}
\frac{1}{288}(36-x^2) \ \ \textup{for} -6<x<6\\
0 \ \ \ \ \ \ \textup{elsewhere}
\end{array}\right.\]
By Def. 4.3
\[\mu = \int_{-\infty}^{\infty}xf(x)dx\]
\[\mu = \int_{-\infty}^{-6}0dx+\int_{-6}^{6}\frac{x}{288}(36-x^2)dx+\int_{6}^{\infty}0dx\]
\[\mu = \frac{1}{288}[18x^2-\frac{1}{4}x^4]_{-6}^6\]
\[\mu = 0\]
By Def. 4.4
\[\mu^{'}_{2}=\int_{-\infty}^{\infty}x^2f(x)dx\]
\[\mu^{'}_{2}=\int_{-\infty}^{-6}0dx+\int_{-6}^{6}\frac{x^2}{288}(36-x^2)dx+\int_{6}^{\infty}0dx\]
\[\mu^{'}_{2} = \frac{1}{288}[12x^3-\frac{1}{5}x^5]_{-6}^6\]
\[\mu^{'}_{2}=\frac{36}{5}\]
By Thm. 4.6
\[\sigma^2=\mu^{'}_{2}-\mu^2\]
\[\sigma^2=\frac{36}{5}\]
\boxed{\textup{The R.V. X has a mean of 0 and a variance of }\frac{36}{5}}
\newpage
\maketitle HW p.129 \#24,31,34,35 \ p.142 \#62,72,76

\section[20pt]{p. 142 \#76}
A study of the nutritional value of a certain kind of bread shows that the amount of thiamine (vitamin B\(_1\) in a slice may be looked upon as a random variable with \(\mu = 0.260\) milligrams and \(\sigma = 0.005\) milligrams. According to Chebychev's theorem, between what values must be the thiamine content of \newline
\textbf{(a)} at least \(\frac{35}{36}\) of all slices of this bread; \newline
\textbf{(b)} at least \(\frac{143}{144}\) of all slices of this bread; \newline
\newline
\textbf{(a)}By Chebychev's Theorem, we have
\[P(|x-\mu|<k\sigma) \geq 1 - \frac{1}{k^2}, \ k\geq0\]
Using the frequency interpretation of probability, \[P(|x-\mu|<k\sigma) = \frac{35}{36}\]
\[1 - \frac{1}{k^2} \geq \frac{35}{36}\]
\[1 - \frac{1}{k^2} \geq 1- \frac{1}{36}\]
\[k^2 \geq 36\]
Since \(k\geq0\)
\[k \geq 6\]
Plugging in values for k, \(\mu\) and \(\sigma\), we have
\[|x-0.26|\leq6*0.005\]
\boxed{
\begin{array}{ccc}
  \textup{The amount of thiamine (in milligrams) must be in the range [0.23,0.29]} \\
  \textup{to satisfy Chebychev's Theorem.}
  \end{array}}
\newline
\newline
\newline
\textbf{(b)} By a similar argument, without loss of generalization we have
\[k \geq 6\]
Plugging in values for k, \(\mu\) and \(\sigma\), we have
\[|x-0.26|\leq12*0.005\]
\boxed{
\begin{array}{ccc}
  \textup{The amount of thiamine (in milligrams) must be in the range [0.20,0.32]} \\
  \textup{to satisfy Chebychev's Theorem.}
  \end{array}
  }
\end{document}