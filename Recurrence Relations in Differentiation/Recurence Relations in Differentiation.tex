\documentclass[12pt]{article}
\usepackage[utf8]{inputenc}
\usepackage{amsmath}
\usepackage{amsfonts}
\usepackage{amssymb}
\usepackage{empheq}
\usepackage{tikz}
\usepackage{changepage}
\usetikzlibrary{automata, positioning, shapes, arrows}
\addtolength{\topmargin}{-0.875in}
\addtolength{\textheight}{1.75in}

\title{Recurrence Relations in Differentiation}
\author{Ethan Jensen}
\date{April 17, 2019}
\begin{document}
\maketitle
\tikzset{
  ->, % makes the edges directed
  >=triangle 45, % makes the arrow heads bold
  node distance=3cm, % specifies the minimum distance between two nodes. Change if necessary.
  every state/.style={thick, fill=gray!10}, % sets the properties for each ’state’ node
  initial text=$ $, % sets the text that appears on the start arrow
}
\bibliographystyle{ieeetran}
\noindent
\textbf{Abstract.} This paper shows how dynamic programming can be used to easily compute the derivatives of rational functions of polynomials of \(e^x\). Computing these derivatives allows us to determine various interesting identities using the generating function for the Bernoulli numbers.
\section{Introduction to Bernoulli numbers}
\textbf{Definition 0.1.}\ \
The Bernoulli numbers \(B_i\) are defined recursively by
\[\sum_{i=0}^n\binom{n+1}{i}B_i=n+1,\ \ B_0 = 1\]
The first few Bernoulli numbers are \(1, \frac{1}{2}, \frac{1}{6}, 0, \frac{-1}{30}, 0, \frac{1}{42}...\)
\newline
\textbf{Definition 0.2.}
\[\textup{Define } S_k(n) = \sum_{i=0}^ni^k\]
which is called a power sum. Some power sum formulas are given below.
\[S_0(n) = \sum_{i=0}^ni^0 = n\]
\[S_1(n) = \sum_{i=1}^ni^1 = \frac{1}{2}n^2 + \frac{1}{2}n\]
\[S_2(n) = \sum_{i=1}^ni^2 = \frac{1}{3}n^3 + \frac{1}{2}n^2 + \frac{1}{6}n\]
\[S_3(n) = \sum_{i=1}^ni^3 = \frac{1}{4}n^4 + \frac{1}{2}n^3 + \frac{1}{4}n^2\]
\[S_4(n) = \sum_{i=1}^ni^4 = \frac{1}{5}n^5 + \frac{1}{2}n^4 + \frac{1}{3}n^3 -\frac{1}{30}n\]
These power sum formulas are used to efficiently compute sums of powers.
\newpage
\noindent
\textbf{Theorem 0.1.}
\newline
\[S_k(n) = \frac{1}{k+1}\sum_{j=0}^k\binom{k+1}{j}B_jx^j\]
\textbf{Proof.}\cite{arakawa2014bernoulli}
\newline
Using telescoping series, \(S_k(n+1) - S_k(n) = (n+1)^k\)
\[S_k'(x+1) - S_k'(x) = k(x+1)^{k-1}\]
\[\sum_{x=0}^{n-1}S_k'(x+1) - S_k'(x) = \sum_{x=0}^{n-1}k(x+1)^{k-1}\]
Using telescoping series,
\[S_k'(n) - S_k'(0) = kS_{k-1}(n)\]
\[S_k'(x) = kS_{k-1}(x) + S_k'(0)\]
\(S_k'(0)\) is just some constant. Call it \(a_k\).
\newline
\boxed{S_k'(x) = kS_{k-1}(x) + a_k}
\newline \(\square\) \newline
From this formula we have
\[S_k'(0) = kS_{k-1}(0) + a_k\]
Since \(S_k(0) = 0 \forall k\)
\[S_k'(0) = a_k\]
\[S_k''(0) = kS_{k-1}'(0) = ka_{k-1}\]
\[S'''(0) = kS''(k-1)(0) = k(k-1)a_{k-2}\]
And in general, \newline
\boxed{S^{(j)}(0) = k(k-1)(k-2)...(k-j+2)a_{k-j+1}}
\newline \(\square\) \newline
Using Taylor series,
\[S_k(x) = \sum_{j=0}^{k+1}\frac{S_k^{(j)}(0)}{j!}x^j\]
From the above formula we have
\[S_k(x) = \sum_{j=0}^{k+1}\frac{k!}{(k-j+1)!j!}a_{k-j+1}x^j\]
\[S_k(x) = \frac{1}{k+1}\sum_{j=0}^{k+1}\binom{k+1}{j}a_{k+1-j}x^j\]
\[S_k(x) = \frac{1}{k+1}\sum_{j=0}^{k+1}\binom{k+1}{j}a_jx^{k+1-j}\]
Now we must determine the constants \(a_k\). Plugging in 1 for x, we have
\[S_k(1) = \frac{1}{k+1}\sum_{j=0}^{k+1}\binom{k+1}{j}a_j\]
\[k+1 = \sum_{j=0}^{k+1}\binom{k+1}{j}a_j\]
So \(a_j = B_j\) are the Bernoulli numbers by the recursive definition.
\[\therefore S_k(n) = \frac{1}{k+1}\sum_{j=0}^k\binom{k+1}{j}B_jx^j\]
\(\blacksquare\)
\newline
The power sum formula for k = 10 is
\[S_{10}(n) = \sum_{i=1}^ni^{10} = \frac{1}{11}n^{11} + \frac{1}{2}n^{10} + \frac{5}{6}n^9 -n^7 + n^5 -\frac{1}{2}n^3 + \frac{5}{66}n\]
Plugging in n = 1000 we have
\[1^{10} + 2^{10} + ... + 1000^{10} = \frac{1}{11}1000^{11} + \frac{1}{2}1000^{10} + \frac{5}{6}1000^9 -1000^7 + 1000^5 -\frac{1}{2}1000^3 + \frac{5}{66}1000\]
\[ = 91409924241424243424241924242500\]
\textbf{Theorem 0.2.}
\newline
\[B_{2n+1} = 0, \textup{ for } n > 0\]
\newpage
\bibliography{References}
\end{document}
