\documentclass[12pt]{article}
\usepackage[utf8]{inputenc}
\usepackage{amsmath}
\usepackage{amsfonts}
\usepackage{amssymb}
\usepackage{empheq}
\usepackage{tikz}
\usetikzlibrary{automata, positioning, arrows, shapes}
\addtolength{\topmargin}{-0.875in}
\addtolength{\textheight}{1.75in}

\title{Math 331 A - Probability}
\author{Ethan Jensen}
\date{November 22nd, 2019}

\begin{document}
	\maketitle HW p.184 \#27, p.193 \#37, p.202 \#63,64,65,70,72,78, p.208 \#2
	\section[20pt]{p. 184 \#27}
	Sketch the graphs of the beta densities having \newline
	\textbf{(a)} \(\alpha=2\) and \(\beta=2\); \newline
	\textbf{(b)} \(\alpha=\frac{1}{2}\) and \(\beta=1\); \newline
	\textbf{(c)} \(\alpha=2\) and \(\beta=\frac{1}{2}\); \newline
	\textbf{(d)} \(\alpha=2\) and \(\beta=5\);
	\begin{figure}[ht]
	\centering
	\begin{tikzpicture}
	\draw
	(-1,0) edge[dashed] (5,0)
	(0,-1) edge[dashed] (0,4);
	
	\draw
	(7,0) edge[dashed] (13,0)
	(8,-1) edge[dashed] (8,4);
	
	\draw
	(-1,6) edge[dashed] (5,6)
	(0,5) edge[dashed] (0,10);
	
	\draw
	(7,6) edge[dashed] (13,6)
	(8,5) edge[dashed] (8,10);
	
	\draw	
	(-0.2,1) edge[] (0.2,1)
	(-0.2,2) edge[] (0.2,2)
	(-0.2,3) edge[] (0.2,3)
	(-0.2,4) edge[] (0.2,4)
	(7.8,1) edge[] (8.2,1)
	(7.8,2) edge[] (8.2,2)
	(7.8,3) edge[] (8.2,3)
	(7.8,4) edge[] (8.2,4)
	
	(-0.2,7) edge[] (0.2,7)
	(-0.2,8) edge[] (0.2,8)
	(-0.2,9) edge[] (0.2,9)
	(-0.2,10) edge[] (0.2,10)
	(7.8,7) edge[] (8.2,7)
	(7.8,8) edge[] (8.2,8)
	(7.8,9) edge[] (8.2,9)
	(7.8,10) edge[] (8.2,10);
	
	\draw
	(1,-0.2) edge[] (1,0.2)
	(2,-0.2) edge[] (2,0.2)
	(3,-0.2) edge[] (3,0.2)
	(4,-0.2) edge[] (4,0.2)
	(5,-0.2) edge[] (5,0.2)
	
	(9,-0.2) edge[] (9,0.2)
	(10,-0.2) edge[] (10,0.2)
	(11,-0.2) edge[] (11,0.2)
	(12,-0.2) edge[] (12,0.2)
	(13,-0.2) edge[] (13,0.2)
	
	(1,5.8) edge[] (1,6.2)
	(2,5.8) edge[] (2,6.2)
	(3,5.8) edge[] (3,6.2)
	(4,5.8) edge[] (4,6.2)
	(5,5.8) edge[] (5,6.2)
	
	(9,5.8) edge[] (9,6.2)
	(10,5.8) edge[] (10,6.2)
	(11,5.8) edge[] (11,6.2)
	(12,5.8) edge[] (12,6.2)
	(13,5.8) edge[] (13,6.2);
	
	\node [] (x) at (5,-0.5) {$x$};
	\node [] (x) at (5,5.5) {$x$};
	\node [] (x) at (13,-0.5) {$x$};
	\node [] (x) at (13,5.5) {$x$};
	
	\node [] (bx) at (1.5,9.5) {$\textbf{(a)}\ b(x;2,2)$};
	\node [] (bx) at (9.5,9.5) {$\textbf{(b)}\ b(x;\frac{1}{2},1)$};
	\node [] (bx) at (1.5,3.5) {$\textbf{(c)}\ b(x;2,\frac{1}{2})$};
	\node [] (bx) at (9.5,3.5) {$\textbf{(d)}\ b(x;2,5)$};
	
	\node[shape = ellipse, fill=black, draw, inner sep =0.06cm] (spicy) at (0,6) {};
	\node[shape = ellipse, fill=black, draw, inner sep =0.06cm] (spicy) at (1,6.152) {};
	\node[shape = ellipse, fill=black, draw, inner sep =0.06cm] (spicy) at (2,6.184) {};
	\node[shape = ellipse, fill=black, draw, inner sep =0.06cm] (spicy) at (3,6.168) {};
	\node[shape = ellipse, fill=black, draw, inner sep =0.06cm] (spicy) at (4,6.135) {};
	\node[shape = ellipse, fill=black, draw, inner sep =0.06cm] (spicy) at (5,6.103) {};
	
	\node[shape = ellipse, fill=black, draw, inner sep =0.06cm] (spicy) at (8.125,7.408) {};
	\node[shape = ellipse, fill=black, draw, inner sep =0.06cm] (spicy) at (8.25,6.879) {};
	\node[shape = ellipse, fill=black, draw, inner sep =0.06cm] (spicy) at (8.5,6.484) {};
	\node[shape = ellipse, fill=black, draw, inner sep =0.06cm] (spicy) at (8.75,6.308) {};
	\node[shape = ellipse, fill=black, draw, inner sep =0.06cm] (spicy) at (9,6.208) {};
	\node[shape = ellipse, fill=black, draw, inner sep =0.06cm] (spicy) at (11,6.016) {};
	\node[shape = ellipse, fill=black, draw, inner sep =0.06cm] (spicy) at (13,6.002) {};
	
	\node[shape = ellipse, fill=black, draw, inner sep =0.06cm] (spicy) at (0,0) {};
	\node[shape = ellipse, fill=black, draw, inner sep =0.06cm] (spicy) at (0.25,0.607) {};
	\node[shape = ellipse, fill=black, draw, inner sep =0.06cm] (spicy) at (0.5,0.736) {};
	\node[shape = ellipse, fill=black, draw, inner sep =0.06cm] (spicy) at (0.75,0.669) {};
	\node[shape = ellipse, fill=black, draw, inner sep =0.06cm] (spicy) at (1,0.541) {};
	\node[shape = ellipse, fill=black, draw, inner sep =0.06cm] (spicy) at (1.5,0.299) {};
	\node[shape = ellipse, fill=black, draw, inner sep =0.06cm] (spicy) at (3,0.033) {};
	\node[shape = ellipse, fill=black, draw, inner sep =0.06cm] (spicy) at (5,0.001) {};
	
	\node[shape = ellipse, fill=black, draw, inner sep =0.06cm] (spicy) at (8,0) {};
	\node[shape = ellipse, fill=black, draw, inner sep =0.06cm] (spicy) at (9,0.033) {};
	\node[shape = ellipse, fill=black, draw, inner sep =0.06cm] (spicy) at (10,0.054) {};
	\node[shape = ellipse, fill=black, draw, inner sep =0.06cm] (spicy) at (11,0.066) {};
	\node[shape = ellipse, fill=black, draw, inner sep =0.06cm] (spicy) at (12,0.072) {};
	\node[shape = ellipse, fill=black, draw, inner sep =0.06cm] (spicy) at (13,0.074) {};
	\end{tikzpicture}
	\caption {Sketches}
	\end{figure}
	\newpage
	\maketitle HW p.184 \#27, p.193 \#37, p.202 \#63,64,65,70,72,78, p.208 \#2
	\section[20pt]{p. 193 \#37}
	If \(X\) is a random variable having the standard normal distribution and \(Y=X^2\), show that cov\((X,Y)=0\) even though \(X\) and \(Y\) are evidently not independent. \newline \newline
	Via an interesting fact derived in class, Y is a \(\chi^2\) R.V. with \(\nu=1\). \newline \newline
	By Corollary 6.2 and Def. 6.7 we have
	\[\mu_X=0,\ \mu_Y=1\]
	By Theorem 4.11 we have
	\[\textup{cov}(X,Y)=\mu_{1,1}'-\mu_X\mu_Y\]
	\[\textup{cov}(X,Y)=\mu_{1,1}'-0\cdot 1=\mu_{1,1}'\]
	By Def. 4.7,
	\[\textup{cov}(X,Y)=E(XY)=E(X^3)\]
	By the definition of expected value, we have
	\[\textup{cov}(X,Y)=\frac{1}{\sqrt{2\pi}}\int_{-\infty}^{\infty}x^3e^{-\frac{1}{2}x^2}\]
	Note that the above integral is integrating an odd function over symmetric limits. This makes the result of the integral 0. \newline
	\boxed{\textup{cov}(X,Y)=0} \newline \newline
	\(\blacksquare\)
	\newpage
	\maketitle HW p.184 \#27, p.193 \#37, p.202 \#63,64,65,70,72,78, p.208 \#2
	\section[20pt]{p. 202 \#63}
	If Z is a random variable having the standard normal distribution, find the probabilities that it will take on a value \newline
	\textbf{(a)}\ greater than 1.14; \newline
	\textbf{(b)}\ greater than -0.36; \newline
	\textbf{(c)}\ between -0.46 and -0.09; \newline
	\textbf{(d)}\ between -0.58 and 1.12. \newline \newline
	Referencing Table III, z = 1.14 corresponds to the value 0.3729. Thus, \(P(Z>1.14)=0.5-0.3729\). \newline
	\boxed{\textbf{(a)}\ P(Z>1.14)=0.1271}
	\newline
	\newline
	Referencing Table III, z = 0.36 corresponds to the value 0.1406. Thus, \(P(Z>-0.36)=0.5+0.1406\). \newline
	\boxed{\textbf{(b)}\ P(Z>-0.36)=0.6406}
	\newline
	\newline
	Referencing Table III, z = 0.46 and z = 0.09 correspond to the values 0.1772 and 0.0359 respectively. Thus, \(P(-0.46<Z<0.09)=0.1772+0.0359\). \newline
	\boxed{\textbf{(c)}\ P(Z>-0.36)=0.2131}
	\newline
	\newline
	Referencing Table III, z = 0.58 and z = 1.12 correspond to the values 0.2190 and 0.3686 respectively. Thus, \(P(-0.58<Z<1.12)=0.2190+0.3686\). \newline
	\boxed{\textbf{(c)}\ P(-0.58<Z<1.12)=0.5876}
	\newpage
	\maketitle HW p.184 \#27, p.193 \#37, p.202 \#63,64,65,70,72,78, p.208 \#2
	\section[20pt]{p. 202 \#64}
	If Z is a random variable having the standard normal distribution, find the respective values \(z_1,z_2,z_3,\) and \(z_4\) such that \newline
	\textbf{(a)}\ \(P(0<Z<z_1)=0.4306\); \newline
	\textbf{(b)}\ \(P(Z\geq z_2)=0.7704\); \newline
	\textbf{(c)}\ \(P(Z>z_3)=0.2912\); \newline
	\textbf{(d)}\ \(P(-z_4<Z<z_4)=0.9700\). \newline
	\newline
	Referencing Table III, \(z_1=1.48\) satisfies the given condition. \newline
	\boxed{\textbf{(a)}\ z_1=1.48} \newline \newline
	Referencing Table III, if \(z_2=-0.74\) then \(P(Z\geq z_2)=0.5+0.2704=0.7704\), satisfying the given condition.
	\newline
	\boxed{\textbf{(b)}\ z_2=-0.74} \newline \newline
	Referencing Table III, if \(z_3=0.55\) then \(P(Z> z_3)=0.5-0.2088=0.2912\), satisfying the given condition.
	\newline
	\boxed{\textbf{(c)}\ z_3=0.55} \newline \newline
	Referencing Table III, if \(z_4=2.17\) then \(P(-z_4<Z<z_4)=0.4850+0.4850=0.9700\), satisfying the given condition.
	\newline
	\boxed{\textbf{(d)}\ z_4=2.17}
	\newpage
	\maketitle HW p.184 \#27, p.193 \#37, p.202 \#63,64,65,70,72,78, p.208 \#2
	\section[20pt]{p. 202 \#65}
	Find z if the standard-normal curve area \newline
	\textbf{(a)}\ between 0 and z is 0.4726; \newline
	\textbf{(b)}\ to the left of z is 0.9868; \newline
	\textbf{(c)}\ to the right of z is 0.1314; \newline
	\textbf{(d)}\ between -z and z is 0.8502. \newline
	\newline
	Referring to Table III, \(z=1.92\) satisfies the given condition. \newline
	\boxed{\textbf{(a)}\ z=1.92} \newline \newline
	Referring to Table III, \(z=-2.22\) satisfies the given condition. \newline
	\boxed{\textbf{(b)}\ z=-2.22} \newline \newline
	Referring to Table III, \(z=-1.12\) satisfies the given condition. \newline
	\boxed{\textbf{(c)}\ z=-1.12} \newline \newline
	Referring to Table III, \(z=1.44,z=-1.44\) satisfy the given condition. \newline
	\boxed{\textbf{(d)}\ z=1.44,\ z=-1.44}
	\newpage
	\maketitle HW p.184 \#27, p.193 \#37, p.202 \#63,64,65,70,72,78, p.208 \#2
	\section[20pt]{p. 202 \#70}
	Suppose that during periods of meditation the reduction of a person's oxygen consumption is a random variable having a normal distribution with \(\mu=37.6\)cc per minute and \(\sigma=4.6\)cc per minute. Find the probabilities that during a period of meditation a person's oxygen consumption will be reduced by \newline
	\textbf{(a)}\ at least 44.5 cc per minute; \newline
	\textbf{(b)}\ at most 35.0 cc per minute; \newline
	\textbf{(c)}\ anywhere from 30.0 to 40.0 cc per minute.
	\newline \newline
	By Theorem 6.7, we find that
	\[z=\frac{45-37.6}{4.6}=1.5\]
	Thus, referring to Table III with the z entry being 1.5, we obtain the value 0.4332. \newline
	\boxed{\textup{The probability the oxygen is reduced by at least 44.5cc is about 0.0768 or  7.68\%}} \newline \newline
	By Theorem 6.7, we find that
	\[z=\frac{35-37.6}{4.6}=\frac{-13}{23}\approx 0.5652\]
	Thus, referring to Table III with the z entry being 0.5652, we obtain the value 0.2157. \newline
	\boxed{\textup{The probability the oxygen is reduced by at most 35cc is about 0.7157 or  71.57\%}} \newline \newline
	By Theorem 6.7, we find that
	\[z=\frac{30-37.6}{4.6}=\frac{-38}{23}\approx -1.6522\]
	\[z=\frac{40-37.6}{4.6}=\frac{12}{23}\approx 0.5217\]
	Thus, referring to Table III with the z values 1.6522 and 0.5217 we obtain the values 0.4505 and 0.1985 respectively. \newline
	\boxed{\textup{The probability the reduced oxygen is between 30-40cc is about 0.649 or 64.9\%}}
	\newpage
	\maketitle HW p.184 \#27, p.193 \#37, p.202 \#63,64,65,70,72,78, p.208 \#2
	\section[20pt]{p. 202 \#72}
	A random variable has a normal distribution with \(\sigma=10\). If the probability that the random variable will take on a value less than 82.5 is 0.8212, what is the probability that it will take on a value greater than 58.3? \newline \newline
	Since \(\sigma\) is given to us, all we need to do to find the random variable's probability distribution function is use our information to solve for \(\mu\).
	\newline \newline
	By Theorem 6.7, we know that \(z=\frac{\mu-82.5}{10}\) must have the value 0.3212. \newline \newline
	Referencing Table III we have \(z=0.92\). \newline
	Solving for \(\mu\) we have
	\[\frac{82.5-\mu}{10}=0.92 \implies \mu=73.3\]
	Now that we have a value for \(\mu\), we use Theorem 6.7 a second time.
	\[z=\frac{73.3-58.3}{10}=1.53\]
	Referencing Table III with z = 1.53, we have the value 0.4370. \newline
	\boxed{P(X>58.3)=0.9370}
	\newpage
	\maketitle HW p.184 \#27, p.193 \#37, p.202 \#63,64,65,70,72,78, p.208 \#2
	\section[20pt]{p. 202 \#78}
	If 23 percent of all patients with high blood pressure have bad side effects from a certain kind of medicine, use the normal approximation to find the probability that among 120 patients with high blood pressure treated with this medicine more than 32 will have bad side effects. \newline \newline
	The amount of patients that suffer from bad side effects treated with the medicine can be modelled with a binomial random variable with \(\theta=0.23\) and \(n=120\). \newline \newline
	By Theorem 6.8, we can model the binomial distribution with a normal distribution with parameters \(\mu=n\theta\) and \(\sigma=\sqrt{n\theta(1-\theta)}\).
	\newline
	Plugging in our values for \(n\) and \(\theta\) we have
	\[\mu=27.6,\ \ \sigma=4.6100\]
	Then with Theorem 6.7, \(z=\frac{32-27.6}{4.61}=0.9544\).
	Referring to Table III, \(z=0.9544\) corresponds to the value 0.3289. Therefore, \newline
	\boxed{\textup{The probablity more than 32 patients will have bad side effects is} \approx 0.1711 \textup{ or } 17\%}
	\newpage
	\maketitle HW p.184 \#27, p.193 \#37, p.202 \#63,64,65,70,72,78, p.208 \#2
	\section[20pt]{p. 208 \#2}
	If the probability density of \(X\) is given by
	\[f(x)=\left\{\begin{array}{ccc}
	2xe^{-x^2}\ \textup{for }x>0\\0 \ \ \ \ \ \ \ \textup{elsewhere}
	\end{array}\right.\]
	and \(Y=X^2\), find \newline
	\textbf{(a)}\ the distribution function of \(Y\);
	\textbf{(b)}\ the probability density of \(Y\).
	\newline \newline
	By Definition 3.5 the CDF of Y is given by
	\[F(y)=P(Y\leq x)=P(X^2\leq x)=P(-\sqrt{x}\leq X\leq \sqrt{x})=\int_{-\sqrt{x}}^{\sqrt{x}}f(t)dt\]
	\[P(X\leq \sqrt{x})=\int_{0}^{\sqrt{x}}2te^{-t^2}dt\]
	Let \(u=t^2,\ du=2t\ dt\)
	\[F(y)=P(X\leq x)=\int_{0}^{x^{\frac{1}{4}}}e^{-u}du=1-e^{-{x^{\frac{1}{4}}}}\]
	We can now differentiate \(F(y)\) to obtain \(f(y)\).
	\[f(y)=\frac{1}{4}x^{\frac{-3}{4}}e^{-{x^{\frac{1}{4}}}}\]
	\boxed{\begin{array}{ccc}
		\textbf{(a)}\ F(y)=1-e^{-{x^{\frac{1}{4}}}}\\ \textbf{(b)}\ f(y)=\frac{1}{4}x^{\frac{-3}{4}}e^{-{x^{\frac{1}{4}}}}
	\end{array}}
\end{document}