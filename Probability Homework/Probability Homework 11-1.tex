\documentclass[12pt]{article}
\usepackage[utf8]{inputenc}
\usepackage{amsmath}
\usepackage{amsfonts}
\usepackage{amssymb}
\usepackage{empheq}
\usepackage{tikz}
\usetikzlibrary{automata, positioning, arrows, shapes}
\addtolength{\topmargin}{-0.875in}
\addtolength{\textheight}{1.75in}

\title{Math 331 A - Probability}
\author{Ethan Jensen}
\date{November 1st, 2019}

\begin{document}
	\maketitle HW p.90 \#42,44,50,51 \ p.91 \#54,62,68 \ p.100 \#70ab,76
	\tikzset{
		node distance=3cm, % specifies the minimum distance between two nodes. Change if necessary.
		every state/.style={thick, fill=gray!10}, % sets the properties for each ’state’ node
		initial text=$ $, % sets the text that appears on the start arrow
	}
	\section[20pt]{p. 90 \#42}
	If the values of the joint probability distribution of X and Y are as shown in the table
	\begin{figure}[ht]
		\centering
		\begin{tikzpicture}
		\node (s1) at (1.5,5.75) {$x$};
		\node (s2) at (0.5,5.25) {$0$};
		\node (s3) at (1.5,5.25) {$1$};
		\node (s4) at (2.5,5.25) {$2$};
		\node (s5) at (-0.5,4.25) {$0$};
		\node (s6) at (-0.5,3.00) {$1$};
		\node (s7) at (-0.5,1.75) {$2$};
		\node (s8) at (-0.5,0.50) {$3$};
		\node (s9) at (-1,2.375) {$y$};
		\node (s10) at (0.5,4.25) {$\frac{1}{12}$};
		\node (s11) at (0.5,3) {$\frac{1}{4}$};
		\node (s12) at (0.5,1.75) {$\frac{1}{8}$};
		\node (s13) at (0.5,0.5) {$\frac{1}{120}$};
		\node (s14) at (1.5,4.25) {$\frac{1}{6}$};
		\node (s15) at (1.5,3) {$\frac{1}{4}$};
		\node (s16) at (1.5,1.75) {$\frac{1}{20}$};
		\node (s17) at (2.5,4.25) {$\frac{1}{24}$};
		\node (s18) at (2.5,3) {$\frac{1}{40}$};
		\draw
		(0,0) edge[]  (3,0)
		(3,0) edge[]  (3,5)
		(3,5) edge[]  (0,5)
		(0,5) edge[]  (0,0);
		\end{tikzpicture}
	\end{figure}
\newline
find
\newline
\textbf{(a)} \(P(X=1,Y=1)\);\ \ \ \ \ \ \ \ \textbf{(b)} \(P(X=0,1\leq Y <3)\); \newline
\textbf{(c)} \(P(X+Y\leq1)\);\ \ \ \ \ \ \ \ \ \ \ \textbf{(d)} \(P(X>Y)\);
\[P(X=1,Y=1)=\frac{1}{4}\]
\[P(X=0,1\leq Y <3)=\frac{1}{4}+\frac{1}{8}=\frac{3}{8}\]
\[P(X+Y\leq1)=\frac{1}{12}+\frac{1}{4}+\frac{1}{6}=\frac{1}{2}\]
\[P(X>Y)=\frac{1}{4}+\frac{1}{8}+\frac{1}{120}+\frac{1}{20}=\frac{13}{30}\]
\boxed{\begin{array}{ccc}
\textbf{(a)} P(X=1,Y=1)=\frac{1}{4}\textup{;\ \ \ \ \ \ } \textbf{(b)} P(X=0,1\leq Y <3)=\frac{3}{8}\\ \textbf{(c)} P(X+Y\leq1)=\frac{1}{2}\textup{;\ \ \ \ \ \ \ \ \ } \textbf{(d)} P(X>Y)=\frac{13}{30} \ \ \ \ \ \ \ \ \ \ \ \ 
\end{array}}
\newpage
\maketitle HW p.90 \#42,44,50,51 \ p.91 \#54,62,68 \ p.100 \#70ab,76
\section[20pt]{p. 90 \#44}
If the joint probability distribution of X and Y is given by
\[f(x,y)=c(x^2+y^2)\ \ \  \textup{for} x=-1,0,1,3;\ \ y=-1,2,3\]
find the value of c. \newline \newline
By Theorem 3.7 we know
\[\sum_x\sum_yf(x,y)=1 \]
\[\sum_x\sum_yc(x^2+y^2)=1\]
\[c\left[3\sum_xx^2+4\sum_yy^2\right]=1\]
\[c[3(1+0+1+9)+4(1+4+9)]=89c=1\]
\boxed{c=\frac{1}{89}}
\newpage
\maketitle HW p.90 \#42,44,50,51 \ p.91 \#54,62,68 \ p.100 \#70ab,76
\section[20pt]{p. 90 \#50}
If the joint probability density of X and Y is given by
\[f(x,y)=\left\{\begin{array}{ccc}
	24xy \textup{ for }0<x<1,0<y<1,x+y<1\\0 \ \ \ \ \ \textup{ elsewhere\ \ \ \ \ \ \ \ \ \ \ \ \ \ \ \ \ \ \ \ \ \ \ \ \ \ \ \ \ \ \ \ }
	\end{array}\right.\]
find \(P(X+Y<\frac{1}{2})\). \newline \newline
\[x+y<\frac{1}{2}\implies y<\frac{1}{2}-x\]
By Def. 3.8 we have
\[P(X+Y<\frac{1}{2})=\int_0^1\int_{0}^{\frac{1}{2}-x}24xy dy dx\]
After computing this integral we have \newline
\boxed{P(X+Y<\frac{1}{2})=\frac{1}{2}}
\newpage
\maketitle HW p.90 \#42,44,50,51 \ p.91 \#54,62,68 \ p.100 \#70ab,76
\section[20pt]{p. 90 \#51}
If the joint probability density of X and Y is given by
\[f(x,y)=\left\{\begin{array}{ccc}
2 \textup{ for }x>0,y>0,x+y<1\\0 \ \textup{ elsewhere\ \ \ \ \ \ \ \ \ \ \ \ \ \ \ \ \ \ \ \ \ }
\end{array}\right.\]
find \newline
\textbf{(a)} \(P(X\leq\frac{1}{2},Y\leq\frac{1}{2})\); \newline
\textbf{(b)} \(P(X+Y>\frac{2}{3})\); \newline
\textbf{(c)} \(P(X>2Y)\). \newline
\newline
By Def. 3.8 we have
\[P(X\leq\frac{1}{2},Y\leq\frac{1}{2})=\int_0^{\frac{1}{2}}\int_0^{\frac{1}{2}}2dydx=\frac{1}{2}\]

\[P(X+Y>\frac{2}{3})=\int_0^{\frac{2}{3}}\int_0^{\frac{2}{3}-x}2dydx=\frac{4}{9}\]

\[P(X>2Y)=\int_0^{\frac{2}{3}}\int_{\frac{1}{2}x}^{1-x}2dydx=\frac{2}{3}\]
\boxed{\begin{array}{ccc}
\textbf{(a)}P(X\leq\frac{1}{2}Y\leq\frac{1}{2})=\frac{1}{2}\\
\textbf{(b)}P(X+Y>\frac{2}{3})=\frac{4}{9} \ \ \\
\textbf{(c)}P(X>2Y)=\frac{2}{3} \ \ \ \ \ \ 
	\end{array}}
\newpage
\maketitle HW p.90 \#42,44,50,51 \ p.91 \#54,62,68 \ p.100 \#70ab,76
\section[20pt]{p. 91 \#54}
Find the joint probability density of the two random variables X and Y whose joint distribution function is given by
\[F(x,y)=\left\{\begin{array}{ccc}
(1-e^{-x^2})(1-e^{-y^2}) \textup{ for }x>0,y>0 \\0 \ \ \ \ \ \ \ \ \ \ \ \ \ \ \ \ \ \ \ \ \ \ \ \ \ \textup{ elsewhere\ \ \ \ \ \ }
\end{array}\right.\]
\newline
By Def. 3.9 we define the joint distribution function to be
\[F(x,y) = \int_{-\infty}^y\int_{-\infty}^xf(s,t)dsdt\]
Thus,
\[f(x,y)=\frac{\partial^2}{\partial x \partial y}F(x,y)\]
\[f(x,y)=\frac{\partial^2}{\partial x \partial y}(1-e^{-x^2})(1-e^{-y^2})\textup{ for }x>0,y>0\]
\[f(x,y)=2xe^{-x^2}2ye^{-y^2}\textup{ for }x>0,y>0\]
\boxed{f(x,y)=\left\{\begin{array}{ccc}
	4xye^{-x^2}e^{-y^2} \textup{ for }x>0,y>0 \\0 \ \ \ \ \ \ \ \ \ \ \ \ \ \ \textup{ elsewhere\ \ \ \ \ \ \ \ }
	\end{array}\right.}
\newpage
\maketitle HW p.90 \#42,44,50,51 \ p.91 \#54,62,68 \ p.100 \#70ab,76
\section[20pt]{p. 91 \#62}
Find k if the joint probability distribution of X, Y, and Z is given by
\[f(x,y,z) = kxyz\]
for \(x=1,2;y=1,2,3;z=1,2\).
\newline
\newline
By Theorem 3.7 we have
\[\sum_{(x,y,z)}f(x,y,z)=1\]
Considering all points and plugging them into our joint probability distribution and the distributive law we have
\[k(1+2)(1+2+3)(1+2)=1\]
\[k*54=1\]
\boxed{k=\frac{1}{54}}
\newpage
\maketitle HW p.90 \#42,44,50,51 \ p.91 \#54,62,68 \ p.100 \#70ab,76
\section[20pt]{p. 91 \#68}
If the joint probability density of X,Y, and Z is given by
\[f(x,y,z)=\left\{\begin{array}{ccc}
\frac{1}{3}(2x+3y+z) \textup{ for }0<x<1,0<y<1,0<z<1\\0 \ \ \ \ \ \ \ \ \ \ \ \ \ \ \ \ \ \ \textup{ elsewhere\ \ \ \ \ \ \ \ \ \ \ \ \ \ \ \ \ \ \ \ \ \ \ \ \ \ \ \ \ \ \ \ }
\end{array}\right.\]
find \newline
\textbf{(a)} \(P(X=\frac{1}{2},Y=\frac{1}{2},Z=\frac{1}{2})\); \newline
\textbf{(b)} \(P(X<\frac{1}{2},Y<\frac{1}{2},Z<\frac{1}{2})\). \newline
\newline
Because X,Y, and Z are continuous-values random variables, the probability X,Y or Z takes on a specific value is 0.
\[P((x,y,z)=(\frac{1}{2},\frac{1}{2},\frac{1}{2}))=0\]
By Definition 3.8 we have
\[P(X<\frac{1}{2},Y<\frac{1}{2},Z<\frac{1}{2})=\int_0^\frac{1}{2}\int_0^\frac{1}{2}\int_0^\frac{1}{2}\frac{1}{3}(2x+3y+z)dxdydz\]
After computation we have
\[P(X<\frac{1}{2},Y<\frac{1}{2},Z<\frac{1}{2})=\frac{1}{16}\]
\boxed{\begin{array}{ccc}
		\textbf{(a)}P(X=\frac{1}{2},Y=\frac{1}{2},Z=\frac{1}{2})=0 \ \ \\
		\textbf{(b)}P(X<\frac{1}{2},Y<\frac{1}{2},Z<\frac{1}{2})=\frac{1}{16}
\end{array}}
\newpage
\maketitle HW p.90 \#42,44,50,51 \ p.91 \#54,62,68 \ p.100 \#70ab,76
\section[20pt]{p. 100 \#70ab}
With reference to Exercise 3.42 on page 90, find \newline
\textbf{(a)} the marginal distribution of X; \newline
\textbf{(b)} the marginal distribution of Y; \newline
\newline
By Definition 3.10 the marginal distributions of X and Y are given by
\[g(x)=\sum_yf(x,y),\ h(y)=\sum_xf(x,y)\]
where \(g(x)\) and \(h(y)\) are the marginal distributions of X and Y respectively. \newline
In Exercise 3.42, we were given a PDF in the form of a table. To calculate marginal densities using Def. 3.10, the sums of the columns and of the rows correspond to the marginal distributions of X and Y respectively.

\boxed{g(x)=\left\{\begin{array}{ccc}

		\frac{7}{15} \textup{ for }x=0\\
\frac{7}{15} \textup{ for }x=1\\
\frac{1}{15} \textup{ for }x=2\\0 \ \textup{ elsewhere }	\end{array}\right.} \newline \newline

\boxed{h(y)=\left\{\begin{array}{ccc}
		
		\frac{7}{24} \textup{ for }y=0\\
		\frac{21}{40} \textup{ for }y=1\\
		\frac{7}{40} \textup{ for }y=2\\
		\frac{1}{120} \textup{ for }y=3\\0 \ \textup{ elsewhere }	\end{array}\right.}
\newpage
\maketitle HW p.90 \#42,44,50,51 \ p.91 \#54,62,68 \ p.100 \#70ab,76
\section[20pt]{p. 100 \#76}
If the joint probability density of X and Y is given by
\[f(x,y)=\left\{\begin{array}{ccc}
24y(1-x-y) \textup{ for }x>0,y>0,x+y<1\\0 \ \ \ \ \ \ \ \ \ \ \ \ \ \ \ \ \ \ \ \textup{elsewhere}\ \ \ \ \ \ \ \ \ \ \ \ \ \ \ \ \ \ \ \ \
\end{array}\right.\]
find \newline
\textbf{(a)} the marginal density of X; \newline
\textbf{(b)} the marginal density of Y. \newline
Also determine whether the two random variables are independent. \newline \newline
By Def. 3.11
\[g(x)=\int_0^{1-x}24y(1-x-y)dy\]
\boxed{\textbf{(a)}\ g(x)=4(1-x)^3,\ 0<y<1}
\newline
\newline
By Def. 3.11
\[h(y)=\int_0^{1-y}24y(1-x-y)dx\]
\boxed{\textbf{(b)}\ h(y)=12y(1-y)^2,\ 0<x<1} \newline \newline
By Def. 3.14, random variables X and Y are independent if and only if \(f(x,y)=g(x)h(y)\). \newline \newline
But \(f(x,y)\neq g(x)h(y)\) \newline
\boxed{\textup{R.V.s X and Y are not independent.}}

\end{document}