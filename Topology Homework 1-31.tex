\documentclass[12pt]{article}
\usepackage[utf8]{inputenc}
\usepackage{amsmath}
\usepackage{amsfonts}
\usepackage{amssymb}
\usepackage{empheq}
\usepackage{tikz}
\usepackage{enumitem}
\usepackage{eucal}
\usepackage{changepage}
\usetikzlibrary{automata, positioning, shapes}
\addtolength{\topmargin}{-0.75in}
\addtolength{\textheight}{1.75in}
\addtolength{\evensidemargin}{-0.5in}

\title{Topology Homework 02}
\author{Ethan Jensen, Luke Lemaitre, Kasandra Lassagne}
\date{January 31, 2020}

\begin{document}
	\maketitle
	\noindent
	\textbf{EXERCISE 1.1.} \textit{Determine all of the possible topologies on} \(X = \{a,b\}\). \newline \newline
	\(\mathcal{T}_1=\{\varnothing,\ X\}\) \newline
	\(\mathcal{T}_1=\{\varnothing,\ \{a\},\ X\}\) \newline
	\(\mathcal{T}_1=\{\varnothing,\ \{b\},\ X\}\) \newline
	\(\mathcal{T}_1=\{\varnothing,\ \{a\},\ \{a,b\},\ X\}\) \newline
	\(\mathcal{T}_1=\{\varnothing,\ \{b\},\ \{a,b\},\ X\}\) \newline
	\(\mathcal{T}_1=\{\varnothing,\ \{a\},\ \{b\},\ \{a,b\},\ X\}\) \newline
	\newpage
	\noindent
	\textbf{EXERCISE 1.3.} \textit{Prove that a topology} \(\mathcal{T}\) \textit{on X is the discrete topology if and only if} \(\{x\}\in\mathcal{T}\) \textit{for all} \(x \in X\). \newline \newline
	\textbf{Proof.} \newline
	(\(\rightarrow\)) Assume \(\mathcal{T}\) is the discrete topology of X. \newline
	Since \(x \in X\), \(\{x\} \subseteq X\) \newline
	\(\therefore \{x\}\in\mathcal{T}\) \textit{for all} \(x \in X\)
	\newline \(\square\) \newline
	(\(\leftarrow\)) Assume \(\{x\}\in\mathcal{T}\) \textit{for all} \(x \in X\) \newline
	Let A a set such that \(A \subseteq X\).
	for every point \(x \in A\), we can find a set \(x \in \{x\}\in\mathcal{T}\). \newline
	By the Union lemma we can say \(A = \bigcup_{x \in A} \{x\}\) \newline
	Since \(\mathcal{T}\) is a topology, any union of open sets in \(\mathcal{T}\) is an open set in \(\mathcal{T}\). \newline \newline
	This means A is an open set in \(\mathcal{T}\). \newline
	Since every subset A is an open set in X, \(\mathcal{T}\) is the discrete topology.
	\newline \(\square\) \newline
	\(\therefore\) A topology \(\mathcal{T}\) on X is the discrete topology if and only if \(\{x\}\in\mathcal{T}\) for all \(x \in X\). \newline \(\blacksquare\) \newline
	\newpage
	\noindent
	\textbf{EXERCISE 1.4.} \newline
	\textbf{(a)} \textit{Give an example of a space where the discrete topology is the same as the finite complement topology.} \newline
	\textbf{(b)} \textit{Make and prove a conjecture indicating for what class of sets the discrete and finite complement topologies coincide.}
	\newline \newline
	\textbf{(a)} In the set \(\varnothing\), both the discrete and finite complement topologies are \(\{\varnothing\}\).
	\newline \newline
	\textbf{(b)} Conjecture: The discrete and finite complement topologies for a set are equivalent if and only if the set is finite.
	\newline \newline
	\textbf{Proof. } \newline
	(\(\rightarrow\)) Assume the discrete and finite complement topologies for a set X are equivalent. \newline \newline
	Consider some \(A \subseteq X\) \newline
	\(A' \in \mathcal{T}\) since \(\mathcal{T}\) is the discrete topology. \newline
	Thus, \(A'\) have a finite complement. \newline
	Thus A is finite. \newline
	Since every subset of X is finite, X is finite.
	\newline \(\square\) \newline \newline
	(\(\leftarrow\)) Let X be some finite set. \newline \newline
	Consider some \(A \subseteq X\). \newline
	Obviously, \(A' \subseteq X\). \newline
	Since \(A'\) is a subset of a finite set, \(A'\) is finite. \newline
	Thus, \(A\) has a finite complement and is in the finite complement topology. \newline
	Since every \(A \subseteq X\) is in the finite complement topology and the discrete topology on X, the finite complement topology and the discrete topology are equivalent for the set X.
	\newline \(\square\) \newline
	\(\therefore\) The discrete and finite complement topologies for a set are equivalent if and only if the set is finite. \newline
	\(\blacksquare\)
	\newpage
	\noindent
	\textbf{EXERCISE 1.5.} \textit{Find three topologies on the five point set} \(X=\{a,b,c,d,e\}\) \textit{such that the first is strictly finer than the second and the second strictly finer than the third without using either the trivial or the discrete topology. Find a topology on X that is not comparable to each of the first three you found.}
	\newline \newline
	Let \(\mathcal{T}_1=\{\varnothing, \{a\}, \{a,b\}, \{a,b,c\}, X\}\)
	\newline
	Let \(\mathcal{T}_2=\{\varnothing, \{a\}, \{a,b\}, X\}\)
	\newline
	Let \(\mathcal{T}_3=\{\varnothing, \{a\}, X\}\)
	\newline \newline
	It is easy to see that \(\mathcal{T}_3 \subseteq \mathcal{T}_2 \subseteq \mathcal{T}_1\). \newline
	Thus, \(\mathcal{T}_1\) is strictly finer than \(\mathcal{T}_2\) which is strictly finer than \(\mathcal{T}_3\).
	\newline \newline
	Let \(\mathcal{T}_q=\{\varnothing, \{b\}, X\}\) \newline \newline
	Each \(\mathcal{T}_1,\mathcal{T}_2,\mathcal{T}_3\) contain \(\{a\}\), but \(\{a\} \notin \mathcal{T}_q\). \newline
	Each \(\mathcal{T}_1,\mathcal{T}_2,\mathcal{T}_3\) does not contain \(\{b\}\), but \(\{b\} \in \mathcal{T}_q\). \newline \newline
	Thus, \(\mathcal{T}_q\) does not compare to the other topologies.
	\newline
	\newpage
	\noindent
	\textbf{EXERCISE 1.6.} \textit{Define a topology on }\(\mathbb{R}\)\textit{ (by listing the open sets within it) that contain the open sets} (0,2) \textit{and} (1,3) \textit{and that contain as few open sets as possible.} \newline \newline
	\(\mathcal{T}=\{\varnothing, (0,2), (1,3), (1,2), (0,3), \mathbb{R}\}\)
	\newpage
	\noindent
	\textbf{EXERCISE 1.7} \textit{Let X be a set and assume }\(p \in X\) \textit{. Show that the collection} \(\mathcal{T}\)\textit{, consisting of} \(\varnothing\) \textit{and all subsets of X containing p, is a topology on X. This topology is called the \textbf{particular point topology} on X, and we denote it by }\(PPX_p\). \newline \newline
	\textbf{Proof. } \newline
	\textbf{(1)} By definition, \(\varnothing \in PPX_p\). \newline
	Additionally, \(p \in X\) so \(X \in PPX_p\).
	\newline \(\square\) \newline \newline
	\textbf{(2)} Let \(U_1,U_2,...U_n\) be some finite number of open sets in \(PPX_p\). \newline
	Consider \(\bigcap_{i=1}^n U_i\). \newline \newline
	Each \(U_i \subseteq X\) so \(\bigcap_{i=1}^n U_i \subseteq X\) \newline
	Since each \(U_i\) is an open set in \(PPX_p\) and \(p \in U_i\  \forall i\) we know \(p \in \bigcap_{i=1}^n U_i\)
	Thus, \(\bigcap_{i=1}^n U_i\) is an open set in \(PPX_p\).
	\newline \(\therefore \) Any finite intersections of open sets in \(PPX_p\) is open in \(PPX_p\).
	\newline \(\square\) \newline \newline
	\textbf{(3)} Let \(U_1, U_2,U_3...\) be some number of open sets in \(PPX_p\). \newline
	Consider \(\bigcup U_i\). \newline \newline
	Each \(U_i \subseteq X\) so \(\bigcup U_i \subseteq X\). \newline
	Since \(U_1\) is an open set in \(PPX_p\), \(p \in U_1\)
	\newline
	Thus, \(p \in \bigcup U_i\), which is open in \(PPX_p\). \newline
	\(\therefore\) Any union of open sets in \(PPX_p\) is open in \(PPX_p\).
	\newline \(\square\) \newline
	By definition, \(PPX_p\) follows all three rules of being a topology. \newline \(\therefore PPX_p\) is a topology on X.
	\newline \(\blacksquare\) \newpage
	\noindent
	\textbf{EXERCISE 1.8} \textit{Let X be a set and assume }\(p\in X\)
	\textit{. Show that the collection} \(\mathcal{T}\) \textit{consisting of X and all subsets of X that exclude p,
	is a topology on X. This topoogy is called the \textbf{excluded point topology} on X, and we denote it by} \(EPX_p\).
	\newline \newline
	\textbf{Proof. } \newline
	\textbf{(1)} By definition, \(X \in EPX_p\). \newline
	Additionally, \(p \notin \varnothing\) so \(\varnothing \in EPX_p\).
	\newline \(\square\) \newline \newline
	\textbf{(2)} Let \(U_1, U_2,...U_n\) be some finite number of open sets in \(EPX_p\). \newline
	Consider \(\bigcap_{i=1}^n U_i\). \newline \newline
	Each \(U_i \subseteq X\) so \(\bigcap_{i=1}^n U_i \subseteq X\) \newline
	Additionally, since \(p \notin U_1\), it is certainly the case that \(p \notin \bigcap_{i=1}^n U_i\). \newline
	Thus, \(\bigcap_{i=1}^n U_i\) is an open set in \(EPX_p\). \newline
	\(\therefore \) Any finite intersection of open sets in \(EPX_p\) is open in \(EPX_p\).
	\newline \(\square\) \newline \newline
	\textbf{(3)} Let \(U_1, U_2,U_3...\) be some number of open sets in \(EPX_p\). \newline
	Consider \(\bigcup U_i\). \newline \newline
	Each \(U_i \subseteq X\) so \(\bigcup U_i \subseteq X\) \newline
	There is no set \(U_i\) such that  \(p \in U_i\), so \(p \notin \bigcup U_i\)\newline
	Thus, \(\bigcup U_i\) is an open set in \(EPX_p\). \newline
	\(\therefore \) Any union of open sets in \(EPX_p\) is open in \(EPX_p\).
	\newline \(\square\) \newline
	By definition, \(EPX_p\) follows all three rules of being a topology.
	\newline \(\therefore EPX_p\) is a topology on X.
	\newline \(\blacksquare\) \newpage
\end{document}
