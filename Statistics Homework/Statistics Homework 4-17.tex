\documentclass[12pt]{article}
\usepackage[utf8]{inputenc}
\usepackage{amsmath}
\usepackage{amsfonts}
\usepackage{amssymb}
\usepackage{empheq}
\usepackage{graphicx}
\usepackage{tikz}
\usepackage{changepage}
\usetikzlibrary{automata, positioning, arrows, shapes}
\addtolength{\topmargin}{-0.875in}
\addtolength{\textheight}{1.75in}
\title{Math 332 A - Mathematical Statistics}
\author{Ethan Jensen}
\date{April 17, 2020}
\begin{document}
\maketitle HW p.380 \#80,\#82\ \ p.378 \#42,\ \ p.448 \#18
	\section[20pt]{p. 380 \#80}
Four coins were tossed 160 times and 0,1,2,3, or 4 heads showed, respectively, 19, 54, 58, 23, and 6 times. Use the 0.05 level of significance te suppose that the coins are balanced and randomly tossed.
\newline \newline
\(H_0: \textup{The population is binomial with } \theta = 0.5, n = 4\).
\newline
\(H_1: \textup{The population is not binomial with } \theta = 0.5, n = 4\).
\newline
\(\alpha = 0.05\)
\newline
\begin{tabular}{|p{3cm}|p{2cm}|}
    \hline
    \textbf{Num. heads} & \textbf{Totals} \\
    \hline
    \(0\) & 19 \\
    \hline
    \(1\) & 54 \\
    \hline
    \(2\) & 58 \\
    \hline
    \(3\) & 23 \\
    \hline
    \(4\) & 6 \\
    \hline
\end{tabular}
\newline
The expected values for a binomial population with \(\theta = 0.5, n = 4\) and a sample size of 160 is:
\newline
\begin{tabular}{|p{3cm}|p{2cm}|}
    \hline
    \textbf{Num. heads} & \textbf{Totals} \\
    \hline
    \(0\) & 10 \\
    \hline
    \(1\) & 40 \\
    \hline
    \(2\) & 60 \\
    \hline
    \(3\) & 40 \\
    \hline
    \(4\) & 10 \\
    \hline
\end{tabular}
\newline
\(\sum_{i=0}^4\frac{(f_i-e_i)^2}{e_i} = 18.89167\)
\newline
However, the value of \(\chi^2_{\alpha = 0.05, 4} = 9.488\), which is less than our sum.
\newline
Thus, we must reject the null hypothesis.
\newline
\newline
\boxed{\textup{The coins are not equally weighted.}}
\newpage
	\section[20pt]{p. 380 \#82}
Each day, Monday through Saturday, a baker bakes three large chocolate cakes, and those not sold on the same day are given away to a food bank. Use the data shown in the following table to test at the 0.05 level of significance whether they may be looked upon as values of a binomial random variable:
\newline
\begin{tabular}{|p{2.5cm}|p{2cm}|}
    \hline
    \textbf{Number of cakes sold} & \textbf{Number of days} \\
    \hline
    \(0\) & 1 \\
    \hline
    \(1\) & 16 \\
    \hline
    \(2\) & 55 \\
    \hline
    \(3\) & 228 \\
    \hline
\end{tabular}
\newline
The total number of days is 300. The toal number of cakes sold is 810.
\newline
Thus, the expected number of cakes sold per day is 2.7 cakes.
\newline
Thus, we test the population against a binomial population with \(\theta = 2.7/3 = 0.9\).
\newline
The expected values for a binomial population with \(\theta = 0.0, n = 3\) and a sample size of 300 is:
\newline
\begin{tabular}{|p{2.5cm}|p{2cm}|}
    \hline
    \textbf{Number of cakes sold} & \textbf{Number of days} \\
    \hline
    \(0\) & 0.3 \\
    \hline
    \(1\) & 8.1 \\
    \hline
    \(2\) & 72.9 \\
    \hline
    \(3\) & 218.7 \\
    \hline
\end{tabular}
\newline
\(\sum_{i=0}^3\frac{(f_i-e_i)^2}{e_i} = 14.12894019\)
\newline
However, the value of \(\chi^2_{\alpha = 0.05, 3} = 7.815\), which is less than our sum.
\newline
Thus, we must reject the null hypothesis.
\newline
\newline
\boxed{\textup{The cakes sold are not binomially distributed.}}
\newpage
  \section[20pt]{p. 378 \#42 (part 1)}
To compare two kinds of front-end designs, six of each kind were installed on a certain make of compact car. Then each car was run into a concrete wall at 5 miles per hour, and the following are the costs of repairs (in dollars):
\newline \newline
\ \ \ \ \textit{Design 1:}\ 127\ \ 168\ \ 143\ \ 165\ \ 122\ \ 139
\newline
\ \ \ \ \textit{Design 2:}\ 154\ \ 135\ \ 132\ \ 171\ \ 153\ \ 149
\newline \newline
Use the four steps of page 354 to test at the 0.01 level of significance whether the difference between means of these two samples is significant.
\newline \newline
\(H_0: \mu_1 = \mu_2\)
\newline
\(H_1: \mu_1 \neq \mu_2\)
\newline
\(\alpha = 0.01\)
\newline
\newline
By some simple calculations,
\(\overline{x}_1 = 144,\ \overline{x}_2 = 149,\ s_1^2 = 302.67,\ s_2^2 = 168.33,\ s_p^2 = 235.5\)
\newline
We will perform a two tailed test with t as our test statistic.
\[t = \frac{\overline{x}_2 - \overline{x}_2}{s_p\sqrt{\frac{1}{n_1}+\frac{1}{n_2}}} = \frac{149 - 144}{15.346\sqrt{\frac{1}{6}+\frac{1}{6}}} = 0.564332\]
\[t_{0.005,10}=3.169\]
Since \(0.564332 < 3.169\), we accept the null hypothesis.
\newpage
  \section[20pt]{p. 378 \#42 (part 2)}
\ \ \ \ \textit{Design 1:}\ 127\ \ 168\ \ 143\ \ 165\ \ 122\ \ 139
\newline
\ \ \ \ \textit{Design 2:}\ 154\ \ 135\ \ 132\ \ 171\ \ 153\ \ 149
\newline \newline
\(H_0: \mu_1 = \mu_2\)
\newline
\(H_1: \mu_1 \neq \mu_2\)
\newline
\(\alpha = 0.01\)
\newline
\newline
By some simple calculations,
\(\overline{x}_1 = 144,\ \overline{x}_2 = 149,\ SS(Tr) = 75, SSE = 2826\)
\newline
\(MS(Tr) = 75.\ MSE = 282.6\)
\newline
We will perform a one tailed test with f as our test statistic.
\[f = \frac{MS(Tr)}{MSE} = 0.26539\]
\[f_{0.01,1,5}=16.3\]
Since \(0.26539 < 16.3\), we accept the null hypothesis.
\newpage
  \section[20pt]{p. 448 \#18}
  Three groups of siz guinea pigs each were injected, respectively, with 0.5 milligram, 1.0 milligram, and 1.5 milligrams of a new tranquilizer, and the following are the number of minutes it took them to fall asleep:
  \newline \newline
  0.5 mg: 21, 23, 19, 24, 25, 23
  \newline
  1.0 mg: 19, 21, 20, 18, 22, 20
  \newline
  1.5 mg: 15, 10, 13, 14, 11, 15
  \newline \newline
  Test at the 0.05 level of significance whether the null hypothesis that differences in dosage have no effect can be rejected. Also, estimate the parameters \(\mu, \alpha_1, \alpha_2\), and \(\alpha_3\) of the model used in the analysis.
  \newline \newline
  I finally decided to use an online ANOVA calculator.
  \newline
  The result from the calculator gives a test-statistic value of
  \(f_{0.05, 2, 15} = 39.32432\), which has an extremely small p-value (less than 0.0001).
  \newline
  The null hypothesis can certainly be rejected since the p-value is smaller than 0.05.
  \newline
  \newline
  Also, by some simple calculations -
  \[\mu = \overline{\overline{x}} = 18.5\]
  \[\alpha_1 = \overline{x}_1-\mu = 4\]
  \[\alpha_2 = \overline{x}_2-\mu = 1.5\]
  \[\alpha_3 = \overline{x}_3-\mu = -5.5\]
\end{document}
