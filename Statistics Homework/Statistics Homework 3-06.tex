\documentclass[12pt]{article}
\usepackage[utf8]{inputenc}
\usepackage{amsmath}
\usepackage{amsfonts}
\usepackage{amssymb}
\usepackage{empheq}
\usepackage{graphicx}
\usepackage{tikz}
\usepackage{changepage}
\usetikzlibrary{automata, positioning, arrows, shapes}
\addtolength{\topmargin}{-0.875in}
\addtolength{\textheight}{1.75in}
\title{Math 332 A - Mathematical Statistics}
\author{Ethan Jensen}
\date{March 6, 2020}

\begin{document}
	\maketitle HW p.320 \#4,9 \ p.328 \#22,24,34,36
  \section[20pt]{p. 320 \#4}
  Show that the \((1-\alpha)100\%\) confidence interval
	\[\overline{x} - z_{\alpha/2}\cdot\frac{\sigma}{n}< \mu < \overline{x} + z_{\alpha/2}\cdot\frac{\sigma}{n}\]
	is shorter than the \((1-\alpha)100\%\) confidence interval
	\[\overline{x} - z_{2\alpha/3}\cdot\frac{\sigma}{n}< \mu < \overline{x} + z_{\alpha/3}\cdot\frac{\sigma}{n}\]
	\newline
	Let f(x) be the standard normal distribution. We know
	\[\int_{z_{2\alpha/3}}^{(z_{\alpha/3}+z_{2\alpha/2})/2}f(x) dx + \int_{(z_{\alpha/3}+z_{2\alpha/2})/2}^{z_{\alpha/3}} f(x) dx = \int_{z_{2\alpha/3}}^{z_{\alpha/2}} f(x) dx + \int_{z_{\alpha/2}}^{z_{\alpha/3}} f(x) dx\]
	\[\int_{z_{2\alpha/3}}^{z_{\alpha/2}} f(x) dx = \int_{z_{\alpha/2}}^{z_{\alpha/3}} f(x) dx = \frac{\alpha}{6}\]
	Since \(z_{2\alpha/3} < z_{\alpha/3}\) and f(x) is decreasing on \((0, \infty)\), we know
	\[\int_{z_{2\alpha/3}}^{(z_{\alpha/3}+z_{2\alpha/2})/2}f(x) dx > \int_{(z_{\alpha/3}+z_{2\alpha/2})/2}^{z_{\alpha/3}} f(x) dx\]
	Thus, making a substitution into the first equation we have
	\[2\int_{z_{2\alpha/3}}^{(z_{\alpha/3}+z_{2\alpha/2})/2}f(x) dx > 2\int_{z_{2\alpha/3}}^{z_{\alpha/2}} f(x) dx\]
	Since the integrals have the same lower bounds, and f(x) is always positive,
	\[(z_{\alpha/3}+z_{2\alpha/2})/2 > z_{\alpha/2}\]
	\[z_{\alpha/2} + z_{\alpha/2} < z_{\alpha/3}+z_{2\alpha/2}\]
	\boxed{\therefore \textup{The first confidence interval is shorter than the second.}}
	\newpage
	\section[20pt]{p. 320 \#9}
	Show that \(S_p^2\) is an unbiased estimator of \(\sigma^2\) and find its variance under the conditions of Theorem 11.5.
	\newline \newline
	The pooled estimator of \(\sigma^2\) is defined as follows:
	\[S_p^2 = \frac{(n_1 -1)S_1^2 + (n_2 - 1)S_2^2}{n_1 + n_2 - 2}\]
	Because Expected Value is a linear operator,
	\[E(S_p^2) = \frac{1}{n_1+n_2-2}[(n_1-1)E(S_1^2) + (n_2-1)E(S_2^2)]\]
	Since \(S_1^2\) and \(S_2^2\) are both unbiased estimators of \(\sigma^2\),
	\[E(S_p^2) = \frac{1}{n_1+n_2-2}[(n_1-1)\sigma^2 + (n_2-1)\sigma^2] = \frac{n_1 -1 + n_2 - 1}{n_1+n_2-2}\sigma^2\]
	\[E(S_p^2) = \sigma^2\]
	\boxed{\textup{So }\(E(S_p^2)\)\textup{ is an unbiased estimator of }\(\sigma^2\).}
	\newline \newline
	By Theorem 8.11, under the conditions of Theorem 11.5
	\[\frac{(n_1-1)S_1^2}{\sigma^2} \textup{ and } \frac{(n_2-1)S_2^2}{\sigma^2}\] have chi-square distirbutions with respctively \(n_1 -1\) and \(n_2 -1\) degrees of freedom.
	\newline \newline
	Using Theorem 8.9,
	\[X = \frac{(n_1+n_2-2)S_p^2}{\sigma^2}\]
	has a chi-square distribution with \(n_1+n_2-2\) degrees of freedom.
	\newline \newline
	This means \(S_p^2\) is a linear combination of a chi-square random variable and Corollary 4.3 applies.
	\newline \newline
	\[\textup{var}(S_p^2) = \left(\frac{\sigma^2}{n_1+n_2-2}\right)^2\textup{var}(X)\]
	Using Corollary 6.2 we have
	\[\textup{var}(S_p^2) = \left(\frac{\sigma^2}{n_1+n_2-2}\right)^22(n_1+n_2-2)\]
	\boxed{\textup{var}(S_p^2) = \frac{2\sigma^4}{n_1+n_2-2}}
	\newpage
	\section[20pt]{p. 328 \#22}
	A medical research worker intends to use the mean of a random sample of size \(n=120\) to estimate the mean blood pressure of women in their fifties. If, based on experience, he knows that \(\sigma = 10.5\)mm of mercury, what can he assert with probability 0.99 about the maximum error?
	\newline \newline
	By Theorem 11.1, the maximum error is less than
	\[z_{\alpha/2}\cdot \frac{\sigma}{\sqrt{n}}\]
	With Table III, we find that \(z_{\alpha/2} = 2.575\).
	\newline
	Plugging in \(n=120, \sigma = 10.5\), we have that our maximum error is equal to \(2.575\cdot \frac{10.5}{\sqrt{120}} \approx 2.468\).
	\newline \newline
	\boxed{\textup{with probability 99\%, the maximum error is about } 2.468 \textup{mm of mercury.}}
	\newpage
	\section[20pt]{p. 328 \#24}
	A study of the annual growth of certain cacti showed that 64 of them, selected at random in a desert region, grew on average 52.80mm with a standard deviation of 4.5mm. Construct a 99\% confidence interval for the true average annual growth of the given kind of cactus.
	\newline \newline
	With Theorem 11.1, the maximum error is less than
	\[z_{\alpha/2}\cdot \frac{\sigma}{\sqrt{n}}\]
	With Table III, we find that \(z_{\alpha/2} = 2.575\).
	\newline
	Plugging in \(\mu = 52.80, n = 64, \sigma = 4.5\), we can construct the confidnece interval.
	\(52.80 \pm 2.575\cdot \frac{4.5}{\sqrt{64}} = 51.35 \textup{ and } 54.25\) respectively.
	\newline \newline
	\boxed{\textup{The  99\% confidence interval of the mean is } (51.35, 54.25) \textup{ mm growth.}}
	\newpage
	\section[20pt]{p. 328 \#34}
	A study of two kinds of photocopying equipment shows that 61 failures of the first kind of equipment took on the average 80.7 minutes to repair with a standard deviation of 19.4 minutes, whereas 61 failures of the second kind of equipment took on the average 88.1 minutes to repair with a standard deviation of 18.8 minutes. Find a 99\% confidence interval for the difference between the true average amounts of time it takes to repair failures of the two kinds of photocopying equipment.
	\newline \newline
	With Theorem 11.4 we can deduce the 99\% confidence interval.
	\newline
	Plugging in \(\mu_1 = 80.7, \mu_2 = 88.1, n_1 = 61, n_2 = 61, \sigma_1 = 19.4, \sigma_2 = 18.8\)
	\newline
	And using our spicy value \(z_{\alpha/2} = 2.575\) we have
	\[E = 2.575\cdot \sqrt{\frac{19.4^2}{61}+\frac{18.8^2}{61}} = 8.907\]
	\[\mu_1 - \mu_2 = 7.4\]
	\newline
	\boxed{\textup{The 99\% confidence interval for the true difference of the means is } (0, 16.307)}
	\newline \newline
	\textit{Note that I'm forcing the difference to be a positive value.}
	\newpage
	\section[20pt]{p. 328 \#36}
	The following are the heat-producing capacities of coal from two mines (in millions of calories per ton):
	\begin{adjustwidth}{1cm}{0cm}
    \begin{flushleft}
      \textit{Mine A: }\ 8,500,\ \ \ 8,330,\ \ \ 8,480,\ \ \ 7,960,\ \ \ 8,030 \newline
			\textit{Mine B: }\ 7,710,\ \ \ 7,890,\ \ \ 7,920,\ \ \ 8,270,\ \ \ 7,860
    \end{flushleft}
  \end{adjustwidth}
\(\ \)
\noindent
Assuming that the data constitute independent random samples from normal populations with equal variances, construct a 99\% confidence interval for the difference between the true average heat-producing capacities of coal from the two mines.
\newline \newline
Since the two populations have equal variances, we can use Theorem 11.5. to construct the confidence interval.
\newline \newline
First, we have to calculate \(\overline{x}_1\) and \(\overline{x}_2\), which are 8260 and 7930 respectively.
\newline
Next, we calculate \(S_1^2\) and \(S_2^2\), which are 63450 and 42650 respectively.
\newline
Then, we calculate the pooled variance \(S_p\) which is 205.03.
\newline \newline
Then, we look up the value of \(t_{0.005,8}\) in Table 4 which is 3.355.
\newline \newline
Using Theorem 11.5 we calculate the confidence interval which is \((0, 816.40)\).
\newline
\boxed{\textup{The 99\% confidence interval for the true difference of the means is } (0, 816.40)}
\newline \newline
\textit{Note that I'm forcing the difference to be a positive value.}
\end{document}
