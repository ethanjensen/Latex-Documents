\documentclass[12pt]{article}
\usepackage[utf8]{inputenc}
\usepackage{amsmath}
\usepackage{amsfonts}
\usepackage{amssymb}
\usepackage{empheq}
\usepackage{graphicx}
\usepackage{tikz}
\usepackage{changepage}
\usetikzlibrary{automata, positioning, arrows, shapes}
\addtolength{\topmargin}{-0.875in}
\addtolength{\textheight}{1.75in}
\title{Math 332 A - Mathematical Statistics}
\author{Ethan Jensen}
\date{April 10, 2020}

\begin{document}
	\maketitle HW p.370 \#8,\ \ p.379 \#74,\ \ p.380 \#78
	\section[20pt]{p. 370 \#8}
	Show that the two formulas for \(\chi^2\) on pages 368 and 369 are equivalent.
	\newline
	\newline
	We first define the following terms:
	\newline
	\(f_{i1} = x_i\), and \(f_{i2} = n_i - x_i\)
	\newline
	\(e_{i1} = n_i\hat{\theta}_i\), and \(e_{i2} = n_i(1-\hat{\theta}_i)\)
	\newline
	\newline
	Let us begin.
	\[\sum_{i=1}^k\sum_{j=1}^2\frac{(f_{ij}-e_{ij})^2}{e_{ij}} = \sum_{i=1}^k\sum_{j=1}^2\frac{f_{ij}^2-2e_{ij}f_{ij}+e_{ij}^2}{e_{ij}} = \sum_{i=1}^k\sum_{j=1}^2\frac{f_{ij}^2}{e_{ij}} -2f_{ij} +e_{ij}\]
	\[\sum_{i=1}^k\sum_{j=1}^2\frac{(f_{ij}-e_{ij})^2}{e_{ij}} = \sum_{i=1}^k\frac{f_{i1}^2}{e_{i1}} + \frac{f_{i2}^2}{e_{i2}} - 2(f_{i1} + f_{i2}) +(e_{i1} + e_{i2})\]
	\[\sum_{i=1}^k\sum_{j=1}^2\frac{(f_{ij}-e_{ij})^2}{e_{ij}} = \sum_{i=1}^k\frac{x_i^2}{n_i\hat{\theta}_i}+\frac{(n_i-x_i)^2}{n_i(1-\hat{\theta}_i)}-2n_i + n_i\]
	\[\sum_{i=1}^k\sum_{j=1}^2\frac{(f_{ij}-e_{ij})^2}{e_{ij}} = \sum_{i=1}^k\frac{x_i^2(1-\hat{\theta}_i) + (n_i-x_i)^2\hat{\theta}_i-n_i^2\hat{\theta}_i(1-\hat{\theta}_i)}{n_i\hat{\theta}_i(1-\hat{\theta}_i)}\]
	\[\sum_{i=1}^k\sum_{j=1}^2\frac{(f_{ij}-e_{ij})^2}{e_{ij}} = \sum_{i=1}^k\frac{x_i^2 - x_i^2\hat{\theta}_i + n_i^2\hat{\theta}_i - 2n_ix_i\hat{\theta}_i + x_i^2\hat{\theta}_i - n_i^2\hat{\theta}_i + n_i^2\hat{\theta}_i^2}{n_i\hat{\theta}_i(1-\hat{\theta}_i)}\]
	\[\sum_{i=1}^k\sum_{j=1}^2\frac{(f_{ij}-e_{ij})^2}{e_{ij}} = \sum_{i=1}^k\frac{x_i^2 - 2n_ix_i\hat{\theta}_i + n_i^2\hat{\theta}_i^2}{n_i\hat{\theta}_i(1-\hat{\theta}_i)} = \sum_{i=1}^k\frac{(x_i-n_i\hat{\theta}_i)^2}{n_i\hat{\theta}_i(1-\hat{\theta}_i)}\]
	\boxed{\chi^2 = \sum_{i=1}^k\sum_{j=1}^2\frac{(f_{ij}-e_{ij})^2}{e_{ij}} = \sum_{i=1}^k\frac{(x_i-n_i\hat{\theta}_i)^2}{n_i\hat{\theta}_i(1-\hat{\theta}_i)}}
	\newline \(\blacksquare\)
	\newpage
	\section[20pt]{p. 379 \#74}
	In random samples of 250 people with low incolmes, 200 people with average incomes, and 150 people with high incomes, there were, respectively, 155, 118, and 87 who favor a certain piece of legislation. Use the 0.05 level of significance to test the null hypothesis \(\theta_1 = \theta_2 = \theta_3\) (that the proportion of the people favoring the legislation is the same for all three income groups) against the alternative hypothesis that the three \(\theta\)'s are not all equal.
	\newline \newline
	Constructing our table, we have
	\newline
	\begin{tabular}{|p{3.3cm}|p{2cm}|p{2cm}|p{2cm}|}
			\hline
			\textit{Data (\(f_{ij}\))} & \textbf{Favors} & \textbf{Does not Favor} & \textbf{Totals} \\
	 		\hline
			\textbf{Low Income} & 155 & 95 & 250 \\
			\hline
			\textbf{Mid Income} & 118 & 82 & 200 \\
			\hline
			\textbf{High Income} & 87 & 63 & 150 \\
			\hline
			\textbf{Totals} & 360 & 240 & 600 \\
			\hline
	\end{tabular}
	\newline
	\(H_0:\ \theta_1 = \theta_2 = \theta_3\)
	\newline
	\(H_1:\ \theta_1,\theta_2,\) and \(\theta_3\) are not all equal.
	\newline
	\(\alpha = 0.05\).
	The pooled estimate of \(\theta\) is \(\theta = \frac{360}{600} = 0.6\).
	\newline
	With our pooled estimate, we can construct the estimate table.
	\newline
	\begin{tabular}{|p{3.3cm}|p{2cm}|p{2cm}|p{2cm}|}
			\hline
			\textit{Expected \ \ \ \ \ \ \ \ \ Frequencies (\(e_{ij}\))} & \textbf{Favors} & \textbf{Does not Favor} & \textbf{Totals} \\
	 		\hline
			\textbf{Low Income} & 150 & 100 & 250 \\
			\hline
			\textbf{Mid Income} & 120 & 80 & 200 \\
			\hline
			\textbf{High Income} & 90 & 60 & 150 \\
			\hline
			\textbf{Totals} & 360 & 240 & 600 \\
			\hline
	\end{tabular}
	\newline
	From no particular theorem, the following sum is chi-square with k-1 degrees of freedom.
	\[\sum_{i=1}^k\sum_{j=1}^2\frac{(f_{ij}-e_{ij})^2}{e_{ij}}\]
	So we do the following test with 0.05 level of significance. We check:
	\[\sum_{i=1}^3\sum_{j=1}^2\frac{(f_{ij}-e_{ij})^2}{e_{ij}} < \chi^2_{0.05, 2}\].
	\[\chi^2_{0.05, 2} = 5.991\]
	\[\sum_{i=1}^3\sum_{j=1}^2\frac{(f_{ij}-e_{ij})^2}{e_{ij}} = 0.75 \textup{(very small!)}\]
	Since \(0.75 < 5.991\), we must accept the null hypothesis.
	\newline \newline
	\boxed{\textup{Income level does not affect whether people favor this piece of legislation}}
	\newline
	\newline
	\boxed{\textup{the p-value is 0.687289. (big!)}}
	\newpage
	\section[20pt]{p. 380 \#78}
	The following sample data pertain to the shipments received by a large firm from three different vendors:
	\newline
	\begin{tabular}{|p{2cm}|p{2.4cm}|p{2.4cm}|p{2.4cm}|p{1.6cm}|}
			\hline
			\textit{Data (\(f_{ij}\))} & \textbf{Number\ \ \ rejected} & \textbf{Number imperfect but acceptable} & \textbf{Number perfect} & \textbf{Totals} \\
			\hline
			\textbf{Vendor A} & 12 & 23 & 89 & 124 \\
			\hline
			\textbf{Vendor B} & 8 & 12 & 62 & 82 \\
			\hline
			\textbf{Vendor C} & 21 & 30 & 119 & 170 \\
			\hline
			\textbf{Totals} & 41 & 65 & 270  & 376 \\
			\hline
	\end{tabular}
	\newline
	Test at the 0.01 level of significance whether the three vendors ship products of equal quality.
	\newline
	\newline
	I am not quite sure how this works, but I assume the population is multinomial. We break the cases into three categories into two.
	\newline
	\newline
	\(\theta_1 = 41/376\),\ \(\theta_2 = 65/376\),\ \(\theta_3 = 270/376\)
	\newline \newline
	We construct the table of expected values for each cell using the 3 \(\theta\) values.
	\newline
	\begin{tabular}{|p{2cm}|p{2.4cm}|p{2.4cm}|p{2.4cm}|p{1.6cm}|}
			\hline
			\textit{Expected \ \ \ \ \ \ \ \ \ Frequencies (\(e_{ij}\))} & \textbf{Number\ \ \ rejected} & \textbf{Number imperfect but acceptable} & \textbf{Number perfect} & \textbf{Totals} \\
			\hline
			\textbf{Vendor A} & 12 & 23 & 89 & 124 \\
			\hline
			\textbf{Vendor B} & 8 & 12 & 62 & 82 \\
			\hline
			\textbf{Vendor C} & 21 & 30 & 119 & 170 \\
			\hline
			\textbf{Totals} & 41 & 65 & 270  & 376 \\
			\hline
	\end{tabular}
	\newline
	From no particular theorem, the following sum is chi-square with \((k-1)(c-1)\) degrees of freedom, where k and c are the category sizes.
	\[\sum_{i=1}^k\sum_{j=1}^c\frac{(f_{ij}-e_{ij})^2}{e_{ij}}\]
	So we do the following test with 0.01 level of significance. We check:
	\[\sum_{i=1}^3\sum_{j=1}^3\frac{(f_{ij}-e_{ij})^2}{e_{ij}} < \chi^2_{0.01, 4}\]
	\[\textup{From Table 5, } \chi^2_{0.01, 4} = 13.277\]
	\[\sum_{i=1}^3\sum_{j=1}^3\frac{(f_{ij}-e_{ij})^2}{e_{ij}} = 1.300642\]
	\boxed{\textup{Since }1.300642 < 13.277\textup{, we accept the null hypothesis.}}
	\newline
	The p-value of 1.300642 is 0.86126, which is greater than 0.01.
	\newline \(\blacksquare\)
\end{document}
