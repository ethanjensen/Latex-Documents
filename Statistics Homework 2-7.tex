\documentclass[12pt]{article}
\usepackage[utf8]{inputenc}
\usepackage{amsmath}
\usepackage{amsfonts}
\usepackage{amssymb}
\usepackage{empheq}
\usepackage{graphicx}
\usepackage{tikz}
\usetikzlibrary{automata, positioning, arrows, shapes}
\addtolength{\topmargin}{-0.875in}
\addtolength{\textheight}{1.75in}

\title{Math 332 A - Mathematical Statistics}
\author{Ethan Jensen}
\date{January 31, 2020}

\begin{document}
	\maketitle HW p.254 \#48,49,50 \ p.257 \#88
	\section[20pt]{p. 254 \#48}
	Find the mean and the variance of the sampling distribution of \(Y_1\) for random samples of size n from a continous uniform population with \(\alpha=0\) and \(\beta=1\).
	\newline \newline
	By Theorem 8.16 we can find the density of \(Y_{1}\).
	\[g_1(y_1)=nf(y_1)\left[\int_{y_1}^{\infty}f(x)dx\right]^{n-1}\]
	Since \(Y_1\) comes from a continuous uniform population with \(\alpha=0\) and \(\beta=1\),
	\[f(x)=1,\ 0\leq x \leq 1\]
	Thus, computing \(g_1(y_1)\) we have
	\[g_1(y_1)=n\cdot 1 \cdot \left[\int_{y_1}^{1}1dx\right]^{n-1}=n(1-y_1)^{n-1},\ 0 \leq y_1 \leq 1\]
	Now we compute the first and second moments about the origin.
	\[\mu_1'=\int_0^1n(y_1)(1-y_1)^{n-1}dy_1=\int_0^1n(1-x)x^{n-1}dx=\frac{1}{n+1}\]
	\[\mu_2'=\int_0^1n(y_1)^2(1-y_1)^{n-1}dy_1=\int_0^1n(1-x)^2x^{n-1}dx=\frac{2}{(n+1)(n+2)}\]
	By Theorem 4.6
	\[\sigma^2_{y_1}=\mu_2'-{\mu_1'}^2\]
	\[\sigma^2_{y_1}=\frac{2}{(n+1)(n+2)}-\frac{1}{(n+1)(n+1)}=\frac{n}{(n+1)^2(n+2)}\]
	\boxed{\begin{array}{ccc}
	\mu_{y_1}=\frac{1}{n+1}\ \ \ \ \ \ \ \ \\ \sigma^2_{y_1}=\frac{n}{(n+1)^2(n+2)}
\end{array}}
\newpage
	\maketitle HW p.254 \#48,49,50 \ p.257 \#88
	\section[20pt]{p. 254 \#49}
	Find the sampling distributions of \(Y_1\) and \(Y_n\) for random samples of size n from a population having the beta distribution with \(\alpha=3\) and \(\beta=2\).
	\newline \newline
	The density of a beta R.V. with \(\alpha=3\) and \(\beta=2\) is given by
	\[f(x)=\frac{\Gamma(3+2)}{\Gamma(2)\Gamma(3)}x^2(1-x)^1=12x^2(1-x),\ 0<x<1\]
	By Theorem 8.16 we have
	\[g_1(y_1)=nf(y_1)\left[\int_{y_1}^{\infty}f(x)dx\right]^{n-1}\]
	\[g_n(y_n)=nf(y_2)\left[\int_{-\infty}^{y_2}f(x)dx\right]^{n-1}\]
	Plugging in our density function and its range we have
	\[g_1(y_1)=n12y_1^2(1-y_1)\left[\int_{y_1}^{1}12x^2(1-x)dx\right]^{n-1}\]
	\boxed{g_1(y_1)=12^nny_1^2(1-y_1)^{1+2(n-1)}(3y_1^2+2y_1+1)^{n-1},\ 0 < y_1 < 1}
	\[g_n(y_n)=n12y_2^2(1-y_2)\left[\int_{0}^{y_2}12x^2(1-x)dx\right]^{n-1}\]
	\boxed{g_n(y_n)=12^nny_n^{2+3(n-1)}(y_n-1)(3y_n-4)^{n-1},\ 0 < y_n < 1}
\newpage
	\maketitle HW p.254 \#48,49,50 \ p.257 \#88
	\section[20pt]{p. 254 \#50}
	Find the sampling distribution of the median for random samples of size \(2m+1\) for the population from Exercise 8.49.
	\newline
	\newline
	From Exercise 8.49 the density of the population is given by
	\[f(x)=12x^2(1-x),\ 0<x<1\]
	By Theorem 8.16 we have
	\[h(\tilde{x})=\frac{(2m+1)!}{m!m!}\left[\int_{-\infty}^{\tilde{x}}f(x)dx\right]^mf(\tilde{x})\left[\int_{\tilde{x}}^{\infty}f(x)dx\right]^m\]
	Plugging in our density function and its range we have
	\[h(\tilde{x})=\frac{(2m+1)!}{m!m!}\left[\int_{0}^{\tilde{x}}12x^2(1-x)dx\right]^m12\tilde{x}^2(1-\tilde{x})\left[\int_{\tilde{x}}^{1}12x^2(1-x)dx\right]^m\]
	\boxed{h(\tilde{x})=\frac{12^{2m+1}(2m+1)!}{m!m!}\tilde{x}^{3m+2}(1-\tilde{x})^{2m+1}(3\tilde{x}^2+2\tilde{x}+1)^m(4-3\tilde{x})^m}
	\newpage
		\maketitle HW p.254 \#48,49,50 \ p.257 \#88
		\section[20pt]{p. 257 \#88}
		Find the probability that in a random sample of size \(n=4\) from the continous uniform population of Exercise 8.46, the smallest value will be at least 0.20. \newline \newline
		As seen from Exercise 8.48, the density of the population is given by
		\[f(x)=1,\ 0 < x <1\]
		By Theorem 8.16 the density of the smallest value of the population is
		\[g_1(y_1)=4(1-y_1)^3,\ 0 < y_1 < 1\]
		We want to find \(P(y_1 > 0.20)\).
		\[P(y_1 > 0.20) = \int_{0.20}^14(1-y_1)^3dy_1\]
		\boxed{P(y_1 > 0.20) = 0.4096 \textup{ or about } 40.96\%}
\end{document}
