\documentclass[12pt]{article}
\usepackage[utf8]{inputenc}
\usepackage{amsmath}
\usepackage{amsfonts}
\usepackage{amssymb}
\usepackage{empheq}
\usepackage{tikz}
\usetikzlibrary{automata, positioning, shapes}
\addtolength{\topmargin}{-0.875in}
\addtolength{\textheight}{1.75in}

\title{Regarding Positive Even Zeta Values}
\author{Ethan Jensen}
\date{October 12th, 2019}

\begin{document}
	\maketitle
	\tikzset{
		node distance=3cm, % specifies the minimum distance between two nodes. Change if necessary.
		initial text=$ $, % sets the text that appears on the start arrow
	}
\definecolor{mycolor1}{RGB}{230,230,230}
\definecolor{mycolor2}{RGB}{210,210,210}
\definecolor{mycolor3}{RGB}{190,190,190}

\section[20pt]{Analyzing Positive Even Zeta Values}
The Riemann Zeta function is a complex-valued function that accepts a complex number as an argument. This paper will focus entirely on the Zeta function evaluated at positive even integer values. Positive even Zeta values have an exact form in terms of \(\pi\) and can be calculated recursively several ways. This paper will explain two different ways of calculating said values. \newline \newline
The Riemann Zeta and Dirichlet Eta functions are defined as follows
\[\zeta(s)=\sum_{n=1}^{\infty}\frac{1}{n^s},\ \  \eta(s)=\sum_{n=1}^{\infty}\frac{(-1)^{n-1}}{n^s}\]
Trivially, we can show that \[\eta(s)=(1-2^{1-s})\zeta(s)\]
\textbf{Proof.}
\[\eta(s) = \frac{1}{1^s} - \frac{1}{2^s} + \frac{1}{3^s} - \frac{1}{4^s} + \frac{1}{5^s} - \frac{1}{6^s}...\]
\[\eta(s) = \frac{1}{1^s} + \frac{1}{2^s} + \frac{1}{3^s} + \frac{1}{4^s} + \frac{1}{5^s}...-2\left[\frac{1}{2^s} + \frac{1}{4^s} + \frac{1}{6^s}...\right]\]
\[\eta(s) = \frac{1}{1^s} + \frac{1}{2^s} + \frac{1}{3^s} + \frac{1}{4^s} + \frac{1}{5^s}...-\frac{2}{2^s}\left[\frac{1}{1^s} + \frac{1}{2^s} + \frac{1}{3^s}...\right]\]
\[\eta(s)=\zeta(s)-\frac{2}{2^s}\zeta(s)=(1-2^{1-s})\zeta(s)\]
\(\blacksquare\) \newline
\newpage
\section[20pt]{Fourier Series of even powers of x}
The general formula for Fourier Series is
\[f(x)=\frac{a_0}{2}+\sum_{n=1}^{\infty}a_n\cos(n\pi x)+b_n\sin(n\pi x)\]
To calculate \(a_n\ \rm{and}\ b_n\), we make use of the orthogonality property of sine and cosine.
\[a_n=\int_{-1}^{1}\cos(n\pi x)f(x)dx\]
\[b_n=\int_{-1}^{1}\sin(n\pi x)f(x)dx\]
Consider the the Fourier series of \(x^p\) where p is an even integer. \(p=2k, k\in \mathbb{N}\).
Since f(x) is an even function, we know that \(b_n=0\) for all n.
\[a_n=\int_{-1}^{1}x^p\cos(n\pi x)dx\]
Using integration by parts, we can turn the indefinite integral into two infinite series.
\[\rm{For}\ n>0,\ a_n=\left( \sum_{k=0}^{\frac{p}{2}}(-1)^k\frac{p!}{(p-2k)!}x^{p-2k}\frac{1}{(\pi n)^{2k+1}}\sin(n\pi x)\right)_{-1}^1\]
\[+\left( \sum_{k=0}^{\frac{p}{2}-1}(-1)^k\frac{p!}{(p-2k-1)!}x^{p-2k-1}\frac{1}{(\pi n)^{2(k+1)}}\cos(n\pi x)\right)_{-1}^1\]
This can be further simplified since \(\sin(n\pi x)\) evaluated at 1 and -1 is 0.
\[\rm{For}\ n>0,\ a_n=\left(\sum_{k=0}^{\frac{p}{2}-1}(-1)^k\frac{p!}{(p-2k-1)!}x^{p-2k-1}\frac{1}{(\pi n)^{2(k+1)}}\cos(n\pi x)\right)_{-1}^1\]
\[a_0=\int_{-1}^1x^pdx = \frac{2}{p+1}\]
Since p is even,
\[x^{p-2k-1}\cos(n\pi x)|_{-1}^1=2, \textup{n is even};\ \ -2, \textup{n is odd}\]
The final formula for \(a_n\) is:
\[a_n=\frac{2}{p+1}, n=0;\ \ \  a_n=\sum_{k=0}^{\frac{p}{2}-1}(-1)^k\frac{p!2(-1)^{n-1}}{(p-2k-1)!(n\pi)^{2(k+1)}}, n>0;\] \newline \newpage
\section{Calculating Zeta Values with Fourier Series}
Using our formula for \(a_n\), the Fourier series for \(x^p\) is the following:
\[x^p=\frac{1}{p+1}+\sum_{n=1}^{\infty}\sum_{k=0}^{\frac{p}{2}-1}(-1)^k\frac{p!2(-1)^{n-1}}{(p-2k-1)!(n\pi)^{2(k+1)}}\cos(n\pi x)\]
Since p is even, p can be written in the form \(p=2q\)
\[x^{2q}=\frac{1}{2q+1}+\sum_{n=1}^{\infty}\sum_{k=0}^{q-1}(-1)^k\frac{(2q)!2(-1)^{n-1}}{(2q-2k-1)!(n\pi)^{2(k+1)}}\cos(n\pi x)\]
\[x^{2q}=\frac{1}{2q+1}+\sum_{n=1}^{\infty}\sum_{k=0}^{q-1}(-1)^k\frac{2^{q+1}q!(-1)^{n-1}}{(2q-2k-1)2^{q-k-1}(p-k-1)!(n\pi)^{2(k+1)}}\cos(n\pi x)\]
\[x^{2q}=\frac{1}{2q+1}+\sum_{n=1}^{\infty}\sum_{k=0}^{q-1}(-1)^k\frac{2^{k+2}q!(-1)^{n-1}}{(2q-2k-1)(q-k-1)!(n\pi)^{2(k+1)}}\cos(n\pi x)\]
Finally, after a change of index, we have
\[{x^{2q}=\frac{1}{2q+1}+\sum_{n=1}^{\infty}\sum_{k=1}^q(-1)^{k-1}\frac{2^{k+1}q!(-1)^{n-1}}{(2q-2k+1)(q-k)!(n\pi)^{2k}}\cos(n\pi x)}\]
The left hand part of the summation looks suspiciously like the definition of the Dirichlet Eta function.
Exchanging summations we have
\[x^{2q}=\frac{1}{2q+1}+\sum_{k=1}^q\frac{(-1)^{k-1}2^{k+1}q!}{(2q-2k+1)(q-k)!\pi^{2k}}\sum_{n=1}^{\infty}\frac{(-1)^{n-1}}{n^{2k}}\cos(n\pi x)\]
Plugging in 0 for x, we have
\[0=\frac{1}{2q+1}+\sum_{k=1}^q\frac{(-1)^{k-1}2^{k+1}q!}{(2q-2k+1)(q-k)!\pi^{2k}}\sum_{n=1}^{\infty}\frac{(-1)^{n-1}}{n^{2k}}\]
Using the Definition of the Dirichlet Eta function we have
\[\frac{1}{2q+1}=\sum_{k=1}^q\frac{(-1)^{k}2^{k+1}q!}{(2q-2k+1)(q-k)!\pi^{2k}}\eta(2k)\]
Converting the Eta Function to the Zeta function we have
\newline
\boxed{\frac{1}{2q+1}=\sum_{k=1}^s\frac{(-1)^{k+1}2^{k+1}(1-2^{1-2k})q!}{(2q-2k+1)(q-k)!\pi^{2k}}\zeta(2k)} \newline
\newline
We can now plug in different values for q to calculate positive even zeta values recursively.
\newpage
\section{References}
“Separation of Variables.” Partial Differential Equations for Scientists and Engineers, by Stanley J. Farlow, Dover, 2015, pp. 33–75.
\newline
\newline“Growth and Change in Mathematics.” Understanding Infinity: the Mathematics of Infinite Processes, by A. Gardiner, Dover Publications, 2002, pp. 19–23.
\newline
\newline
Flammable Maths, "The Basel Problem \& its Alternating Formulation [ The Dirichlet Eta Function ]",\textit{YouTube} video, 15:12. Jan. 11, 2019. \newline
https://www.youtube.com/watch?v=MAoI\_\_hbdWM
\newline
\newline
Flammable Maths, "\textbf{BUT HOW DID EULER DO IT}?! A BEAUTIFUL Solution to the FAMOUS Basel Problem!",\textit{YouTube} video, 18:04. May 24, 2019. https://www.youtube.com/watch?v=JAr512hLsEU
\end{document}
