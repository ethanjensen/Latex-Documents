\documentclass[12pt]{article}
\usepackage[utf8]{inputenc}
\usepackage{amsmath}
\usepackage{amsfonts}
\usepackage{amssymb}
\usepackage{empheq}
\usepackage{tikz}
\usetikzlibrary{automata, positioning, arrows, shapes}
\addtolength{\topmargin}{-0.875in}
\addtolength{\textheight}{1.75in}

\title{Math 331 A - Probability}
\author{Ethan Jensen}
\date{November 15th, 2019}

\begin{document}
	\maketitle HW p.183 \#3,10,11,12,17,20, \ p.184 \#23,26 \ p.202 \#54,56,60
	\section[20pt]{p. 183 \#3}
	If a random variable X has a uniform density with the parameters \(\alpha\) and \(\beta\), find its distribution function. \newline \newline
	By Defintion 6.1 the probability density of X is
	\[u(x,\alpha,\beta)=\left\{\begin{array}{ccc}
	\frac{1}{\beta-\alpha} \textup{for}\ \alpha<x<\beta\\0\ \ \  \textup{elsewhere}\ \ \ \ \ 
	\end{array}\right.\]
	By Def. 3.5 we have the distribution function.
	\[F(x)=\int_{-\infty}^xf(x)dx=\int_{\alpha}^{x}\frac{1}{\beta-\alpha}dx,\ \alpha<x<\beta\]
	\boxed{F(x)=\frac{x-\alpha}{\beta-\alpha},\ \alpha<x<\beta}
	\newpage
	\maketitle HW p.183 \#3,10,11,12,17,20, \ p.184 \#23,26 \ p.202 \#54,56,60
	\section[20pt]{p. 183 \#10}
	Find the probabilities that a random variable will exceed 4 if it has a gamma distribution with \newline
	\textbf{(a)}\ \(\alpha=2\) and \(\beta=3\) \newline
	\textbf{(b)}\ \(\alpha=3\) and \(\beta=4\) \newline
	\newline
	Let X be the random variable with  \(\alpha=2\) and \(\beta=3\). \newline
	Let Y be the random variable with  \(\alpha=3\) and \(\beta=4\). \newline
	By Definition 6.2
	\[P(X=x)=g(x;2,3)=\left\{\begin{array}{ccc}
	\frac{1}{3^2 \Gamma(2)}x^{2-1}e^{-x/3}\textup{ for }x>0\\0\ \ \ \ \ \ \ \ \ \ \ \ \ \ \ \ \ \  \textup{elsewhere}
	\end{array}\right.\]
	\[P(Y=y)=g(y;3,4)=\left\{\begin{array}{ccc}
	\frac{1}{4^3 \Gamma(3)}y^{3-1}e^{-y/4}\textup{ for }y>0\\0\ \ \ \ \ \ \ \ \ \ \ \ \ \ \ \ \ \  \textup{elsewhere}
	\end{array}\right.\]
	\[P(X>4)=\int_{4}^{\infty}\frac{1}{3^2 \Gamma(2)}x^{2-1}e^{-x/3}dx\]
	\boxed{\textbf{(a)}\ P(X>4)=\frac{7}{3}e^{-\frac{4}{3}}\approx 0.61506}
	\[P(Y>4)=\int_{4}^{\infty}\frac{1}{4^3 \Gamma(3)}y^{3-1}e^{-y/4}dy\]
	\boxed{\textbf{(b)}\ P(X>4)=\frac{5}{2}e^{-1}\approx 0.91970}
	\newpage
	\maketitle HW p.183 \#3,10,11,12,17,20, \ p.184 \#23,26 \ p.202 \#54,56,60
	\section[20pt]{p. 183 \#11}
	Show that the gamma distribution with \(\alpha>1\) has a relative maximum at \(x=\beta(\alpha-1)\). What happens when \(0<\alpha<1\) and when \(\alpha=1\)? \newline
	\newline
	By Def. 6.2, for \(x>0\) the gamma distribution is defined as follows
	\[g(x;\alpha,\beta)=\frac{1}{\beta^\alpha\Gamma(\alpha)}x^{\alpha-1}e^{-x/\beta}\]
	\[g'(x;\alpha,\beta)=\frac{\alpha-1-\frac{x}{\beta}}{\beta^\alpha\Gamma(\alpha)}x^{\alpha-2}e^{-x/\beta}\]
	To find critical points, we set \(g'(x;\alpha,\beta)\) equal to 0.
	\[\frac{\alpha-1-\frac{x}{\beta}}{\beta^\alpha\Gamma(\alpha)}x^{\alpha-2}e^{-x/\beta}=0\]
	\[(\alpha-1-\frac{x}{\beta})x^{\alpha-2}=0\]
	By the zero product property, and the fact that g is non-differentiable at \(x=0\) we have
	\[\alpha-1-\frac{x}{\beta}=0\]
	\[x=\beta(\alpha-1)\]
	These are our critical points.
	\[g'(0)=0, g'(\alpha\beta)=\frac{-1}{\beta^\alpha\Gamma(\alpha)}(\alpha\beta)^{\alpha-2}e^{-\alpha}<0\]
	Thus, \(g(\beta(\alpha-1))\) has negative concavity.
	\newline
	\newline
	There is a local maximum at \(x=\beta(\alpha-1)\).
	\newline
	\newline
	Note that when \(0<\alpha<1\), the x-value for the local maximum, which is at \(x=\beta(\alpha-1)\) must be negative. But since x is never negative, there is no local maximum anywhere for the function. \newline \newline
	When \(\alpha=1\), the x-value for the local maximum, which is at \(x=\beta(1-1)=0\).
	
	\newpage
	\maketitle HW p.183 \#3,10,11,12,17,20, \ p.184 \#23,26 \ p.202 \#54,56,60
	\section[20pt]{p. 183 \#12}
	Prove Theorem 6.4, making the substitution \(y=x\left(\frac{1}{\beta}-t\right)\) in the integral defining \(M_X(t)\). \newline \newline
	\textbf{Theorem 6.4.} The moment-generating function for the gamma distribution is given by
	\(M_X(t) = (1-\beta t)^{-\alpha}\) \newline
	\textbf{Proof.} \newline
	By Def. 6.2, for \(x>0\) the gamma distribution is defined as follows
	\[g(x;\alpha,\beta)=\frac{1}{\beta^\alpha\Gamma(\alpha)}x^{\alpha-1}e^{-x/\beta}\]
	By Definition 4.6 we have
	\[M_X(t)=\int_{0}^{\infty}e^{tx}\frac{1}{\beta^\alpha\Gamma(\alpha)}x^{\alpha-1}e^{-x/\beta}dx\]
	\[M_X(t)=\frac{1}{\beta^\alpha\Gamma(\alpha)}\int_{0}^{\infty}e^{tx}x^{\alpha-1}e^{-x/\beta}dx\]
	\[M_X(t)=\frac{1}{\beta^\alpha\Gamma(\alpha)}\int_{0}^{\infty}x^{\alpha-1}e^{x(t-1/\beta)}dx\]
	Let \(y=x\left(\frac{1}{\beta}-t\right)\). \(dx = \left(\frac{\beta}{1-\beta t}\right)dy\)
	\[M_X(t)=\frac{1}{\beta^\alpha\Gamma(\alpha)}\int_{0}^{\infty}(\frac{y\beta}{1-\beta t})^{\alpha-1}e^{-y}\left(\frac{\beta}{1-\beta t}\right)dx\]
	\[M_X(t)=\frac{\beta^{\alpha}}{(1-\beta t)^{\alpha}\beta^\alpha\Gamma(\alpha)}\int_{0}^{\infty}y^{\alpha-1}e^{-y}dx\]
	By the definition of the Gamma function we have
	\[M_X(t)=\frac{\Gamma(\alpha)\beta^{\alpha}}{(1-\beta t)^{\alpha}\beta^\alpha\Gamma(\alpha)}\]
	\boxed{M_X(t) = (1-\beta t)^{-\alpha}}
	\newline
	\newline
	\(\blacksquare\)
	\newpage
	\maketitle HW p.183 \#3,10,11,12,17,20, \ p.184 \#23,26 \ p.202 \#54,56,60
	\section[20pt]{p. 183 \#17}
	If X is a random variable having an exponential distribution with the parameter \(\theta\), use Theorems 4.10 on page 128 and 6.4 to find the moment-generating function of the random variable \(Y=X-\theta\).
	\newline \newline
	By Theorem 6.4, the moment-generating function for the gamma distribution is given by
	\[M_X(t)=(1-\beta t)^{-\alpha}\]
	An exponential random variable is a special case of a gamma random variable, where  \(\alpha=1\) and \(\beta=\theta\). Thus, the moment generating function for an exponential random variable must be given by
	\[M_X(t)=(1-\theta t)^{-1}=\frac{1}{1-\theta t}\]
	By Theorem 4.10 we have
	\[M_Y(t)=M_{X-\theta}(t)=\frac{e^{-\theta t}}{1-\theta t}\]
	\boxed{M_Y(t)=\frac{e^{-\theta t}}{1-\theta t}}
	\newpage
	\maketitle HW p.183 \#3,10,11,12,17,20, \ p.184 \#23,26 \ p.202 \#54,56,60
	\section[20pt]{p. 183 \#20}
	A random variable X has a \textbf{Rayleigh distribution} if and only if its probability distribution is given by
	\[f(x) = \left\{\begin{array}{ccc}
	2\alpha xe^{-\alpha x^2} \textup{ for } x>0\\0 \ \ \ \ \ \ \ \ \ \ \ \textup{elsewhere}
	\end{array}\right.\]
	where \(\alpha > 0\). Show that for this distribution \newline
	\textbf{(a)} \(\mu=\frac{1}{2}\sqrt{\frac{\pi}{\alpha}}\);
	\newline
	\textbf{(b)} \(\sigma^2=\frac{1}{\alpha}\left(1-\frac{\pi}{4}\right)\).
	\newline \newline
	By Def. 4.2
	\[\mu_r'=\int_{-\infty}^{\infty}2\alpha x^{r+1}e^{-\alpha x^2}dx\]
	Let \(y=\alpha x^2,\ dy=2\alpha xdx\)
	\[\mu_r'=\int_{0}^{\infty} \left(\frac{y}{\alpha}\right)^{\frac{r}{2}}e^{-y}dy\]
	\[\mu_r'=\frac{1}{\sqrt{\alpha^r}}\int_{0}^{\infty} y^{\frac{r}{2}}e^{-y}dy\]
	By the definition of the Gamma function, we have
	\[\mu_r'=\frac{1}{\sqrt{\alpha^r}}\Gamma\left(\frac{r}{2}+1\right)\]
	\[\mu_1'=\frac{1}{\sqrt{\alpha^1}}\Gamma\left(\frac{3}{2}\right)=\frac{1}{\sqrt{\alpha}}\frac{1}{2}\Gamma\left(\frac{1}{2}\right)=\frac{1}{2}\frac{\sqrt{\pi}}{\sqrt{\alpha}}\]
	\[\mu_2'=\frac{1}{\sqrt{\alpha^2}}\Gamma\left(2\right)=\frac{1}{\alpha}1!=\frac{1}{\alpha}\]
	By Def. 4.3 \newline
	\boxed{\textbf{(a)}\ \mu=\frac{1}{2}\sqrt{\frac{\pi}{\alpha}}}
	\newline
	\newline
	By Theorem 4.6 we have
	\[\sigma^2=\mu_2'-{\mu_1'}^2\]
	\[\sigma^2=\frac{1}{\alpha}-\frac{\pi}{4\alpha}\]
	\boxed{\textbf{(b)}\ \sigma^2=\frac{1}{\alpha}\left(1-\frac{\pi}{4}\right)}\\
	\newpage
	\maketitle HW p.183 \#3,10,11,12,17,20, \ p.184 \#23,26 \ p.202 \#54,56,60
	\section[20pt]{p. 184 \#23}
	A random variable X has a \textbf{Weibull distribution} if and only if its probability density is given by
	\[ f(x)=\left\{\begin{array}{ccc}
	kx^{\beta-1}e^{-\alpha x^\beta} \textup{ for } x>0\\ 0 \ \ \ \ \ \ \ \ \ \ \ \ \textup{elsewhere}
	\end{array}\right. \]
	where \(\alpha>0\) and \(\beta>0\). \newline
	\textbf{(a)} Express k in terms of \(\alpha\) and \(\beta\). \newline
	\textbf{(b)} Show that \(\mu=\alpha^{-1/\beta}\Gamma\left(1+\frac{1}{\beta}\right)\) \newline \newline
	Since f(x) is a probability distribution, integrating over its full range results in 1. (Theorem 3.5)
	\[\int_0^{\infty}kx^{\beta-1}e^{-\alpha x^\beta}dx=1\]
	Let \(y=\alpha x^\beta \implies dy = \alpha x^{\beta-1}\)
	\[\int_0^{\infty}\frac{k}{\alpha}e^{-y}dy=\frac{k}{\alpha}=1\]
	\boxed{\textbf{(a)}\ k=\alpha} \newline \newline
	By Def. 4.2 and 4.3 
	\[\mu=\int_0^{\infty}x\cdot \alpha x^{\beta-1}e^{-\alpha x^\beta}dx=\int_0^{\infty}\alpha x^{\beta}e^{-\alpha x^\beta}dx\]
	Let \(y=\alpha x^\beta\implies dy = \alpha x^{\beta-1}, \ x = y^{1/\beta}\alpha^{-1/\beta}\)
	\[\mu=\int_0^{\infty}y^{1/\beta}\alpha^{-1/\beta}e^{-y}dy=\alpha^{-1/\beta}\int_0^{\infty}y^{1/\beta}e^{-y}dy\]
	By the integral definition of the Gamma function we have
	\newline \boxed{\textbf{(b)}\  \mu=\alpha^{-1/\beta}\Gamma\left(1+\frac{1}{\beta}\right)}
	\newpage
	\maketitle HW p.183 \#3,10,11,12,17,20, \ p.184 \#23,26 \ p.202 \#54,56,60
	\section[20pt]{p. 184 \#26}
	Show that if \(\alpha>1\) and \(\beta>1\), the beta density has a relative maximum at
	\[x=\frac{\alpha-1}{\alpha+\beta-2}\]
	\newline
	By Def. 6.5 the Beta distribution is given by
	\[f(x;\alpha,\beta)=\left\{\begin{array}{ccc}
	\frac{\Gamma(\alpha+\beta)}{\Gamma(\alpha)\Gamma(\beta)}x^{\alpha-1}(1-x)^{\beta-1} \textup{ for }0<x<1\\0 \ \ \ \ \ \ \ \ \ \ \ \ \ \ \ \ \ \ \ \ \ \ \ \ \ \ \ \ \textup{elsewhere} \ \ \ \ 
	\end{array}\right.\]
	where \(\alpha>0\) and \(\beta>0\).
	\newline \newline
	Taking the derivative of the Beta distribution we have
	\[f'(x;\alpha,\beta)=\frac{\Gamma(\alpha+\beta)}{\Gamma(\alpha)\Gamma(\beta)}\left((\alpha-1)x^{\alpha-2}(1-x)^{\beta-1}-(\beta-1)x^{\alpha-1}(1-x)^{\beta-2}\right)\]
	\[f'(x;\alpha,\beta)=\frac{\Gamma(\alpha+\beta)}{\Gamma(\alpha)\Gamma(\beta)}\left(((\alpha-1)(1-x)-(\beta-1)x)x^{\alpha-2}(1-x)^{\beta-2}\right)\]
	\[f'(x;\alpha,\beta)=\frac{\Gamma(\alpha+\beta)}{\Gamma(\alpha)\Gamma(\beta)}\left((\alpha-1+(2-\alpha-\beta)x)x^{\alpha-2}(1-x)^{\beta-2}\right)\]
	Since \(0<x<1\) and \( \lim_{x\rightarrow 0}f(x;\alpha,\beta)=\lim_{x\rightarrow 1}f(x;\alpha,\beta)=0\) We know that if at least one relative maximum exists for some value of x between 0 and 1. \newline \newline
	We find said critical point by letting \(f'(x;\alpha,\beta)=0\)
	\[f'(x;\alpha,\beta)=\frac{\Gamma(\alpha+\beta)}{\Gamma(\alpha)\Gamma(\beta)}\left((\alpha-1+(2-\alpha-\beta)x)x^{\alpha-2}(1-x)^{\beta-2}\right)=0\]
	Once again because \(0<x<1\) we can safely divide out by several terms.
	\[\alpha-1+(2-\alpha-\beta)x=0\]
	\boxed{x=\frac{\alpha-1}{\alpha+\beta-2}\textup{ is a relative maximum.}} \newline \newline
	\(\blacksquare\)
	\newpage
	\maketitle HW p.183 \#3,10,11,12,17,20, \ p.184 \#23,26 \ p.202 \#54,56,60
	\section[20pt]{p. 202 \#54}
	The amount of time that a watch will run without having to be reset is a random variable having an exponential distribution with \(\theta\)=120 days. Find the probabilities that the watch will \newline
	\textbf{(a)} have to be reset in less than 24 days;
	\newline
	\textbf{(b)} not have to be reset in at least 180 days.
	By Def. 6.3 an exponential distribution is given by
	\[f(x)=\left\{\begin{array}{ccc}
	\frac{1}{\theta}e^{-x/\theta}\ \textup{ for } x>0\\ 0 \ \ \ \ \ \ \ \ \textup{elsewhere}
	\end{array}\right.\]
	where \(\theta>0.\)
	\newline \newline
	Thus, by Def. 3.4 we have
	\[P(X<24)=\int_0^{24}\frac{1}{120}e^{-x/120}\]
	\[P(X<24)=1-e^{-\frac{1}{5}} \approx 0.18127\]
	\boxed{\textbf{(a)}\ \textup{The probability the watch will have to be reset in less than 24 days is}\approx 0.18127}
	\newline \newline
	By Def. 3.4 we have
	\[P(X>180)=\int_{180}^{\infty}\frac{1}{120}e^{-x/120}\]
	\[P(X>180)=e^{-3/2}\approx 0.22313\]
	\boxed{\textbf{(b)}\ \textup{The probability it will not have to be reset in at least 180 days is}\approx 0.22313}
	\newpage
	\maketitle HW p.183 \#3,10,11,12,17,20, \ p.184 \#23,26 \ p.202 \#54,56,60
	\section[20pt]{p. 202 \#56}
	The number of bad checks that a bank receives during a 5-hour business day is a Poisson random variable with \(\lambda=2\). What is the probability that it will not receive a bad check on any one day during the first 2 hours of business? \newline \newline
	The exponential random variable can describe the \textbf{waiting time} of a Poisson process, with \(\theta=\frac{2}{5\lambda}\) by scaling the time interval appropriately. \newline \newline
	Let random variable X represent the time before the bank receives a bad check on any one day. From our analysis above, we have
	\[P(X=x)=2e^{-2x} \textbf{ for }x>0\]
	Thus, by Def. 3.4 we have
	\[P(X>2)=\int_2^{\infty}\frac{4}{5}e^{-\frac{4}{5}x}\]
	\[P(X>2)=e^{-\frac{8}{5}}\approx0.20190\]
	\boxed{\begin{array}{ccc}
			\textup{The probability that the bank will not receive a bad check on}\\ \textup{any one day during the first 2 hours of business is}\approx0.20190
		\end{array}}
	\newpage
	\maketitle HW p.183 \#3,10,11,12,17,20, \ p.184 \#23,26 \ p.202 \#54,56,60
	\section[20pt]{p. 202 \#60}
	If the annual proportion of new restaurants that fail in a given city may be looked upon as a random variable having a beta distribution with \(\alpha=1\) and \(\beta=4\), find \newline
	\textbf{(a)}\ the mean of this distribution, that is, the annual proportion of new restaurants that can be expected to fail in the given city; \newline
	\textbf{(b)}\ the probability that at least 25 percent of all the new restaurants will fail in the given city in any one year. \newline \newline
	By Def. 6.5 the Beta distribution is given by
	\[f(x;\alpha,\beta)=\left\{\begin{array}{ccc}
	\frac{\Gamma(\alpha+\beta)}{\Gamma(\alpha)\Gamma(\beta)}x^{\alpha-1}(1-x)^{\beta-1} \textup{ for }0<x<1\\0 \ \ \ \ \ \ \ \ \ \ \ \ \ \ \ \ \ \ \ \ \ \ \ \ \ \ \ \ \textup{elsewhere} \ \ \ \ 
	\end{array}\right.\]
	where \(\alpha>0\) and \(\beta>0\).
	\newline \newline
	Since \(P(X=x)=f(x;1,4)\), for \(x>0\),
	\[P(X=x)=\frac{\Gamma(1+4)}{\Gamma(1)\Gamma(4)}x^{1-1}(1-x)^{4-1}\]
	\[P(X=x)=\frac{4\cdot3\cdot2\cdot1\Gamma(1)}{\Gamma(1)3\cdot2\cdot1\Gamma(1)}(1-x)^{3}\]
	\[P(X=x)=4(1-x)^{3}\]
	By Def. 4.2 and 4.3 we have
	\[\mu=\int_0^1x\cdot4(1-x)^{3}=\frac{1}{5}\]
	\boxed{\textbf{(a)}\ \textup{The proportion of new restaurants that can be expected to fail is }\frac{1}{5}}
	\newline \newline
	By Def. 3.4 we have
	\[P(X>\frac{1}{4})=\int_{\frac{1}{4}}^{1}4(1-x)^{3}=\frac{81}{256}\approx 0.31641\]
	\boxed{\textbf{(b)}\ \textup{The probability that at least 25\% of the new restaurants will fail is}\approx 0.31641}
\end{document}