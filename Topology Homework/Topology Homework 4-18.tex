\documentclass[12pt]{article}
\usepackage[utf8]{inputenc}
\usepackage{amsmath}
\usepackage{amsfonts}
\usepackage{amssymb}
\usepackage{empheq}
\usepackage{tikz}
\usepackage{enumitem}
\usepackage{eucal}
\usepackage{changepage}
\usetikzlibrary{automata, positioning, shapes, decorations.markings}
\addtolength{\topmargin}{-0.75in}
\addtolength{\textheight}{1.75in}
\addtolength{\evensidemargin}{-0.5in}

\title{Topology Homework 08}
\author{Ethan Jensen}
\date{April 18, 2020}

\begin{document}
  \maketitle
  \noindent
  \textbf{EXCERCISE 4.23}
  \newline
  Find three different topologies on the tree-point set \(X=\{a,b,c\}\), each consisting of five open sets (including \(X\) and \(\varnothing\)), such that two of the topologies are homeomorphic to each other, but the third is not homeomorphic to the other two.
  \newline \newline
  \(\mathcal{T}_1 = \{\varnothing, \{a\}, \{c\}, \{a,c\}, X\}\)
  \newline
  \(\mathcal{T}_2 = \{\varnothing, \{b\}, \{c\}, \{b,c\}, X\}\)
  \newline
  \(\mathcal{T}_3 = \{\varnothing, \{a\}, \{a,b\}, \{a,c\}, X\}\)
  \newline
  \newline
  Where \(f:X\rightarrow X\) defined by
  \(f(x) = \left\{ \begin{matrix}
  b,\ x=a  \\
  a,\ x=b  \\
  c,\ x=c
\end{matrix}\right.\)\ \ is a homeomorphism.
\newline
\newline
\(\mathcal{T}_1\) and \(\mathcal{T}_2\) are homeomorphic, since the homeomorphism f exists. However, there is no homeomorphism between \(\mathcal{T}_1\) and \(\mathcal{T}_3\) or \(\mathcal{T}_2\) and \(\mathcal{T}_3\).
\newpage
\noindent
\textbf{EXERCISE 4.24}
\newline
Prove that a bijection \(f: X \rightarrow Y\) is a homeomorphism if and only if \(f\) and \(f^{-1}\) map closed sets to closed sets.
\newline \newline
\textbf{Proof.}
\newline
\((\rightarrow)\) Let \(f: X \rightarrow Y\) be a homeomorphism.
\newline
Consider some closed set \(C \in X\).
\newline
\(C = X - U\) for some open set U in X.
\newline
\(f(C) = f(X-U) = f(X) - f(U)\) since \(f\) is a bijection.
\newline
Since \(f\) is a homeomorphism, \(f(U) = V\), where V is an open set.
\newline
So \(f(C) = Y - V\), which is closed in Y.
\newline \newline
Consider some closed set \(D \in Y\).
\newline
\(D = Y - V\) for some open set V in Y.
\newline
\(f^{-1}(D) = f^{-1}(Y-V) = f^{-1}(Y) - f^{-1}(V)\) since \(f\) is a bijection.
\newline
Since \(f\) is a homeomorphism, \(f^{-1}(V) = U\), where U is an open set.
\newline
So \(f^{-1}(V) = X - U\), which is closed in X.
\newline
Thus, \(f\) and \(f^{-1}\) map closed sets to closed sets.
\newline \(\square\) \newline
\((\leftarrow)\) Assume that \(f\) and \(f^{-1}\) map closed sets to closed sets.
\newline
Consider some closed set \(C \in X\).
\newline
\(C = X - U\) for some open set U in X.
\newline
\(f(C) = f(X-U) = f(X) - f(U)\) since \(f\) is a bijection.
\newline
\(f(C) = Y - f(U)\).
\newline
\(f(C)\) is closed in Y.
\newline
So \(f(U)\) is open in Y.
\newline
Consider some closed set \(D \in Y\).
\newline
\(D = Y - V\) for some open set V in Y.
\newline
\(f^{-1}(D) = f^{-1}(Y - V) = f^{-1}(Y) - f^{-1}(V)\) since \(f\) is a bijection.
\newline
\(f^{-1}(D)\) is closed in X.
\newline
So \(f^{-1}(V)\) is open in X.
\newline
Thus, \(f\) is a homeomorphism.
\newline \(\square\) \newline
\(\therefore\) A bijection \(f: X \rightarrow Y\) is a homeomorphism if and only if \(f\) and \(f^{-1}\) map closed sets to closed sets.
\newline \(\blacksquare\)
\newpage
\noindent
\textbf{EXERCISE 4.25}
\newline
\textbf{(a)} Provide an example of a homeomorphism between \(\mathbb{R}\) and the interval \((-\infty, a)\).
\newline
\textbf{(b)} Provide a formula for a homeomorphism between \(\mathbb{R}\) and the interval \((a,b)\), with \(a < b\).
\newline
\textbf{(c)} Given the homeomorphism in Example 4.12 and the first two parts of this exercise, prove that if \(I_1\) and \(I_2\) are in collection (i) in Example 4.12, then \(I_1\) and \(I_2\) are topologically equivalent.
\newline \newline
\textbf{(a)} Let \(f:\mathbb{R} \rightarrow (-\infty, a)\) where \(f(x) = a - e^{-x}\).
\newline \(\square\) \newline
\textbf{(b)} Let \(f:\mathbb{R} \rightarrow (a,b)\) where \(f(x) = a + \frac{b-a}{1+be^-x}\).
\newline \(\square\) \newline
\textbf{(c)} Let \(I_1\) and \(I_2\) be in the collection (i) consisting of open intervals: \((a,b), (-\infty,a), (a,\infty), \mathbb{R}\).
\newline
So there are four cases for \(I_1\) and \(I_2\).
\newline \newline
Case 1: \(I_1 = (a,b)\) for some \(a,b\in \mathbb{R}\) and \(a<b\).
\newline
\(I_1\) is homeomorphic to \(\mathbb{R}\) using \(f:\mathbb{R} \rightarrow I_1\) where \(f(x) = a + \frac{b-a}{1+be^-x}\).
\newline \newline
Case 2: \(I_1 = (-\infty, a)\) for some \(a \in \mathbb{R}\).
\newline
\(I_1\) is homeomorphic to \(\mathbb{R}\) using \(f:\mathbb{R} \rightarrow I_1\) where \(f(x) = a - e^{-x}\).
\newline \newline
Case 3: \(I_1 = (a, \infty)\) for some \(a \in \mathbb{R}\).
\newline
\(I_1\) is homeomorphic to \(\mathbb{R}\) using \(f:\mathbb{R} \rightarrow I_1\) where \(f(x) = a + e^{x}\).
\newline \newline
Case 4: \(I_1 = \mathbb{R}\).
\newline
\(I_1\) is homeomorphic to \(\mathbb{R}\) using \(f:\mathbb{R} \rightarrow I_1\) where \(f(x) = x\).
\newline
\newline
In every case, we can find homeomorphisms \(f:\mathbb{R} \rightarrow I_1\) and \(g:\mathbb{R} \rightarrow I_2\)
\newline \newline
The function \(g \circ f^{-1} (x): I_1 \rightarrow I_2\) is a homeomorphism, since continuity is maintained through composition.
\newline
\(\therefore I_1\) and \(I_2\) are topologically equivalent.
\newline \(\blacksquare\)
\newpage
\noindent
\textbf{EXERCISE 4.26}
\newline
\textbf{(a)} Provide a formula for a homeomorphism between the intervals \([0, \infty)\) and \([a,b)\), with \(a<b\).
\newline
\textbf{(b)} Provide a formula for a homeomorphism between the intervals \((-\infty, 0]\) and \((a,b]\), with \(a<b\).
\newline
\textbf{(c)} Given the homeomorphisms in Example 4.12 and the first two parts of this exercise, prove that if \(I_1\) and \(I_2\) are intervals in the collection (iii) in Example 4.12, then \(I_1\) and \(I_2\) are topologically equivalent.
\newline
\newline
\textbf{(a)} Let \(f:[0,\infty) \rightarrow [a,b)\) where \(f(x) = a + (b-a)(1-e^{-x})\).
\newline \(\square\) \newline
\textbf{(b)} Let \(f:(-\infty, 0] \rightarrow (a,b]\) where \(f(x) = b + (a-b)(1-e^x)\).
\newline \(\square\) \newline
\textbf{(c)} Let \(I_1\) and \(I_2\) be in the collection (iii) consisting of open intervals: \([a,b), (a,b], (-\infty,a], [a,\infty)\).
\newline
So there are four cases for \(I_1\) and \(I_2\).
\newline \newline
Case 1: \(I_1 = [a,b)\) for some \(a,b\in \mathbb{R}\) where \(a<b\).
\newline
\(I_1\) is homeomorphic to \([a,b)\) using
\newline
\(f:I_1\rightarrow [a,b)\) where \(f(x) = b+a-x\).
\newline \newline
Case 2: \(I_1 = (a,b]\) for some \(a,b\in\mathbb{R}\) where \(a<b\).
\newline
\(I_1\) is homeomorphic to \([a,b)\) using
\newline
\(f:I_1 \rightarrow [a,b)\) where \(f(x) = x\).
\newline \newline
Case 3: \(I_1 = (-\infty,a]\) for some \(a \in \mathbb{R}\).
\newline
\(I_1\) is homeomorphic to \([a,b)\) using
\newline
\(f:I_1 \rightarrow (a,b]\) where \(f(x) = a+(b-a)(1-e^{x-a})\).
\newline \newline
Case 4: \(I_1 = [a,\infty)\) for some \(a \in \mathbb{R}\).
\newline
\(I_1\) is homeomorphic to \([a,b)\) using
\newline
\(f:I_1 \rightarrow [a,b)\) where \(f(x) = a + (b-a)(1-e^{a-x})\).
\newline \newline
In every case, we can find homeomorphisms \(f:I_1 \rightarrow [a,b)\) and \(g:I_2 \rightarrow [a,b)\)
\newline \newline
The function \(g^{-1} \circ f(x): I_1 \rightarrow I_2\) is a homeomorphism, since continuity is maintained through composition.
\newline
\(\therefore I_1\) and \(I_2\) are topologically equivalent.
\newpage
\noindent
\textbf{EXERCISE 4.32}
\newline
Show that homeomorphism preserves interior, closure, and boundary as indicated in the following implications:
\newline
\textbf{(a)} If \(f: X \rightarrow Y\) is a homeomorphism, then \(f(\textup{Int}(A)) = \textup{Int}(f(A))\) for every \(A \subseteq X\).
\newline
\textbf{(b)} If \(f: X \rightarrow Y\) is a homeomorphism, then \(f(\textup{Cl}(A)) = \textup{Cl}(f(A))\) for every \(A \subseteq X\).
\newline
\textbf{(c)} If \(f: X \rightarrow Y\) is a homeomorphism, then \(f(\partial(A)) = \partial(f(A))\) for every \(A \subseteq X\).
\newline \newline
\textbf{(a)}
Since f is a homeomorphism, open sets in \(X\) map to open sets in \(Y\). Since homeomorphisms are injective, we know that for all \(U,V \subseteq X\),
\newline
\(U \subseteq V \implies f(U) \subseteq f(V)\)
\newline
So the biggest open subset of \(A\) maps to the biggest open subset of \(f(A)\).
\newline
\(\therefore f(\textup{Int}(A)) = \textup{Int}(f(A))\)
\newline \(\square\) \newline \newline
\textbf{(b)}
Since f is a homeomorphism, closed sets in \(X\) map to closed sets in \(Y\). Since homeomorphisms are injective, we know that for all \(U,V \subseteq X\),
\newline
\(U \subseteq V \implies f(U) \subseteq f(V)\)
\newline
So the smallest closed superset of \(A\) maps to the smallest closed superset of \(f(A)\).
\newline
\(\therefore f(\textup{Cl}(A)) = \textup{Cl}(f(A))\)
\newline \(\square\) \newline \newline
\textbf{(c)}
\(f(\partial A) = f(\textup{Cl}(A) - \textup{Int}(A))\)
\newline
Since f is a homeomorphism, and homeomorphisms are injective, we know that for all \(U,V \subseteq X\), \(f(U - V) = f(U) - f(V)\).
\newline
\(f(\partial (A)) = f(\textup{Cl}(A)) - f(\textup{Int}(A))\)
\newline
From \textbf{(a)} and \textbf{(b)}, \(f(\partial (A)) = \textup{Cl}f((A)) - \textup{Int}f((A))\)
\newline
\(\therefore f(\partial (A)) = \partial(f(A))\)
\newline \(\blacksquare\)
\end{document}
