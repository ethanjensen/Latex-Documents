\documentclass[12pt]{article}
\usepackage[utf8]{inputenc}
\usepackage{amsmath}
\usepackage{amsfonts}
\usepackage{amssymb}
\usepackage{empheq}
\usepackage{tikz}
\usetikzlibrary{automata, positioning, shapes}
\addtolength{\topmargin}{-1.75in}
\addtolength{\textheight}{1.75in}
\addtolength{\oddsidemargin}{-0.5in}
\addtolength{\evensidemargin}{-0.5in}

\title{Topology Homework}
\author{Ethan Jensen}
\date{January 9, 2020}

\begin{document}
	\maketitle
  \section[20pt]{Exercise 1.34}
	Prove that on a finite set, the discrete topology is the only topology that is Hausdorff.
	\newline \newline
	\textbf{Proof. } Assume that a finite set \(F\) with topology \(T\) is Hausdorff. \newline
	Consider the following recursive algorithm. \newline \newline
	|def Split(\(S\)): \newline
	|\(\ \ \)if (\(S\).length == 1): \newline
	|\(\ \ \ \ \)return \(S\) \newline
	|\(\ \ \)else: \newline
	|\(\ \ \ \ \)Pick two random points in \(S\), \(x_1\) and \(x_2\). \newline
	|\(\ \ \ \ \)## Note: The following operation is legal since S is Hausdorff. \newline
	|\(\ \ \ \ \)Pick two disjoint subsets in S, \(S_1\) and \(S_2\) that contain \(x_1\) and \(x_2\) resp. \newline
	|\(\ \ \ \ \)return Split(\(S_1\)) , Split(\(S_2\)) \newline \newline
	The result of Split(\(F\)) is a collection \(C\) of subsets of length 1, making up a partition of \(F\). \newline
	Since Split(S) is a combination of legal operations, \(C \in T\). \newline
	All subsets of \(F\) can be represented as a union of elements in \(C\), since they are length one. \newline
	By Definition 1.1, any union of open sets in a Topology is also an open set. \newline
	Thus, \(T\) is the discrete topology. \newline
	\(\therefore\) On any finite set, the discrete topology is the only topology that is Hausdorff. \newline
	\(\blacksquare\)


\end{document}
