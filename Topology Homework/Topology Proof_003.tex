\documentclass[12pt]{article}
\usepackage[utf8]{inputenc}
\usepackage{amsmath}
\usepackage{amsfonts}
\usepackage{amssymb}
\usepackage{empheq}
\usepackage{tikz}
\usepackage{enumitem}
\usepackage{changepage}
\usetikzlibrary{automata, positioning, shapes}
\addtolength{\topmargin}{-0.75in}
\addtolength{\textheight}{1.75in}
\addtolength{\evensidemargin}{-0.5in}

\title{Topology Homework}
\author{Ethan Jensen, Luke Lemaitre, Kasandra Lassagne}
\date{January 27, 2020}

\begin{document}
	\maketitle
	The following proofs were done by recalling previous Real Analysis notes, in which, similar theorems regarding countability were proved. \newline \newline
	A complete collection of the proofs required is provided on later pages.
	\newline
	\newline
	\textbf{Lemma 1: The union of two countable sets is countable}
	\newline \newline
	\textbf{Proof. } \newline
	Let S and T be countable sets.
	Thm\(^*\) tells us that  \(\exists f:\mathbb{N}\rightarrow S \textup{ and } g:\mathbb{N}\rightarrow T \ni f \textup{ and } g \textup{ are surjective.}\) \newline
	\[\textup{Consider } \left\{ \begin{array}{ccc} f\left(\frac{n+1}{2}\right) \textup{ if n is odd}\\ g\left(\frac{n}{2}\right) \textup{ if n is even }\
	\end{array}\right.\]
	so \(h:\mathbb{N}\rightarrow S \cup T\) is a surjection
	\newline
	so, by thm *, \(S \cup T\) is countable. //
	\newline \newline
	\textbf{Lemma 2: The product of two countable sets is countable}
	\newline \newline
	\textbf{Proof. } \newline
	Let S and T be countable sets. \newline
	thm * \(\exists\) injections \(f:S \rightarrow \mathbb{N}\) and \(g: T \rightarrow \mathbb{N}\). \newline
	define \(h(s,t)=2^{f(s)}3^{g(t)}, h: S \times T \rightarrow \mathbb{N}\) \newline
	let \(h(s,t)=h(u,v)\) \newline
	so \(2^{f(s)}3^{g(t)}=2^{f(u)}3^{g(v)}\) \newline
	Since prime factorization is unique, \newline
	\(2^{f(s)=2^{f(u)}}\) so \(s=u\) \newline
	and \(3^{g(t)}=3^{g(v)}\) so \(t=v\) \newline
	because f and g are injective. \newline
	so \(s,t) = (u,v))\) \newline
	\(\therefore\) h is injective from \(S \times T \rightarrow N\)
	\newline
	\(\therefore\) by thm * \(S \times T\) is countable. //
	\newline \(\blacksquare\) \newpage
	\(\ \) \newline
	\textbf{THEOREM 0.29. (iv)} \textit{A subset of a countable set is a countable set.} \newline \newline
	\textbf{Proof. } \newline
	Let S be a countable set and let \(T \subseteq S\) \newline
	Case I: T is finite, then T is countable \newline
	Case II: T is not finite or infinite \newline
	so S must be infinite and thus, S is denumerable \newline
	So \(\exists\) a bijection \(f:\mathbb{N}\rightarrow S\) and we can write \newline
	\(S=\{s_1,s_2,s_3,...\}\) so \(f(n)=s_n\) \newline
	Let \(A = \{n \in \mathbb{N}|\ s_n \in T\}\) \newline
	since A is a non empty subset of \(\mathbb{N}\)  it has a least element. call it \(a_1\). \newline
	so \(A-\{a_1\}\) has a least element. call it \(a_2\) \newline
	so \(A-\{a_1,a_2,...a_{k-1}\}\) has a least element. call it \(a_k\) \newline
	define \(g: \mathbb{N} \rightarrow \mathbb{N} \ni g(n) = a_n\) \newline
	Since \(a_{n+1} \notin \{a_1,a_2,...a_n\}\) g is injective \newline
	so \(f=g: \mathbb{N}\rightarrow S\) is injective \newline
	since every element of T is included in \(S_1\) \newline
	\(g(\mathbb{N})\) includes all subscripts of T \newline
	so \(f \circ g\) is a bijection from \(\mathbb{N}\rightarrow T\) \newline
	so T is denumerable and thus countable. //
	\newline \newline
	\textbf{THEOREM 0.29. (v)} \textit{A countable union of countable sets is a countable set.} \newline \newline
	\textbf{Proof. } \newline
	Let \(A_1, A_2, A_3,...\) be countable sets. \newline
	\textbf{Basis step: } \(A_1\) is countable by Lemma 1.
	\newline \(\square\) \newline
	\textbf{Induction hypothesis: } Suppose \(\bigcup_{i=1}^{k} A_i\) is countable for some \(k \in \mathbb{Z}_+\). \newline
	\textbf{Induction step: } Consider some set \(S = \bigcup_{i=1}^{k+1} A_i\). \newline \newline
	S can be written as \(S = A_{k+1} \cup \bigcup_{i=1}^{k} A_i\) \newline
	By the induction hypothesis, \(\bigcup_{i=1}^{k} A_i\) is a countable set. Thus, S is a union of two countable sets, so it is a countable set by Lemma 1. \newline \newline
	\(\therefore \bigcup_{i=1}^{n} A_i \) is a countable set for \(n=1,2,3...\), a countable union. \newline
	\(\blacksquare\) \newline \newline
	\textbf{THEOREM 0.29. (vi)} \textit{A product of countable sets is a countable set.} \newline \newline
	\textbf{Proof. } \newline
	Let \(A_1, A_2, A_3,...\) be countable sets. \newline
	\textbf{Basis step: } \(A_1\) is countable by Lemma 2.
	\newline \(\square\) \newline
	\textbf{Induction hypothesis: } Suppose \(A_1 \times A_2 \times ...\times A_k\) is countable for some \(k \in \mathbb{Z}_+\). \newline
	\textbf{Induction step: } Consider some set \(S = A_1 \times A_2 \times ...\times A_k \times A_{k+1}\). \newline \newline
	S can be written as \(S = A_{k+1} \times (A_1 \times A_2 \times ...\times A_k\) \newline
	By the induction hypothesis, \((A_1 \times A_2 \times ...\times A_k\) is a countable set. Thus, S is a product of two countable sets, so it is a countable set by Lemma 2. \newline \newline
	\((A_1 \times A_2 \times ...\times A_n\) is a countable set for \(n=1,2,3...\), a countable product. \newline
	\(\blacksquare\) \newline \newline
	\textbf{THEOREM: The set of real numbers is uncountable}
	\newline \newline
	\textbf{Proof: } \newline
	let \(J=(0,1)\) \newline
	NTS: J is uncountable \newline
	Suppose J is countable \newline
	J can be wrriten as \newline
	\(x_1 = 0.\){\Large \textcircled{\small\(a_{11}\)}}\(a_{12}a_{13}a_{14}\) \newline
	\(x_2 = 0.a_{21}\){\Large \textcircled{\small\(a_{22}\)}}\(a_{23}a_{24}\) \newline
	\(x_3 = 0.a_{31}a_{32}\){\Large \textcircled{\small\(a_{33}\)}}\(a_{34}\) \newline
	\(x_4 = 0.a_{41}a_{42}a_{43}\){\Large \textcircled{\small\(a_{44}\)}}
	\newline \(\vdots \) \newline
	Where each \(a_{ij}\in\{1,2,...9\}\) \newline
	construct the number \newline
	\(y = 0.b_1b_2b_3b_4...\) by
	\newline
	\[b_1 =\left\{ \begin{array}{ccc} 3 \textup{ if } a_{ii} \neq 3\\ 7 \textup{ if } a_{ii} = 3
	\end{array}\right.\]
	We know \(y \in J\) but \(y \neq x_i \forall i \in \mathbb{N}\) so \(y \notin J\ \ \rightarrow \leftarrow\)
	\newline
	\(\therefore\) J is uncountable \newline
	\(\therefore\) since \(J \in \mathbb{R},\ \mathbb{R}\) is uncountable (thm) //
	\newpage

	The following proofs is what our group came up with independent of any notes from Real Analysis. \newline \newline
	\textbf{THEOREM 0.9.} \textit{For sets A, B, and C, the following laws hold:}
	\newline
	\newline
	\textit{Distributive Laws:}
	\begin{adjustwidth}{-3cm}{0cm}
		\[(i)\ A \cap (B \cup C) = (A \cap B) \cup (A \cap C) \]
		\[(ii)\ A \cup (B \cap C) = (A \cup B) \cap (A \cup C) \]
		\[(iii)\ A \times (B \cup C) = (A \times B) \cup (A \times C) \]
		\[(iv)\ A \times (B \cap C) = (A \times B) \cap (A \times C) \]
		\[(v)\ A \times (B - C) = (A \times B) - (A \times C) \]
	\end{adjustwidth}
	\textit{DeMorgan's Laws:}
	\begin{adjustwidth}{-3cm}{0cm}
		\[(vi)\ A - (B \cup C) = (A - B) \cap (A - C) \]
		\[(vii)\ A - (B \cap C) = (A - B) \cup (A - C) \]
	\end{adjustwidth}
	\[\ \]
	\textbf{(i)\ Proof.}
	\newline
	Assume \(x \in A \cap (B \cup C)\). \newline
	\(x \in A \textup{ and } x \in (B \cup C)\) by the Definition of \(\cap\)
	\newline
	\(x \in A \textup{ and } (x \in B \textup{ or } x \in C)\) by the Definition of \(\cup\)
	\newline
	\((x \in A \textup{ and } x \in B) \textup{ or } (x \in A \textup{ and } x \in C)\) by the Distributive Law
	\newline
	\(x \in A \cap B \textup{ or } x \in A \cap C\) by the Definition of \(\cap\)
	\newline
	\(x \in (A \cap B) \cup (A \cap C)\) by the Definition of \(\cup\)
	\newline
	\(\therefore A \cap (B \cup C) \subseteq (A \cap B) \cup (A \cap C)\)
	\newline \(\square\) \newline
	Each step is reversible.
	\newline
	\((A \cap B) \cup (A \cap C) \subseteq A \cap (B \cup C)\)
	\newline
	\(\therefore A \cap (B \cup C) = (A \cap B) \cup (A \cap C)\)
	\newline \(\blacksquare\) \newline \newline
	\textbf{(ii)\ Proof.}
	\newline
	Assume \(x \in A \cup (B \cap C)\). \newline
	\(x \in A \textup{ and } x \in (B \cap C)\) by the Definition of \(\cup\)
	\newline
	\(x \in A \textup{ or } (x \in B \textup{ and } x \in C)\) by the Definition of \(\cap\)
	\newline
	\((x \in A \textup{ or } x \in B) \textup{ and } (x \in A \textup{ or } x \in C)\) by the Distributive Law
	\newline
	\(x \in A \cup B \textup{ and } x \in A \cup C\) by the Definition of \(\cup\)
	\newline
	\(x \in (A \cup B) \cap (A \cup C)\) by the Definition of \(\cap\)
	\newline
	\(\therefore A \cup (B \cap C) \subseteq (A \cup B) \cap (A \cup C)\)
	\newline \(\square\) \newline
	Each step is reversible.
	\newline
	\((A \cup B) \cap (A \cup C) \subseteq A \cup (B \cap C)\)
	\newline
	\(\therefore A \cup (B \cap C) = (A \cup B) \cap (A \cup C)\)
	\newline \(\blacksquare\)
	\newline
	\newline
	\textbf{(iii)\ Proof.}
	\newline
	Assume \((x,y) \in A \times (B \cup C)\). \newline
	\(x \in A \textup{ and } y \in (B \cup C)\) by the Definition of \(\times\)
	\newline
	\(x \in A \textup{ and } (y \in B \textup{ or } y \in C)\) by the Definition of \(\cup\)
	\newline
	\((x \in A \textup{ and } y \in B) \textup{ or } (x \in A \textup{ and } y \in C)\) by the Distributive Law
	\newline
	\((x,y) \in A \times B \textup{ or } (x,y) \in A \times C\) by the Definition of \(\times\)
	\newline
	\((x,y) \in (A \times B) \cup (A \times C)\) by the Definition of \(\cup\)
	\newline
	\(\therefore A \times (B \cup C) \subseteq (A \times B) \cup (A \times C)\)
	\newline \(\square\) \newline
	Each step is reversible.
	\newline
	\((A \times B) \cup (A \times C) \subseteq A \times (B \cup C)\)
	\newline
	\(\therefore A \times (B \cup C) = (A \times B) \cup (A \times C)\)
	\newline \(\blacksquare\)
	\newline
	\newline
	\textbf{(iv)\ Proof.}
	\newline
	Assume \((x,y) \in A \times (B \cap C)\). \newline
	\(x \in A \textup{ and } y \in (B \cap C)\) by the Definition of \(\times\)
	\newline
	\(x \in A \textup{ and } (y \in B \textup{ and } y \in C)\) by the Definition of \(\cap\)
	\newline
	\((x \in A \textup{ and } y \in B) \textup{ and } (x \in A \textup{ and } y \in C)\) by the Associative Law
	\newline
	\((x,y) \in A \times B \textup{ and } (x,y) \in A \times C\) by the Definition of \(\times\)
	\newline
	\((x,y) \in (A \times B) \cap (A \times C)\) by the Definition of \(\cap\)
	\newline
	\(\therefore A \times (B \cap C) \subseteq (A \times B) \cap (A \times C)\)
	\newline \(\square\) \newline
	Each step is reversible.
	\newline
	\((A \times B) \cap (A \times C) \subseteq A \times (B \cap C)\)
	\newline
	\(\therefore A \times (B \cap C) = (A \times B) \cap (A \times C)\)
	\newline \(\blacksquare\) \newline \newline
	\textbf{(v)\ Proof.}
	Assume \((x,y) \in A \times (B - C)\).
	\newline
	\(x \in A \textup{ and } y \in (B - C)\) by the Definition of \(\times\)
	\newline
	\(x \in A \textup{ and } (y \in B \textup{ and } \sim y \in C)\) by the Definition of \(-\)
	\newline
	\((x \in A \textup{ and } y \in B) \textup{ and } \sim(x \in A \textup{ and } y \in C)\) by the Annullment Law.
	\newline
	\((x,y) \in A \times B \textup{ and } \sim (x,y) \in A \times C\) by the Definition of \(\times\)
	\newline
	\((x,y) \in (A \times B) - (A \times C)\) by the Definition of \(-\)
	\newline
	\(\therefore A \times (B - C) \subseteq (A \times B) - (A \times C) \)
	\newline \(\square\) \newline
	Assume \((x,y) \in (A \times B) - (A \times C)\)
	\newline
	\((x,y) \in (A \times B) \textup{ and } \sim (x,y) \in (A \times C)\) by the Definition of \(-\)
	\newline
	\(x \in A \textup{ and } y \in B \textup{ and } \sim(x \in A \textup{ and } y \in C)\) by the Definition of \(\times\)
	\newline
	\(x \in A \textup{ and } y \in B \textup{ and } (\sim x \in A \textup{ or } \sim y \in C)\) by DeMorgan's Law
	\newline
	\(x \in A \textup{ and } y \in B \textup{ and } \sim y \in C\) by Elimination
	\newline
	\(x \in A \textup{ and } y \in (B - C)\) by the Definition of \(-\)
	\newline
	\((x,y) \in A \times (B - C)\) by the Definition of \(\times\)
	\newline
	\((A \times B) - (A \times C) \subseteq A \times (B - C) \)
	\newline
	\(\therefore A \times (B - C) = (A \times B) - (A \times C)\)
	\newline \(\blacksquare\) \newline \newline
	\textbf{(v)\ Proof.}
	Assume \(x \in A - (B \cup C)\).
	\newline
	\(x \in A \textup{ and } \sim x \in (B \cup C)\) by the Definition of \(-\)
	\newline
	\(x \in A \textup{ and } \sim (x \in B \textup{ or } x \in C)\)
	\newline
	\(x \in A \textup{ and } (\sim x \in B \textup{ and } \sim x \in C)\) by Demorgan's Law
	\newline
	\((x \in A \textup{ and } \sim x \in B) \textup{ and } (x \in A \textup{ and } \sim x \in C)\) by the Associative law
	\newline
	\(x \in A - B \textup{ and } x \in A - C\) by the Definition of -
	\newline
	\(x \in (A - B) \cap (A - C)\) by the Definition of \(\cap\)
	\newline
	\(\therefore A - (B \cup C) \subseteq (A - B) \cap (A - C)\)
	\newline \(\square\) \newline
	Each step is reversible.
	\newline
	\((A - B) \cap (A - C) \subseteq A - (B \cup C)\)
	\newline
	\(\therefore A - (B \cup C) = (A - B) \cap (A - C)\)
	\newline \(\blacksquare\)
	\newline
	\textbf{(vi)\ Proof.}
	Assume \(x \in A - (B \cap C)\).
	\newline
	\(x \in A \textup{ and } \sim x \in (B \cap C)\) by the Definition of \(-\)
	\newline
	\(x \in A \textup{ and } \sim (x \in B \textup{ and } x \in C)\)
	\newline
	\(x \in A \textup{ and } (\sim x \in B \textup{ or } \sim x \in C)\) by Demorgan's Law
	\newline
	\((x \in A \textup{ and } \sim x \in B) \textup{ or } (x \in A \textup{ and } \sim x \in C)\) by the Associative law
	\newline
	\(x \in A - B \textup{ or } x \in A - C\) by the Definition of -
	\newline
	\(x \in (A - B) \cup (A - C)\) by the Definition of \(\cup\)
	\newline
	\(\therefore A - (B \cap C) \subseteq (A - B) \cup (A - C)\)
	\newline \(\square\) \newline
	Each step is reversible.
	\newline
	\((A - B) \cup (A - C) \subseteq A - (B \cap C)\)
	\newline
	\(\therefore A - (B \cap C) = (A - B) \cup (A - C)\)
	\newline \(\blacksquare\)
	\newpage
	\textbf{THEOREM 0.21} \textit{If} \(f:X \rightarrow Y\) \textit{is a function and A and B are subsets of X, then}
	\begin{adjustwidth}{-2cm}{0cm}
		\begin{flushleft}
		\[(i)\ f(A\cup B) = f(A) \cup f(B). \]
		\[(ii)\ f(A \cap B) \subseteq f(A) \cap f(B). \]
		\[(iii)\ f(A) - f(B) \subseteq f(A - B) \]
	\end{flushleft}
	\end{adjustwidth}
	\(\ \) \newline
	\textbf{(i)\ Proof.}
	\newline
	Assume \(y \in f(A\cup B)\)
	\newline
	\(\exists x \in A \cup B \ni y = f(x)\)
	\newline
	\(\exists x \in A \textup{ or }\exists x \in B \ni y = f(x)\)
	\newline
	\(y \in f(A) \textup{ or }y \in f(B)\)
	\newline
	\(y \in f(A) \cup f(B)\)
	\newline
	\(f(A \cup B) \subseteq f(A) \cup f(B)\)
	\newline
	Assume \(y \in f(A) \cup f(B)\)
	\newline
	\(\exists x \in A \ni y = f(x)\textup{ or } \exists x \in B\ni y = f(x)\)
	\newline
	\(\exists x \in A \cup B\ni y = f(x)\) since \(A\) and \(B\) are both subsets of \(A \cup B\)
	\newline
	\(y \in f(A \cup B)\)
	\newline
	\(f(A) \cup f(B) \subseteq f(A \cup B)\)
	\newline
	\(\therefore f(A \cup B) = f(A) \cup f(B)\)
	\newline \(\blacksquare\) \newline \newline
	\textbf{(ii)\ Proof.}
	\newline
	Assume \(y \in f(A\cap B)\)
	\newline
	\(\exists x \in A\cap B\ni y = f(x)\)
	\newline
	\(\exists x \in A \ni y = f(x)\textup{ and } \exists x \in B\ni y = f(x)\) since \(A\cap B\) is a subset of both \(A\) and \(B\).
	\newline
	\(y \in f(A) \textup{ and } y \in f(B)\)
	\newline
	\(y \in f(A) \cap f(B)\)
	\newline
	\(\therefore f(A \cap B) \subseteq f(A) \cap f(B)\)
	\newline \(\blacksquare\) \newline \newline
	\textbf{(iii)\ Proof.}
	\newline
	Assume \(y \in f(A) - f(B)\)
	\newline
	\(y \in f(A) \textup{ and }\sim y \in f(B) \)
	\newline
	\(\exists x \in A \ni y = f(x)\textup{ and }\sim \exists x \in B\ni y = f(x)\)
	\newline
	\(\exists x \in A\cap B'\ni y = f(x)\) since \(x \in A\), but it cannot be in \(B\).
	\newline
	\(y \in f(A\cap B')\)
	\newline
	\(y \in f(A-B)\), which is a different way to write the same thing.
	\newline
	\(\therefore f(A)-f(B) \subseteq f(A-B)\)
	\newline \(\blacksquare\)
	\newpage
	\textbf{THEOREM 0.22. }\textit{if }\(f:X\rightarrow Y\) \textit{ is a function and V and W are subsets of Y, then}
	\begin{adjustwidth}{-3cm}{0cm}
		\begin{flushleft}
		\[(i)\ f^{-1}(V\cup W)=f^{-1}(V)\cup f^{-1}(W).\]
		\[(ii)\ f^{-1}(V\cap W)=f^{-1}(V)\cap f^{-1}(W).\]
		\[(iii)\ f^{-1}(V - W)=f^{-1}(V) - f^{-1}(W).\]
	\end{flushleft}
	\end{adjustwidth}
	\(\ \) \newline
	\textbf{(i)\ Proof.}
	Assume \(x \in f^{-1}(V \cup W)\)
	\newline
	\(f(x) \in V \cup W\)
	\newline
	\(f(x) \in V \textup{ or } f(x) \in W\)
	\newline
	\(x \in f^{-1}(V) \textup{ or } x \in f^{-1}(W)\)
	\newline
	\(x \in f^{-1}(V) \cup f^{-1}(W)\)
	\newline
	\(f^{-1}(V) \cup f^{-1}(W) \subseteq f^{-1}(V \cup W)\)
	\newline \(\square\) \newline
	Each step is reversible.
	\newline
	\(f^{-1}(V \cup W)\subseteq f^{-1}(V) \cup f^{-1}(W)\)
	\newline
	\(\therefore f^{-1}(V\cup W)=f^{-1}(V)\cup f^{-1}(W)\)
	\newline \(\blacksquare\) \newline
	\newline
	\textbf{(ii)\ Proof.}
	Assume \(x \in f^{-1}(V \cap W)\)
	\newline
	\(f(x) \in V \cap W\)
	\newline
	\(f(x) \in V \textup{ and } f(x) \in W\)
	\newline
	\(x \in f^{-1}(V) \textup{ and } x \in f^{-1}(W)\)
	\newline
	\(x \in f^{-1}(V) \cap f^{-1}(W)\)
	\newline
	\(f^{-1}(V) \cap f^{-1}(W) \subseteq f^{-1}(V \cap W)\)
	\newline \(\square\) \newline
	Each step is reversible.
	\newline
	\(f^{-1}(V \cap W)\subseteq f^{-1}(V) \cap f^{-1}(W)\)
	\newline
	\(\therefore f^{-1}(V\cap W)=f^{-1}(V)\cap f^{-1}(W)\)
	\newline \(\blacksquare\) \newline
	\newline
	\textbf{(iii)\ Proof.}
	Assume \(x \in f^{-1}(V-W)\)
	\newline
	\(f(x) \in V-W\)
	\newline
	\(f(x) \in V \textup{ and }\sim f(x) \in W\)
	\newline
	\(x \in f^{-1}(V) \textup{ and }\sim x \in f^{-1}(W)\)
	\newline
	\(x \in f^{-1}(V) - f^{-1}(W)\)
	\newline
	\(f^{-1}(V-W) \subseteq f^{-1}(V) - f^{-1}(W)\)
	\newline \(\square\) \newline
	Each step is reversible
	\newline
	\(f^{-1}(V) - f^{-1}(W) \subseteq f^{-1}(V-W)\)
	\newline
	\(\therefore f^{-1}(V - W)=f^{-1}(V) - f^{-1}(W)\)
	\newline \(\blacksquare\)
	\newpage
	\textbf{THEOREM 0.29. }
	\begin{adjustwidth}{1cm}{0cm}
		\textit{(i) A subset of a finite set is a finite set} \newline
		\textit{(ii) A finite union of finite sets is a finite set} \newline
		\textit{(iii) A product of finite sets is a finite set} \newline
		\textit{(iv) A subset of a countable set is a countable set} \newline
		\textit{(v) A countable union of countable sets is a countable set} \newline
		\textit{(vi) A product of countable sets is a countable set}
	\end{adjustwidth}
	\(\ \) \newline
	\textbf{(i)\ Proof.} Consider some finite set A. \newline
	A is empty or there exists a bijection \(f:\{1,2,...n\}\rightarrow A\)
	\newline
	If A is empty, any subset of A is empty, and is therefore finite.
	\newline
	If A is non empty, f gives a way to order elements in A.
	\newline \newline
	Consider some subset S of A. Since S is a subset of A, elements in S can also be ordered by f. \newline \newline
	Let \(g:\{1,2,3,...k\}\rightarrow S\) where \(g(a)\) is the ath smallest element in \(f^{-1}(S)\). Every element in \(S\) has a unique element in \(\{1,2,3,...k\}\) that g maps to it, so g is bijective. \newline
	Thus, S is finite. \newline
	\(\therefore\) A subset of a finite set is a finite set.
	\newline \(\blacksquare\) \newline \newline
	\textbf{(ii)\ Proof.} Consider the union of finite sets \(\bigcup_{i=1}^m A_i\).
	\newline
	The union of any set A and the empty set is A. Thus, if any of the \(A_i\) are empty, then we can construct an equivalent finite union \(\bigcup_{i=1}^n A_i\) such that no \(A_i\) is empty.
	\newline
	Next, assume all \(A_i\) are mutually disjoint.
	\newline
	Since each \(A_i\) are finite and nonempty, there exist bijective functions
	\(f_1,f_2,...f_n\ni\ f_i:\{1,2,...k_i\}\rightarrow A_i\) for some values \(k_i\in \mathbb{Z}_+\) \newline \newline
	Let \(P = \{2,3,5,...p_n\}\) be the set of the first n primes.
	\newline
	Let \(S = \{p_i^q| i \leq n, 1 \leq q \leq k_i\}\) be a subset of all prime powers.
	\newline \newline
	Let \(g: S \rightarrow \bigcup_{i=1}^n A_i\) where \(g(p_i^q)=f_i^{-1}(q)\)
	\newline
	By the Fundamental Theorem of Arithmetic, and the fact that all \(A_i\) are mutually disjoint, every element in \(\bigcup_{i=1}^n A_i\) has a unique element in \(S\) that g maps to it, so g is bijective.
	\newline \newline
	S is a subset of the integers, so can be ordered.
	\newline
	Let \(h: \{1,2,...(k_1+k_2+...k_n)\} \rightarrow S\) where \(h(a)\) is the ath smallest element in S.
	\newline
	Every element in S gets a unique element in \(\{1,2,...(k_1+k_2+...k_n)\}\), that h maps to it, so h is bijective.
	\newline \newline
	h and g are both bijective so \(g \circ h : \{1,2,...(k_1+k_2+...k_n)\} \rightarrow \bigcup_{i=1}^n A_i\) is also bijective.
	\newline
	So any finite union of mutually disjoint finite sets is finite.
	\newline
	If \(A_i\) are not mutually disjoint, then a union of such sets are equivalent to a finite union of mutually disjoint finite sets anyways, as shown below:
	\[\bigcup_{i=1}^n A_i = A_1 \cup \bigcup_{i=2}^n \left(A_i-\bigcup_{k=1}^{i-1} A_k\right)\]
	Note that each set in the above union is a subset of a finite set and therefore finite by Theorem 0.29 (i). \newline
	\(\therefore\) A finite union of finite sets is finite.
	\newline \(\blacksquare\) \newline \newline
	\textbf{(iii) Proof.} Consider some finite sets \(A_1,A_2,...A_n\).
	\newline
	Let \(S=A_1 \times A_2 \times ...\times A_n = \{(x_1,x_2,...x_n)| x_i \in A_i\ \forall i\}\) \newline
	If any of \(A_i\) are empty, \(S\) is empty, and therefore finite.
	\newline
	Otherwise, all \(A_i\) are nonempty.
	Since each \(A_i\) are finite and nonempty, there exist bijective functions
	\(f_1,f_2,...f_n\ni\ f_i:\{1,2,...k_i\}\rightarrow A_i\) for some values \(k_i\in \mathbb{Z}_+\)
	\newline \newline
	Let \(P = \{2,3,5,...p_n\}\) be the set of the first n primes.
	\newline
	Let \(Q = \{y\in \mathbb{Z}_+ | \textup{y has between 1 and } k_i \textup{ factors of }p_i, k_m = 0\textup{ for }m > n\}\) \newline
	\newline
	Let \(g: Q \rightarrow S\) where \(g(p_1^{x_1}p_2^{x_2}...p_n^{x_n})=(x_1,x_2,x_3)\)
	\newline
	By the Fundamental Theorem of Arithmetic, each n-tuple \((x_1,x_2,...x_n) \in S\) has a unique integer in Q that g maps to it, so g is bijective. \newline \newline
	Q is a subset of the integers, so it is ordered.
	\newline
	Let \(h: \{1,2,...(k_1+k_2+...k_n)\} \rightarrow S\) where \(h(a)\) is the ath smallest element in Q.
	\newline
	Every element in Q gets a unique element in \(\{1,2,...(k_1+k_2+...k_n)\}\), that h maps to it, so h is bijective.
	\newline \newline
	h and g are both bijective so \(g \circ h : \{1,2,...(k_1+k_2+...k_n)\} \rightarrow S\) is also bijective.
	\newline
	Thus, S is finite.
	\newline
	\(\therefore\) A product of finite sets is a finite set.
	\newline \(\blacksquare\) \newline \newline
	\textbf{(iv) Proof.} Let A be a countable set.
	\newline
	If A is finite, any subset of A is also finite by \textbf{(i)} and therefore countable. \newline
	If A is not finite, there exists a bijective function \(f: \mathbb{Z}_+ \rightarrow A\). \newline
	f gives a way to order elements in A. \newline
	Consider some subset S of A. Since S is a subset of A, elements in S can also be ordered by f. \newline \newline
	Let \(g: \mathbb{Z}_+ \rightarrow S\) where g(a) is the ath smallest element in \(f^{-1}(S)\). \newline
	Every element in S has a unique element in \(\mathbb{Z}_+\) that g maps to it, so g is bijective. \newline
	\(\therefore\) A subset of a countable set is a countable set.
	\newline \(\blacksquare\) \newline \newline
	\textbf{(v) Proof.} Consider the countable sets \(A_1,A_2,A_3...\)
	\newline
	Assume all \(A_i\) are mutally disjoint.
	\newline \newline
	There exist bijective functions \(f_1,f_2,f_3... \ni\ f_i: \mathbb{Z}_+ \rightarrow A_i\ \forall i \in \mathbb{Z}_+\)
	\newline
	\newline
	Let \(P = \{2,3,5...p_i...\}\) be the set of all prime numbers.
	\newline
	Let \(S = \{2, 2^2, 2^3, 3, 3^2,...107^{67}...\}\) be the set of all prime powers. \newline \newline
	Let \(g: S \rightarrow \bigcup_{i \geq 1} A_i\) where \(g(p_i^k)=f^{-1}_i(k)\) \newline
	By the Fundamental Theorem of Arithmetic, and the fact that all \(A_i\) are mutually disjoint, every element in \(\bigcup_{i \geq 1} A_i\) has a unique element in S that g maps to it, so g is bijective. \newline \newline
	S is a subset of the integers, so can be ordered.
	\newline
	Let \(h: \{1,2,...(k_1+k_2+...k_n)\} \rightarrow S\) where \(h(a)\) is the ath smallest element in S.
	\newline
	Every element in S gets a unique element in \(\mathbb{Z}_+\), that h maps to it, so h is bijective.
	\newline \newline
	h and g are both bijective so \(g \circ h : \mathbb{Z}_+ \rightarrow \bigcup_{i \geq 1} A_i\) is also bijective.
	\newline So any countable union of mutually disjoint countable sets is countable.
	\newline \newline
	Using the same reasoning as \(\textbf{(ii)}\), If \(A_i\) are not mutually disjoint, then a union of such sets is equivalent to a countable union of mutually disjoint countable sets anyways, as shown below:
	\[\bigcup_{i \geq 1} A_i = A_1 \cup \bigcup_{i \geq 2}\left(A_i-\bigcup_{k=1}^{i-1}\right)\]
	Note that each set in the above union is a subset of a countable set and therefore countable by Theorem 0.29 (iv). \newline
	\(\therefore\) A countable union of countable sets is countable.
	\newline \(\blacksquare\) \newline \newline
	\textbf{(vi) Proof.} Consider some countable sets \(A_1, A_2,A_3...\).
	\newline
	Let \(S = A_1\times A_2...A_n = \{(x_1,x_2,...x_n)|x_i\in A_i\ \forall i\}\)
	If any of \(A_i\) are empty, \(S\) is empty, and therfore countable.
	\newline
	Otherwise, all \(A_i\) are nonempty. Since each \(A_i\) are countable and nonempty, there exist bijective functions \(f_1,f_2,...f_n\ni\ f_i: \mathbb{Z}_+ \rightarrow A_i\ \forall i\)
	\newline \newline
	Let \(P = \{2,3,5...p_n\}\) be the set of the first n numbers.
	\newline
	Let \(W = \{y \in \mathbb{Z}_+|y \textup{ contains no factors of } p_m \textup{ if } m>n\}\)
	\newline \newline
	Let \(g: W \rightarrow S\) where \(g(p_1^{x_1}p_2^{x_2}...p_n^{x_n})=(x_1,x_2,...x_n)\)
	\newline
	By the Fundamental Theorem of Arithmetic, each n-tuple \((x_1,x_2,...x_n)\in S\) has a unique integer in W that g maps to it, so g is bijective.
	\newline \newline
	W is a subset of the integers, so it is ordered.
	\newline
	Let \(h: \mathbb{Z}_+ \rightarrow S\) where \(h(a)\) is the ath smallest element in W.
	\newline
	Every element in W gets a unique element in \(\mathbb{Z}_+\), that h maps to it, so h is bijective.
	\newline \newline
	h and g are both bijective so \(g \circ h : \{\mathbb{Z}_+ \rightarrow S\) is also bijective.
	\newline
	Thus, S is a countable set.
	\newline
	\(\therefore\) A product of countable sets is a countable set.
	\newline \(\blacksquare\)
	\newpage
	\textbf{Prove that }\(\mathbb{Q}\) \textbf{ is countable}
	\newline \newline
	\textbf{Proof. }
	\newline
	Let \(f: \mathbb{Z}_+ \rightarrow \mathbb{Z}_-\) where \(f(a) = -a\).
	each element in \(\mathbb{Z}_-\) get a unique element in \(\mathbb{Z}_+\) that f maps to it, so f is bijective.
	\newline
	Thus, \(\mathbb{Z}_-\) is a countable set.
	\newline \newline
	\(\mathbb{Z} = \{0\} \cup \mathbb{Z}_- \cup \mathbb{Z}_+\), being a countable union of countable sets is countable. \newline \newline
	\(\mathbb{Z}^2 = \mathbb{Z} \times \mathbb{Z}\), being a product of countable sets is countable.
	\(\{(a,b)\in \mathbb{Z}^2| b \neq 0\} \subseteq \mathbb{Z}^2\), being a subset of a countable set is countable.
	\newline
	\newline
	By definition, \(\mathbb{Q} = \{\frac{a}{b}\in \mathbb{R}|\ (a,b)\in \mathbb{Z}^2, b \neq 0\}\)
	\newline
	Let \(g: \{(a,b)\in \mathbb{Z}^2, b \neq 0\} \rightarrow \mathbb{Q}\) where \(g(a,b) = \frac{a}{b}\)
	\newline
	each element in \(\mathbb{Q}\) get a unique element in \(\{(a,b)\in \mathbb{Z}^2, b \neq 0\}\) that g maps to it, so g is bijective.
	\newline \newline
	Since \(\{(a,b)\in \mathbb{Z}^2, b \neq 0\}\) is countable, can can define a bijection \(h: \mathbb{Z}_+ \rightarrow \{(a,b)\in \mathbb{Z}^2, b \neq 0\}\) \newline \newline
	h and g are both bijective so \(g \circ h : \{\mathbb{Z}_+ \rightarrow \mathbb{Q}\) is also bijective. \newline
	\(\therefore\) \(\mathbb{Q}\) is countable.
	\newline \(\blacksquare\) \newpage
	\textbf{Prove that }\(\mathbb{R}\) \textbf{ is uncountable}
	\newline \newline
	\textbf{Proof.}
	\newline
	Recall that any real number \(r \in (0,1)\) can be represented as an infinite series \(r=\sum_{n=1}^{\infty}a_n2^{-n}\), where \(a_n\) is either 0 or -1.
	\newline \newline
	Suppose that the closed inteval \((0,1)\) is countable. \newline
	Then there exists a bijective function \(f: \mathbb{Z}_+ \rightarrow (0,1)\), where f must have the property that \(f(i) = \sum_{n=1}^{\infty}a_n^i2^{-n}\) for some coefficients \(a_n^i\).
	\newline \newline
	Let \(x = \sum_{n=1}^{\infty}(1-a_n^n)2^{-n}\) \newline
	\(x \in (0,1)\) since each coefficient \(1-a_n^n\) is either a 0 or a 1. \newline \newline
	But \(\sim \exists i \ni f(i) = x\) since the coeffiecient \(1-a_n^n\) is different from \(a_i^i\) for all i. This means f is not onto. That's bad.
	\newline \newline
	We assumed that f was bijective, but have shown that it is not onto. This is a contradiction. \newline
	Thus, \((0,1)\) is uncountable. \newline
	\((0,1) \subseteq \mathbb{R}\), and supersets of uncountable sets are uncountable. \newline \newline
	\(\therefore \mathbb{R}\) is uncountable.
	\newline \(\blacksquare\) \newline \newline
	The fact that the real numbers in \((0,1)\) can be represented as a sum of negative powers of 2 is shown in a book titled "Digital Logic Design" by Peter J. Ashenden.
\end{document}
