\documentclass[12pt]{article}
\usepackage[utf8]{inputenc}
\usepackage{amsmath}
\usepackage{amsfonts}
\usepackage{amssymb}
\usepackage{empheq}
\usepackage{tikz}
\usepackage{enumitem}
\usepackage{eucal}
\usepackage{changepage}
\usetikzlibrary{automata, positioning, shapes}
\addtolength{\topmargin}{-0.75in}
\addtolength{\textheight}{1.75in}
\addtolength{\evensidemargin}{-0.5in}

\title{Topology Homework 04}
\author{Ethan Jensen, Luke Lemaitre, Kasandra Lassagne}
\date{February 23, 2020}

\begin{document}
	\maketitle
	\noindent
	\textbf{EXERCISE 2.1}
	Determine Int(A) and Cl(A) in each case.
	\begin{adjustwidth}{1cm}{0cm}
		\begin{flushleft}
			\textbf{(a)} \(A = (0,1]\) in the lower limit topology on \(\mathbb{R}\). \newline
			\textbf{(b)} \(A = \{a\}\) in \(X = \{a,b,c\}\) with topology \(\{X, \varnothing, \{a\}, \{a,b\}\}\). \newline
			\textbf{(c)} \(A = \{a,c\}\) in \(X = \{a,b,c\}\) with topology \(\{X, \varnothing, \{a\}, \{a,b\}\}\). \newline
			\textbf{(d)} \(A = \{b\}\) in \(X = \{a,b,c\}\) with topology \(\{X, \varnothing, \{a\}, \{a,b\}\}\). \newline
			\textbf{(e)} \(A = (-1,1) \cup \{2\}\) in the standard topology on \(\mathbb{R}\). \newline
			\textbf{(f)} \(A = (-1,1) \cup \{2\}\) in the lower limit topology on \(\mathbb{R}\). \newline
			\textbf{(g)} \(A = \{(x,0)\in\mathbb{R}^2|\ x \in \mathbb{R}\}\) in \(\mathbb{R}^2\) with the standard topology. \newline
			\textbf{(h)} \(A = \{(0,y) \in \mathbb{R}^2|\ y \in \mathbb{R}\}\) in \(\mathbb{R}^2\) with the vertical interval topology. \newline
			\textbf{(i)} \(A = \{(x,0)\in\mathbb{R}^2|\ x \in \mathbb{R}\}\) in \(\mathbb{R}^2\) with the vertical interval topology.
	\end{flushleft}
\end{adjustwidth}
	\(\ \) \newline
	\textbf{(a)} \boxed{\textup{Int(A)} = (0,1),\ \textup{Cl(A)} = (0,1]} \newline
	\textbf{(b)} \boxed{\textup{Int(A)} = \{a\},\ \textup{Cl(A)} = X} \newline
	\textbf{(c)} \boxed{\textup{Int(A)} = \{a\},\ \textup{Cl(A)} = X} \newline
	\textbf{(d)} \boxed{\textup{Int(A)} = \varnothing,\ \textup{Cl(A)} = \{c,b\}} \newline
	\textbf{(e)} \boxed{\textup{Int(A)} = (-1,1),\ \textup{Cl(A)} = [-1,1] \cup \{2\}} \newline
	\textbf{(f)} \boxed{\textup{Int(A)} = [-1,1),\ \textup{Cl(A)} = [-1,1) \cup \{2\}} \newline
	\textbf{(g)} \boxed{\textup{Int(A)} = \varnothing,\ \textup{Cl(A)} = A} \newline
	\textbf{(h)} \boxed{\textup{Int(A)} = A,\ \textup{Cl(A)} = A} \newline
	\textbf{(i)} \boxed{\textup{Int(A)} = \varnothing,\ \textup{Cl(A)} = A} \newline
	\newpage
	\noindent
	\textbf{EXERCISE 2.2} \textbf{Prove Theorem 2.2, parts (ii), (iv), and (vi):} Let X be a topological space and A and B be subsets of X.
	\begin{adjustwidth}{1cm}{0cm}
		\begin{flushleft}
			\textbf{(a)} If C is a closed set in \(X\) and \(A \subseteq C\) then Cl(A) \(\subseteq C\). \newline
			\textbf{(b)} If \(A \subseteq B\) then Cl(A) \(\subseteq\) Cl(B). \newline
			\textbf{(c)} A is closed if and only if \(A =\) Cl(A).
	\end{flushleft}
\end{adjustwidth}
\(\ \)
\newline
\textbf{(a) Proof.}
\newline
Let C be a closed set in X and \(A \subseteq C\).
\newline
By Definition,
\[\textup{Cl(A)} = \bigcap_{A \subseteq C_i}C_i\]
Since \(A \subseteq C\)
\[\bigcap_{A \subseteq C_i}C_i \subseteq C\]
\(\therefore\) Cl(A) \(\subseteq C\).
\newline \(\blacksquare\) \newline \newline
\textbf{(b) Proof.}
\newline
Assume \(A \subseteq B\).
\newline
Since \(B \subseteq\) Cl(B), \(A \subseteq\) Cl(B).
By Theorem 2.2 part ii, \newline
Cl(A) \(\subseteq\) Cl(B).
\newline \(\blacksquare\) \newline \newline
\textbf{(c) Proof.}
\newline
\((\rightarrow)\) Assume A is closed.
We know that \(A \subseteq\) Cl(A). \newline
By Definition,
\[\textup{Cl(A)} = \bigcap_{A \subseteq C_i}C_i\]
Since A is a closed set and \(A \subseteq A\),
\newline
Cl(A) \(\subseteq A\).
\newline
Thus, \(A =\)Cl(A).
\newline \(\square\) \newline
\((\leftarrow)\) Assume \(A =\)Cl(A).
\newline
A = Cl(A) is an intersection of closed sets and is closed.
\newline \(\blacksquare\)
\newpage
\noindent
\textbf{EXERCISE 2.4} Consider the particular point topology \(PPX_p\) on a set \(X\). Determine Int(A) and Cl(A) for sets A containing p and for sets A not containing p.
\newline \newline
If A contains p, \newline
Int(A) \(=A\),\ Cl(A) \(=X\).
\newline
\newline
If A does not contain p, \newline
Int(A) \(=\varnothing\),\ Cl(A) \(=A\).
\newpage
\noindent
\textbf{EXERCISE 2.6} Prove that Cl\((\mathbb{Q})=\mathbb{R}\) in the standard topology on \(\mathbb{R}\).
\newline \newline
\textbf{Proof.} \newline
Let \(\mathcal{T}\) be the standard topology on \(\mathbb{R}\).
\newline
Since Cl\((\mathbb{Q})\) is closed,\ \ \(\exists U \in \mathcal{T} \ni \mathbb{R} - \)Cl\((\mathbb{Q})=U\)
\newline
Since \(\mathbb{Q} \subseteq\) Cl\((\mathbb{Q})\),\ \ \(U \subseteq \mathbb{R} - \mathbb{Q}\).
\newline \newline
So U does not contain any rational numbers. \newline
But, since \(\mathbb{Q}\) is dense, every open interval contains rational numbers. \newline
This means \(U = \varnothing\).
\newline \newline
\(\mathbb{R} - \)Cl\((\mathbb{Q}) = \varnothing\)
\newline
\(\therefore\) Cl\((\mathbb{Q})=\mathbb{R}\).
\newline \(\blacksquare\)
\newpage
\noindent
\textbf{EXERCISE 2.10} \textbf{Prove Theorem 2.5.} Let X be a topological space, A be a subset of X and y be an element of X. Then \(y \in\) Cl(A) if and only if every open set containing y intersects A. \newline \newline
\textbf{Proof.} \newline
\((\rightarrow)\)Let \(y \in\) Cl(A). \newline
By the defintion of Cl(A),
\[y \in \bigcap_{\begin{array}{ccc}C \subseteq A \\ \textup{C closed} \end{array}}C\]
By Demorgan's Law, we have
\[y \notin \bigcup_{\begin{array}{ccc}U \subseteq X-A \\ \textup{U open} \end{array}}U\]
Thus, if \(y \in U\), where \(U\) is open, then \(U \nsubseteq X - A\). \newline
\(\therefore\) Every open set containing y intersects A.
\newline \(\square\) \newline
\((\leftarrow)\) Assume every open set containing y intersects A.
\newline
Let \(U = X - \)Cl\((A)\).
\newline
Since Cl\((A)\) is closed, \(U\) is open.
\newline
\(y \notin U\) since U does not intersect A.
\newline
Thus, \(y \in X - U\).
\newline
\(\therefore y \in \)Cl\((A)\).
\newline \(\square\) \newline
\(\therefore\) \(y \in\) Cl(A) if and only if every open set containing y intersects A. \newline \(\blacksquare\)
\newpage
\noindent
\textbf{EXERCISE 2.11} \textbf{Prove Theorems 2.6 parts (ii) and (iv):} For sets A and B in a topological space X, the following hold:
\begin{adjustwidth}{1cm}{0cm}
	\begin{flushleft}
		\textbf{(a)} Cl\((X-A)\) = \(X - \) Int\((A)\) \newline
		\textbf{(b)} Int\((A) \cap \) Int\((B) = \) Int\((A \cap B)\).
\end{flushleft}
\end{adjustwidth}
\(\ \)
\newline
\textbf{(a) Proof.} \newline
Theorem 2.6 part (i) states that for any \(A \subseteq X\)
\[Int(X-A) = X - Cl(A)\]
Substituting \(X-A\) for A we have
\[Int(A) = X - Cl(X-A)\]
\(\therefore\) Cl\((X-A)\) = \(X - \) Int\((A)\)
\newline \(\blacksquare\) \newline
\textbf{(b) Proof.} \newline
\((\rightarrow)\)Let \(x \in Int(A) \cap Int(B)\) \newline
\(x \in Int(A)\) and \(x \in Int(B)\). \newline
Int\((A) \subseteq A\) and Int\((B) \subseteq B\) \newline
\(x \in A\) and \(x \in B\). \newline
\(x \in A \cap B\). \newline \newline
\(Int(A) \cap Int(B) \subseteq Int(A) \cap B \subseteq A \cap B\).
\newline
Also, \(Int(A) \cap Int(B)\) is an intersection of open sets, which is open. \newline
So x is an element of an open subset of \(A \cap B\). \newline
Since Int\((A \cap B)\) is the union of open subsets of \(A \cap B\) we can write \newline
\(x \in\) Int\((A \cap B)\)
\newline
\(Int(A) \cap Int(B) \subseteq\) Int\((A \cap B)\)
\newline \(\square\) \newline
\((\leftarrow)\)Let \(x \in\) Int\((A \cap B)\).
\newline
By Definition, Int\((A \cap B)\) is an open set. \newline
Int\((A \cap B) \subseteq A \cap B\) \newline
Int\((A \cap B) \subseteq A\) and Int\((A \cap B) \subseteq B\) \newline \newline
So x is in an open set in A and x is in an open set in B. \newline
This means that \(X \in \) Int\((A)\) and \(x \in\) Int\((B)\).
\newline \newline
Thus, \(x \in Int(A) \cap Int(B)\). \newline
Int\((A \cap B) \subseteq Int(A) \cap Int(B)\)
\newline \(\square\) \newline
Int\((A) \cap \) Int\((B) = \) Int\((A \cap B)\).
\newline \(\blacksquare\)

\end{document}
