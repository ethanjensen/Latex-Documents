\documentclass[12pt]{article}
\usepackage[utf8]{inputenc}
\usepackage{amsmath}
\usepackage{amsfonts}
\usepackage{amssymb}
\usepackage{empheq}
\usepackage{tikz}
\usepackage{enumitem}
\usepackage{eucal}
\usepackage{changepage}
\usetikzlibrary{automata, positioning, shapes}
\addtolength{\topmargin}{-0.75in}
\addtolength{\textheight}{1.75in}
\addtolength{\evensidemargin}{-0.5in}

\title{Topology Homework 03}
\author{Ethan Jensen, Luke Lemaitre, Kasandra Lassagne}
\date{February 17, 2020}

\begin{document}
	\maketitle
	\noindent
	\textbf{EXERCISE 1.10}
	\textit{Show that} \(\mathbb{B}\) \textit{is a basis for a topology on} \(\mathbb{R}\). \newline \newline
	\textbf{Proof.} \newline
	Notice that \(\bigcup_{B \in \mathbb{B}}B = \mathbb{R}\)
	\newline
	\(\therefore \forall x \in \mathbb{R} \exists B \in \mathbb{B}  \ni x \in B\).
	\newline \(\square\) \newline
	Consider \(B_1, B_2 \in \mathbb{B}\) where \(B_1 = [a,b)\) and \(B_2 = [c,d)\). \newline
	Let x \in \(B_1 \cap B_2\).
	\newline
	Let \(x \in B_1 \cap B_2\). \newline
	So max\((a,c) \leq x <\)min\((b,d)\).
	\newline
	Thus, \(x \in [\textup{max}(a,c), \textup{min}(b,d)) \in \mathbb{B}\). Call it \(B'\). \newline
	So \(\exists B' \subseteq B_1 \cap B_2 \ni x \in B' \in \mathbb{B}\).
	\newline \(\blacksquare\)
	\newpage
	\noindent
	\textbf{EXERCISE 1.11} \textit{Show which collections of subsets of} \(\mathbb{R}\) \textit{are bases.}
	\newline \newline
	\textbf{(a)} \(C_1 = \{(n,n+2) \subseteq \mathbb{R} | n \in \mathbb{Z}\}\) \newline
	\((0,2)\) and \((1,3)\) are basis elements of \(C_1\). \newline
	By inspection, \(1.5 \in (0,2)\) and \(1.5 \in (1,3)\), so \(1.5 \in (0,2) \cap (1,3) = (1,2)\). \newline
	If \(C_1\) was a basis, then there exists a basis element \(B' \subseteq (1,2)\) such that \(1.5 \in B'\). \newline
	However, since all basis elements are of length 2, no basis element can be a subset of the open interval \((1,2)\). \newline
	\boxed{C_1 \textup{ is not a basis}}
	\newline
	\textbf{(b)}\ \(C_2 = \{[a,b] \subseteq \mathbb{R}| a < b\}\)
	\newline
	\([0,1]\) and \([1,2]\) are basis elements of \(C_2\).
	\newline
	By inspection, \(1 \in [0,1] \cap [1,2] = \{1\}\). \newline
	If \(C_2\) was a basis, then there exists a basis element \(B' \subseteq \{1\}\) such that \(1 \in B'\). \newline
	However, since the upper and lower limits of the intervals in the basis set are distinct, no basis element can be a subset of \(\{1\}\). \newline
	\boxed{C_2 \textup{ is not a basis}}
	\newline
	\textbf{(c)}\ \(C_3 = \{[a,b] \subseteq \mathbb{R}| a \leq b\}\)
	\newline
	Consider a point \(x \in \mathbb{R}\). \newline
	The basis element \([x,x] = \{x\}\) contains x.
	\newline \(\square\) \newline
	Consider two basis elements \([a,b]\) and \([c,d]\) with nonempty intersection, and let \(x \in [a,b] \cap [c,d]\).
	\newline
	Let \(j = \textup{max}(a,c)\) and \(k = \textup{min}(b,d)\).
	\newline
	Since \([a,b]\) and \([c,d]\) have a nonempty intersection, we know that \(j \leq k\). \newline
	So \([j,k]\) is a basis element. Call it \(B'\).
	\newline
	\([a,b] \cap [c,d] = [j,k]\)
	\newline
	Thus, \(x \in [a,b] \cap [c,d] \implies x \in B'\) where B' is a basis element,
	\newline
	\boxed{C_3 \textup{ is a basis}}
	\newline
	\textbf{(d)}\ \(C_4 = \{(-x,x)\subseteq \mathbb{R}|\ x \in \mathbb{R}\}\) \newline
	Consider a point \(x \in \mathbb{R}\). \newline
	\(x \in (x-1, x+1)\) where \((x-1, x+1)\) is a basis element.
	\newline \(\square\) \newline
	Consider two basis elements \((-a,a)\) and \((-b,b)\) with nonempty intersection, and let \(x \in (-a, a) \cap (-b, b)\).
	\newline
	Either \(a<b\) or \(b<a\). Without loss of generality, say that \(a<b\). \newline
	Since the sets are nested, \((-a,a) \cap (-b,b) = (-a,a)\), which is a basis element. \newline
	Thus, \(x \in (-a,a) \cap (-b,b) \implies x \in (-a,a)\) where \((-a,a)\) is a basis element.
	\newline
	\boxed{C_4 \textup{ is a basis}}
	\newline
	\textbf{(e)}\ \(C_5 = \{(a,b)\cup \{b+1\}\subseteq \mathbb{R}|\ a < b\}\) \newline
	Each basis element in \(C_5\) is nested and covers the full space \(\mathbb{R}\), so by a similar argument to part \textbf{(d)}, \(C_5\) is a basis for a topology on \(\mathbb{R}\).
	\newline
	\boxed{C_5 \textup{ is a basis}}
	\newpage
	\noindent
	\textbf{EXERCISE 1.13} \textit{Consider the following six topologies defined on} \(\mathbb{R}\): \textit{the trivial topology, the discrete topology, the finite complement topology, the standard topoiogy, the lower limit topology, and the upper limit toology. Show how they compare to each other (finer, strictly finer, coarser, strictly coarser, noncomparable) and justify your claim.}
	\newline \newline
	The \textbf{discrete topology} is the finest. It is strictly finer than all the other topologies. This is because it contains all possible subsets of \(\mathbb{R}\). The other topologies have other elementhood conditions.
	\newline \newline
	The \textbf{lower limit} and \textbf{upper limit} topologies are noncomparable to each other, but both are finer than the standard topology. This is because \(\bigcup_{x>a} [x,b) = \bigcup_{x < b} (a,x] = (a,b)\) by the Union Lemma. All basis elements of the standard topology are open sets in said topologies.
	\newline \newline
	The \textbf{standard topology} is strictly finer than the finite complement topology. By a similar idea, complements of finite sets in \(\mathbb{R}\) can be expressed as a union of open intervals. However, in general, open sets in the standard topology do not have a finite complement.
	\newline \newline
	The \textbf{finite complement topology} is strictly finer than the trivial topology. The trivial topology only contains \(\varnothing\) and \(\mathbb{R}\), which all the topologies (including the finite complement topology) contain.
	\newline \newline
	The \textbf{trivial topology} is the coarsest of all topologies.
	\newline \newline
	\newpage
	\noindent
  \textbf{EXERCISE 1.16.} \textit{On the plane} \(\mathbb{R}^2\) \textit{, let}
	\[\mathbb{B}=\{(a,b)\times (c,d) \subseteq \mathbb{R}^2 | a<b,\ c<d\}\]
	\textbf{(a)}\ \textit{Show that} \(\mathbb{B}\) \textit{is a basis for a topology on} \(\mathbb{R}^2\). \newline
	\textbf{(b)}\ \textit{Show that the topology} \(\mathcal{T}'\) \textit{generated by} \(\mathbb{B}\) \textit{is the standard topology on} \(\mathbb{R}^2\).
	\newline \newline
	\textbf{(a)}\newline (1)\ \ Consider some \((x,y) \in \mathbb{R}^2\). \newline
	Let \(B'=(x-1 , x+1) \times (y-1, y+1)\). \newline
	Since \(x \in (x-1) , x+1)\) and \(y \in (y-1, y+1),\ (x,y) \in B'\) \newline
	So \(\exists B' \in \mathbb{B} \ni (x,y) \in B' \forall (x,y) \in \mathbb{R}^2\)
	\newline \(\square\) \newline
	(2)\ \ Let \(B_1,\ B_2 \subseteq \mathbb{B}\) \newline
	\(B_1 = (a_1, b_1) \times (a_2, b_2),\ a_1 < b_1,\ a_2 < b_2\) and \(B_2 = (c_1, d_1) \times (c_2, d_2),\ c_1 < c_1,\ c_2 < d_2\)
	\newline
	Let \((x,y) \in B_1 \cap B_2\) \newline
	Using the laws of algebra, we know that \newline
	\(\textup{max}(a_1, c_1)< x < \textup{min}(b_1, d_1)\)\ \  and\ \  \(\textup{max}(a_2, c_2) < y < \textup{min}(b_2, d_2)\)
	\newline \newline
	Let \(B'= (\textup{max}(a_1, c_1), \textup{min}(b_1, d_1)) \times (\textup{max}(a_2, c_2), \textup{min}(b_2, d_2))\)
	\newline
	Thus, \((x,y) \in B' \subseteq B_1 \cap B_2\) \newline
	\(\therefore (x,y) \in B_1 \cap B_2 \implies \exists B' \subseteq B_1 \cap B_2 \ni (x,y) \in B'\ \ \forall\  B_1,B_2\in\mathbb{B}\)
	\newline \(\square\) \newline
	\(\therefore \mathbb{B}\) is a basis for a topology.
	\newline \(\blacksquare\) \newline
	\newpage
	\noindent
	\textbf{(b)}\newline
	Let \(\mathbb{B}_1 = \{(a,b)\times (c,d) |\ a<b, c<d\}\) \newline
	Let \(\mathbb{B}_2 = \{B(p,r)|\ r > 0\}\)
	\newline
	Let \(\mathcal{T}_1\) be the topology generated by \(\mathbb{B}_1 = \mathbb{B}\)
	\newline
	Let \(\mathcal{T}_2\) be the topology generated by \(\mathbb{B}_2 \) (the standard topology).
	\newline
	\begin{figure}[ht]
	\centering
	\begin{tikzpicture}
		\draw
		(-1,0) edge (9,0)
		(0,-1) edge (0,6);
	\begin{scope}[very thick,dashed]
		\draw
		(1,1.2) edge (7, 1.2)
		(1,1.2) edge (1, 4.8)
		(1,4.8) edge (7, 4.8)
		(7,1.2) edge (7, 4.8)
		(3,2.2) circle (1cm);
		\node[] (x) at (1,-0.3) {$a$};
		\node[] (x) at (7,-0.3) {$b$};
		\node[] (x) at (-0.3,1.2) {$c$};
		\node[] (x) at (-0.3,4.8) {$d$};

		\node[] (x) at (7.2, 1) {$B_1$};
		\node[] (x) at (4.5, 1.7) {$B(p,r)$};
		\node[] (x) at (3,2) {$(x,y)$};
		\node[] (x) at (3.5,2.4) {$r$};

		\fill (3,2.2) circle[radius=2pt];
	\end{scope}
	\draw (3,2.2) edge (3.7071,2.9071);
	\end{tikzpicture}
	\caption{\(\rightarrow\)}
	\end{figure}
	\newline
	\begin{figure}[ht]
	\centering
	\begin{tikzpicture}
		\draw
		(-1,0) edge (9,0)
		(0,-1) edge (0,6);
	\begin{scope}[very thick,dashed]
		\draw
		(5.43, 3.73) rectangle (6.57, 4.87)
		(4,3.2) circle (3cm);

		\node[] (x) at (5.8, 3.3) {$B_q$};
		\node[] (x) at (7.5, 1.7) {$B(p,r)$};
		\node[] (x) at (4,2.9) {$p$};
		\node[] (x) at (6,4) {$q$};
		\node[] (x) at (5.6, 4.1) {$s$};

		\fill (4,3.2) circle[radius=2pt];
		\fill (6,4.3) circle[radius=2pt];
		\fill (5.6,4.4) circle[radius=2pt];

		\node[] (x) at (2.25,3.5) {$r$};

	\end{scope}
	\draw (4,3.2) edge (1,3.2);
	\end{tikzpicture}
	\caption{\(\leftarrow\)}
	\end{figure}
	\newline
	\(\rightarrow\) Let \(B_1 \in \mathbb{B}_1\)
	\newline
	\(\exists a,b,c,d \ni B_1 = (a,b) \times (c,d)\)
	\newline
	Let \(p = (x,y) \in B_1\), so \(a<x<b\) and \(c<y<d\).
	\newline
	Let \(r=min(x-a,b-x,y-c,d-y)\), so \(r > 0\).
	\newline \newline
	Consider \(B(p,r) \in \mathbb{B}_2\)
	\newline
	\(B(p,r)\in \mathcal{T}_2\)
	\newline
	Let \(q \in B(p,r)\) \newline
	\(\exists m,\ \theta\  \ni 0< m <r\) where \(q=(x+m\cos \theta,\  y+m\sin \theta)\)
	\newline
	Since \(-1 \leq \sin\theta \leq 1\),\ \ \(-r \leq m \sin\theta \leq r\)
	\newline
	Since \(-1 \leq \cos\theta \leq 1\),\ \ \(-r \leq m \cos\theta \leq r\)
	\newline
	\begin{adjustwidth}{2cm}{0cm}
		\begin{flushleft}
			\(a < x-r\)\ \ \ \ \ \ \ \ \ \ \ \ \ \ \(x+r<b\) \newline
			\(a<x-m\)\ \ \ \ \ \ \ \ \ \ \ \ \ \(x+m<b\) \newline
			\boxed{a < x+m\cos \theta}\ \ \ \ \ \ \boxed{x+m\cos \theta < b} \newline
			\newline
			\(c < y-r\)\ \ \ \ \ \ \ \ \ \ \ \ \ \ \(y+r<d\) \newline
			\(c<y-m\)\ \ \ \ \ \ \ \ \ \ \ \ \ \(y+m<d\) \newline
			\boxed{c < y+m\sin \theta}\ \ \ \ \ \ \boxed{y+m\cos \theta < d}
	\end{flushleft}
	\end{adjustwidth}
	\(\ \)
	\newline
	So \(q \in B_1\) \newline
	Thus, \(B(p,r) \subseteq B_1\) \newline
	By the Union Lemma, \(B_1 = \bigcup_{p \in B_1}B(p,r)\). \newline \newline
	Let \(U \in \mathcal{T}_1.
	\newline
	U = \bigcup B_k\) where \(B_k\in
	\mathbb{B}_1\) since \(\mathbb{B}_1\) generates \(\mathcal{T}_1\)
	\newline
	Thus, \(U = \bigcup\bigcup_{p \in B_k}B(p,r)\), which is a union of basis elements from \(\mathcal{T}_2\). \newline
	So \(U \in \mathcal{T}_2\)
	\newline
	\(\therefore \mathcal{T}_1 \subseteq \mathcal{T}_2\)
	\newline \(\square\) \newline \newline
	\(\leftarrow\) Let \(B(p,r) \in \mathbb{B}_2\) \newline
	Let \(q = (x_q, y_q) \in B(p,r)\). \newline
	\(\exists m \ni 0 < m < r\) and \(d(p,q) = m\) \newline
	Let \(B_q = \{(x_q + \frac{m-r}{\sqrt{2}}, x_q +\frac{r-m}{\sqrt{2}}) \times (y_q + \frac{m-r}{\sqrt{2}}, y_q +\frac{r-m}{\sqrt{2}})\}\) \newline
	\(B_q \in \mathbb{B}_1\)
	\newline
	Let \(s \in B_q\). \newline
	By the formula for Euclidian distance
	\[d(q,s)<\sqrt{\left(x_q-(x_q \pm \frac{m-r}{\sqrt{2}})\right)^2+\left(y_q-(y_q \pm \frac{m-r}{\sqrt{2}})\right)^2}\]
	\[d(q,s)<\sqrt{\left(\frac{m-r}{\sqrt{2}}\right)^2 + \left(\frac{m-r}{\sqrt{2}}\right)^2}\]
	\[d(q,s)<\sqrt{(r-m)^2}\]
	Since \(r>m\)
	\[d(q,s)<r-m\]
	By the triangular property \newline
	\(d(p,s)<d(p,q)+d(q,s)\) \newline
	\(d(p,s)<m+(r-m)\)\ \ \ so \(d(p,s)<r\) \newline
	\newline
	Thus, \(s \in B(p,r)\).
	\newline
	So \(B_q \subseteq B(p,r)\) \newline
	By the Union lemma \(\bigcup_{q \in B(p,r)}B_q=B(p,r)\)
	\newline \newline
	Let \(U \in \mathcal{T}_2\).
	\newline \(U = \bigcup B_k\) where \(B_k \in \mathbb{B}_2\) since \(\mathbb{B}_2\) generates \(\mathcal{T}_2\). \newline
	Thus, \(U = \bigcup \bigcup_{q \in B(p,r)}B_q\), which is a union of basis elements from \(\mathcal{T}_1\). \newline
	So \(U \in \mathcal{T}_1\) \newline
	\(\therefore \mathcal{T}_2 \in \mathcal{T}_1\)
	\newline \(\square\) \newline
	\(\therefore \mathcal{T}_1 = \mathcal{T}_2\) \newline
	The topology \(\mathcal{T}'\) generated by \(\mathbb{B}\) is the standard topology on \(\mathbb{R}^2\).
	\newline \(\blacksquare\) \newline
	\newpage
	\noindent
	\textbf{EXERCISE 1.17.} \textit{An} \textbf{open half plane} \textit{is a subset of }\(\mathbb{R}^2\) \textit{in the form} \(\{(x,y) \in \mathbb{R}^2|\ Ax + By < C\}\) \textit{for some} \(A,B,C \in \mathbb{R}\) \textit{with either} \(A\) \textit{or} \(B\)
	\textit{nonzero. Prove that open half planes are open sets in the standard topology on} \(\mathbb{R}^2\). \newline \newline
	\textbf{Proof.} \newline
	Let \(H = \{(x,y) \in \mathbb{R}^2|\ Ax + By < C\}\) be an open half plane and let \(p = (x_p,y_p) \in H\).
	\newline
	Thus, \(Ax_p + By_p < C\) \newline
	Between any two real numbers exists another real number. Thus,
	\[\exists \epsilon \ni Ax_p + By_p + \frac{\epsilon}{A+B} < C,\ \ \frac{\epsilon}{A+B} > 0\]
	Consider the open ball \(B_p(p, \frac{\epsilon}{A+B})\)
	\newline
	Let \(q \in B_p\). \newline \newline
	\(\exists m,\ \theta\  \ni 0< m <\frac{\epsilon}{A+B}\) where \(q=(x_p+m\cos \theta,\  y_p+m\sin \theta)\) \newline
	Since max\((\cos(\theta))\) and max\((\sin(\theta))\) are 1, we know
	\newline \(m\cos \theta < \frac{\epsilon}{A+B}\) and \(m\sin \theta < \frac{\epsilon}{A+B}\)
	\newline
	\[A(x_p + m\cos(\theta)) + B(y_p + m\sin(\theta)) \leq A(x_p + \frac{\epsilon}{A+B}) + B(y_p + \frac{\epsilon}{A+B})\]
	\[A(x_q) + B(y_q) < Ax_p + By_p + \frac{(A+B)\epsilon}{A+B} < C\]
	\[A(x_q) + B(y_q) < C\]
	So \(q \in H\) \newline
	Thus, \(B_p \subseteq H\).
	\newline \(\square\) \newline
	Consider \(\bigcup_{p \in H}B_p\). \newline \newline
	By the union Lemma, \(\bigcup_{p \in H}B_p = H\)
	\newline
	So H is a union of open balls. \newline
	This makes H an open set in the standard topology.
	\newline \newline
	\(\therefore\)\ Half planes are open sets in the standard topology on \(\mathbb{R}^2\).
	\newline \(\blacksquare\)
	\newpage
	\noindent
	\textbf{EXERCISE 1.25.} \textit{Prove that, in a topological space X, if U is open and C is closed, then} \(U-C\) \textit{is open and} \(C-U\) \textit{is closed.} \newline \newline
	\textbf{Proof.} \newline
	Since C is closed, there exists an open set \(U_c\) such that \(C = X - U_c\). \newline
	Since U is open, there exists a closed set \(C_u\) such that \(U = X - C_u\) \newline \newline
	\(U - C = U - (X - U_c) = U \cap U_c\) \newline
	So \(U - C\) is the intersection of finite open sets, and is thus open.
	\newline \newline
	\(C - U = C - (X - C_u) = C \cap C_u\) \newline
	So \(C - U\) is the intersection of closed sets, and it thus closed.
	\newline \(\blacksquare\)
	\newpage
	\noindent
	\textbf{EXERCISE 1.26.} \textit{Prove that closed balls are closed sets on the standard topology on} \(\mathbb{R}^2\).
	\newline \newline
	\textbf{Proof.} \newline
	Consider a closed ball \(\overline{B}(p, r)\). \newline
	Let q be a point outside of the closed ball \(\overline{B}(p, r)\). \newline
	Thus, \(d(p, q) = r + \epsilon\),\ \ where \(\epsilon > 0\)\newline
	Consider \(B_q(q, \epsilon)\). and let some point \(s \in B_q\).
	\[d(s, q) < \epsilon\]
	By the triangle inequality,
	\[d(s, q) + d(s,r) < d(p,q)\]
	\[\epsilon + d(s,r) < r + \epsilon\]
	\[d(s,r) < r\]
	So the point s is not in the closed ball \(\overline{B}(p, r)\). It is in its complement. \newline
	So \(B_q \subseteq X - \overline{B}(p, r)\).
	\newline \(\square\) \newline
	Consider \(\bigcup_{q \in X - \overline{B}(p, r)} B_q\)
	\newline \newline
	By the Union Lemma, \(\bigcup_{q \in X - \overline{B}(p, r)} B_q = X - \overline{B}(p, r)\).
	So \(X - \overline{B}(p, r)\) is a union of open balls.
	\newline
	This makes \(X - \overline{B}(p, r)\) open in the standard topology.
	\newline
	So \(\overline{B}(p, r)\) is closed in the standard topology.
	\newline \newline
	\(\therefore\) Closed balls are closed in the standard topology. \newline \(\blacksquare\)
	\newpage
	\noindent
	\textbf{EXERCISE 1.28} \textit{Which sets are closed sets in the finite complement topology on a topological space X?}
	\newline \newline
	The set X and any all finite subsets of X are closed sets in the finite complement topology on X.
	\newline \newline \newline \newline \newline \newline
	\textbf{EXERCISE 1.29} \textit{Which sets are closed sets in the excluded point topology} \(EPX_p\) \textit{on a set X?}
	\newline \newline
	All open sets in the particular point topology \(PPX_p\) are closed sets in the excluded point topology on X.
	\newline \newline \newline \newline \newline \newline
	\textbf{EXERCISE 1.30} \textit{Which sets are closed sets in the particular point topology} \(EPX_p\) \textit{on a set X?}
	\newline \newline
	All open sets in the excluded point topology \(EPX_p\) are closed sets in the particular point topology on X.
	\newpage
	\noindent
	\textbf{EXERCISE 1.32} \textit{Prove that intervals of the form} \([a,b)\) \textit{are closed in the lower limit topology on} \(\mathbb{R}\).
	\newline \newline
	\textbf{Proof.} \newline
	Consider the sets \(U_l = \bigcup_{c < a}[c,a)\) and \(U_r = \bigcup_{d > b}[b,d)\) \newline
	By the union lemma, \(U_l = (-\infty, a)\) and \(U_r = [b, \infty)\). \newline \newline
	\(U_l\) and \(U_r\) are the union of open sets and are thus open. \newline
	Furthermore, \(U_l \cup U_r\) must also be an open set for the same reason. \newline \newline
	\(U_l \cup U_r\) is the complement of \([a,b)\).
	\newline
	Since the complement of \([a,b)\) is open, \([a,b)\) is closed.
	\newline \newline
	\(\therefore\) Intervals of the form \([a,b)\) are closed in the lower limit topology on \(\mathbb{R}\).
	\newline \(\blacksquare\)
	\newpage
	\noindent
	\textbf{EXERCISE 1.33} \textit{Let X be a topological space}
	\begin{adjustwidth}{0.8cm}{0cm}
		\textbf{(a)}\ Prove that \(\varnothing\) and X are closed sets. \newline
		\textbf{(b)}\ Prove that the intersection of any collection of closed sets in X is a closed set. \newline
		\textbf{(c)}\ Prove that the union of finitely many closed sets in X is a closed set.
	\end{adjustwidth}
	\(\ \)
	\newline
	\textbf{(a) Proof.} \newline
	By the definition of a topology, X and \(\varnothing\) are open sets. \newline
	Their complements, \(\varnothing\) and X are thus closed sets.
	\newline \(\square\) \newline \newline
	\textbf{(b) Proof.} \newline
	Consider an intersection of closed sets \(\bigcap_i C_i\).
	\newline
	This can be written as \(X \cap \bigcap_i C_i\).
	\newline \newline
	Each closed set \(C_i\) can be expressed as the complement of an open set \(X - U_i\) for some \(U_i\).
	\newline
	So \(\bigcap_i C_i = X \cap \bigcap_i (X - U_i)\)
	\newline
	By DeMorgan's law, we can write
	\(\bigcap_i C_i = X - \bigcup_i U_i\)
	\newline \newline
	\(\bigcup_i U_i\) is a union of open sets and is thus open.
	\newline
	Since \(\bigcap_i C_i\) is the complement of an open set, \(\bigcap_i C_i\) is closed.
	\newline \(\square\) \newline \newline
	\textbf{(c) Proof.} \newline
	Consider a finite union of closed sets \(\bigcup_{i=1}^n C_i\).
	\newline
	This can be written as \(X \cap \bigcup_{i=1}^n C_i\).
	\newline \newline
	Each closed set \(C_i\) can be expressed as the complement of an open set \(X - U_i\) for some \(U_i\).
	\newline
	So \(\bigcup_{i=1}^n C_i = X \cap \bigcup_{i=1}^n (X - U_i)\)
	\newline
	By DeMorgan's law, we can write
	\(\bigcup_{i=1}^n C_i = X - \bigcap_{i=1}^n U_i\)
	\newline \newline
	\(\bigcap_{i=1}^n U_i\) is an intersection of finitely many open sets and is thus open. \newline
	Since \(\bigcup_{i=1}^n C_i\) is the complement of an open set, \(\bigcup_{i=1}^n C_i\) is closed.
	\newline \(\blacksquare\)
	\newpage
	\noindent
	\textbf{EXERCISE 1.35}\ \textit{Show that} \(\mathbb{R}\)
	\textit{in the lower limit topology is Hausdorff.}
	\newline \newline
	Consider two distinct points p and q in the lower limit topology on \(\mathbb{R}\). \newline \newline
	Since p and q are distinct, either \(p < q\) or \(q < p\).
	\newline
	Without loss of generality, let p be less than q.
	\newline \newline
	Consider the open sets \(U_1 = [p,q)\) and \(U_2 = [q, q+1)\).
	\newline
	It is easy to see that \(U_1\) and \(U_2\) are disjoint. \newline
	Furthermore, \(p \in U_1\) and \(q \in U_2\).
	\newline \newline
	Thus, there are disjoint neighborhoods \(U_1\) and \(U_2\) around p and around q respectively.
	\newline \newline
	\(\therefore\) \(\mathbb{R}\) in the lower limit topology is Hausdorff.
	\newline \(\blacksquare\)
	\newpage
	\noindent
	\textbf{EXERCISE 1.36}\ \textit{Show that} \(\mathbb{R}\)
	\textit{in the finite complement topology is not Hausdorff.}
	\newline \newline
	Consider two distinct points p and q in the lower limit topology on \(\mathbb{R}\). \newline \newline
	Consider two open sets \(U_1\) and \(U_2\) such that \(p \in U_1\) and \(q \in U_2\). \newline
	\(U_1\) and \(U_2\) can be written as \(\mathbb{R} - F_1\) and \(\mathbb{R} - F_2\) respectively, where \(F_1\) and \(F_2\) are finite sets.
	\newline
	\newline
	Consider \(U_1 \cap U_2\).
	\newline
	\(U_1 \cap U_2 = (\mathbb{R} - F_1) \cap (\mathbb{R}- F_2)\)
	\newline
	By Demorgan's Law, \newline
	\(U_1 \cap U_2 = \mathbb{R} - (F_1 \cup F_2)\)
	\newline \newline
	Since \(\mathbb{R} - (F_1 \cup F_2)\) is the difference between an uncountable set and a finite set, it is nonempty.
	\newline
	This means that \(U_1\) and \(U_2\) are not disjoint.
	\newline
	Thus, there are no disjoint neighborhoods around p and around q. \newline \newline
	\(\therefore\) \(\mathbb{R}\) in the finite complement topology is not Hausdorff.
	\newline \(\blacksquare\)
\end{document}
