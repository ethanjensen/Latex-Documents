\documentclass[12pt]{article}
\usepackage[utf8]{inputenc}
\usepackage{amsmath}
\usepackage{amsfonts}
\usepackage{amssymb}
\usepackage{empheq}
\usepackage{tikz}
\usepackage{enumitem}
\usepackage{eucal}
\usepackage{changepage}
\usetikzlibrary{automata, positioning, shapes, decorations.markings}
\addtolength{\topmargin}{-0.75in}
\addtolength{\textheight}{1.75in}
\addtolength{\evensidemargin}{-0.5in}

\title{Topology Homework 07}
\author{Ethan Jensen}
\date{April 6, 2020}

\begin{document}
  \maketitle
  \noindent
  \textbf{EXERCISE 4.1}
  \newline
  \textbf{(a)}\ \ Let X have the discrete topology and Y be an arbitrary topological space.
  \newline Show that every function \(f:X\rightarrow Y\) is continuous.
  \newline
  \textbf{(b)}\ \ Let Y have the trivial topology and Y be an arbitrary topological space.
  \newline Show that every function \(f:X\rightarrow Y\) is continuous.
  \newline \newline \newline
  \textbf{(a)} Let X have the discrete topology.
  \newline
  Consider some surjective function \(f:X \rightarrow Y\).
  \newline
  Consider some open set \(U \subseteq Y\).
  \newline \newline
  \(f^{-1}(U) \subseteq X\), which is open, since X has the discrete topology.
  \newline
  So for every open set U in Y, \(f^{-1}(U)\) is open in X.
  \newline
  \(\therefore\) Every function \(f:X\rightarrow Y\) is continuous.
  \newline \newline
  \textbf{(b)} Let Y have the trivial topology.
  \newline
  Consider some surjective function \(f:X \rightarrow Y\).
  \newline
  The open sets in Y consist of \(\varnothing\) and \(Y\).
  \newline \newline
  \(f^{-1}(\varnothing) = \varnothing\) and \(f^{-1}(Y) = X\)
  \newline
  \(\varnothing\) and \(X\) are open sets in X.
  \newline
  So for every open set U in Y, \(f^{-1}(U)\) is open in X.
  \newline
  \(\therefore\) Every function \(f:X\rightarrow Y\) is continuous.
  \newpage
  \noindent
  \textbf{EXERCISE 4.2}
  \newline
  \textbf{Prove Theorem 4.8:} Let X and Y be topological spaces. A function \(f:X \rightarrow Y\) is continuous if and only if \(f^{-1}(C)\) is closed in X for every closed set \(C \subseteq Y\).
  \newline \newline
  \textbf{Proof.}
  \newline
  \((\rightarrow)\) Assume \(f:X \rightarrow Y\) is continuous.
  \newline
  Consider some closed set \(C \subseteq Y\).
  \newline
  Then \(Y-C\) is open in Y.
  \newline
  Since f is continuous, \(f^{-1}(Y-C)\) is open in X.
  \newline \newline
  By Theorem 0.22, \(f^{-1}(Y-C) = f^{-1}(Y) - f^{-1}(C)\)
  \newline
  So \(f^{-1}(C)\) is the complement of an open set.
  \newline
  So \(f^{-1}(C)\) is closed in X.
  \newline
  Thus, \(f^{-1}(C)\) is closed in X for every closed set \(C \subseteq Y\).
  \newline \(\square\) \newline
  \((\leftarrow)\) Assume \(f^{-1}(C)\) is closed in X for every closed set \(C \subseteq Y\).
  \newline
  Consider some open set \(U \subseteq Y\).
  \newline
  \(U = Y - C\) for some closed set C in Y.
  \newline \newline
  \(f^{-1}(U)=f^{-1}(Y-C) = f^{-1}(Y) - f^{-1}(C)\) By Theorem 0.22.
  \newline
  \(f^{-1}(U) = X - f^{-1}(C)\), where \(f^{-1}(C)\) is closed in X.
  \newline
  So \(f^{-1}(U)\) the complement of a closed set in X, which is open in X for every set U that is open in Y.
  \newline
  Thus, \(f:X \rightarrow Y\) is continuous.
  \newline \(\square\) \newline
  \(\therefore\) A function \(f:X \rightarrow Y\) is continuous if and only if \(f^{-1}(C)\) is closed in X for every closed set \(C \subseteq Y\).
  \newline \(\blacksquare\)
  \newpage
  \noindent
  \textbf{EXERCISE 4.7}
  \newline
  Suppose X is a space with topologies \(\mathcal{T}_1\) and \(\mathcal{T}_2\). Let \(id(x) = x\), and assume that the domain X has the topology \(\mathcal{T}_1\) and that the range of X has the topology \(\mathcal{T}_2\). Show that \(id\) is continuous if and only if \(\mathcal{T}_1\) is finer than \(\mathcal{T}_2\).
  \newline
  \newline
  \textbf{Proof.}
  \newline
  \((\rightarrow)\) Assume \(id\) is continuous.
  \newline
  Consider some open set in \(U \in \mathcal{T}_2\), where \(U \subseteq Y\).
  \newline
  \(id^{-1}(U) = U\), so \(U \in \mathcal{T}_1\) since \(id\) is continuous.
  \newline
  So \(U \in \mathcal{T}_2 \implies U \in \mathcal{T}_1\).
  \newline
  Thus, \(\mathcal{T}_1\) is finer than \(\mathcal{T}_2\).
  \newline \(\square\) \newline
  \((\leftarrow)\) Assume \(\mathcal{T}_1\) is finer than \(\mathcal{T}_2\).
  \newline
  Consider some open set \(U \in \mathcal{T}_2\), where \(U \subseteq Y\).
  \newline
  Since \(\mathcal{T}_1\) is finer than \(\mathcal{T}_2\), \(id^{-1}(U) = U \in \mathcal{T}_2\).
  \newline
  In other words, \(id^{-1}(U)\) is also open in X.
  \newline
  Thus, \(id\) is continuous.
  \newline \(\square\) \newline
  \(\therefore id\) is continuous if and only if \(\mathcal{T}_1\) is finer than \(\mathcal{T}_2\).
  \newline \(\blacksquare\)
  \newpage
  \noindent
  \textbf{EXERCISE 4.9}
  \newline
  Let \(f,g:X\rightarrow Y\) be continuous functions. Assume that Y is Hausdorff and that there exists a dense subset D of X such that \(f(x) = g(x)\) for all \(x \in D\). Prove that \(f(x) = g(x)\) for all \(x \in X\).
  \newline \newline
  \textbf{Proof.}
  \newline
  Consider some point \(x \in X\).
  \newline
  Suppose \(f(x) \neq g(x)\).
  \newline
  Since Y is Hausdorff, there exist open sets \(U,V \subseteq Y\) such that \(f(x) \in U,\ g(x) \in V, U \cap V = \varnothing\).
  \newline \newline
  Since f and g are continuous functions, there exist two sets \(U_x,V_x \subseteq X \ni f(U_x) = U,\ g(V_x) = V\), where \(U_x \textup{ and } V_x\) are open in X.
  \newline \newline
  \(U_x \textup{and} V_x\) are both open sets that contain the point x.
  \newline
  So \(U_x \cap V_x\) is a non-empty open set.
  \newline
  Since D is a dense subset of X, \(\exists y \in D \ni y \in U_x \cap V_x\).
  \newline
  So \(y \in U_x\) and \(y \in V_x\).
  \newline
  So \(f(y) \in U\) and \(g(y) \in V\).
  \newline
  Since \(y \in D\), \(f(y) = g(y)\), and \(f(y) \in U \cap V\).
  \newline \newline
  This is a contradiction since we said \(U \cap V = \varnothing\).
  \newline
  \(\therefore f(x) = g(x)\) for all \(x \in X\).
  \newline \(\blacksquare\)
  \newpage
  \noindent
  \textbf{EXERCISE 4.13}
  \newline
  \textbf{(a)}\ Let \(f_1:X\rightarrow Y_1\) and \(f_2:X\rightarrow Y_2\) be continuous functions. Show that \(h: X \rightarrow Y_1 \times Y_2\), defined by \(h(x)=(f_1(x), f_2(x))\), is continuous as well.
  \newline
  \textbf{(b)}\ Extend the result of (a) to n functions, for \(n > 2\).
  \newline \newline
  \textbf{(a)}\ Let \(U \times V\) be open in \(Y_1 \times Y_2\)
  \newline
  \(h^{-1}(U \times V) = \{x|\ f_1(x) \in U \textup{ and } f_2(x) \in V\}\)
  \newline
  \(h^{-1}(U \times V) = \{x|\ f_1(x) \in U\} \cap \{x|\ f_2(x) \in V\}\)
  \newline
  \(h^{-1}(U \times V) = f_1^{-1}(U) \cap f_2^{-1}(V)\)
  \newline \newline
  Since \(f_1, f_2\) are continuous, \(f_1^{-1}(U) \textup{ and } f_2^{-1}(V)\) are both open in X.
  \newline
  So \(f_1^{-1}(U) \cap f_2^{-1}(V) = h^{-1}(U \times V)\) is open in X.
  \newline
  \(\therefore\) h is continuous as well.
  \newline \(\blacksquare\) \newline
  \textbf{(b)}
  \newline
  Let \(f_i: X \rightarrow Y_i,\ i = 1,2,...n\) be continuous functions.
  \newline
  Let \(h: X \rightarrow Y_1 \times Y_2 \times ... \times Y_n\) defined by \(h(x) = (f_1(x),f_2(x),...,f_n(x))\).
  \newline \newline
  Let \(U_1 \times U_2 \times ... \times U_n\) be open in \(Y_1 \times Y_2 \times ... \times Y_n\)
  \newline
  \(h^{-1}(U_1 \times U_2 \times ... \times U_n) = \{x|\ f_i(x) \in U_i\ \forall i\}\)
  \newline
  \(h^{-1}(U_1 \times U_2 \times ... \times U_n) = \bigcap_{i=1}^nf_i^{-1}(U_i)\)
  \newline \newline
  Since \(f_i\) is continuous for all i, \(f_i^{-1}(U_i)\) is open in X for all i.
  \newline
  Thus, \(\bigcap_{i=1}^nf_i^{-1}(U_i) = h^{-1}(U_1 \times U_2 \times ... \times U_n)\) is a finite intersection of open sets in X, and thus must also be open in X.
  \newline
  \(\therefore\) h is continuous as well.
  \newline \(\blacksquare\)
  \newpage
  \noindent
  \textbf{EXERCISE 4.14}
  \newline
  Show that the addition function, \(f: \mathbb{R}^2 \rightarrow \mathbb{R}\), given by \(f(x,y) = x+y\), is a continuous function.
  \newline \newline
  Consider an open interval \((a,b) \subseteq \mathbb{R}\).
  \newline
  \(f^{-1}((a,b)) = \{(x,y) \in \mathbb{R}^2|\ a < x+y < b\}\)
  \newline
  Now consider some \(p = (x,y)\) where \(p \in f^{-1}((a,b))\).
  \newline \newline
  Let \(B_p = B(p, r)\) where \(r = min\left(\frac{x+y-a}{\sqrt{2}},\frac{b-x-y}{\sqrt{2}}\right)\).
  \newline
  Now consider some \(q \in B(p, r)\).
  \newline
  Let \(m = d(p,q)\), which is less than \(r\).
  \newline
  So \(q = (x+m\cos\theta, y + m\sin\theta)\) for some \(\theta\)
  \newline
  \[m < r\]
  \[m < \frac{x+y-a}{\sqrt{2}}\]
  \[\sqrt{2}m < x+y-a\]
  Since \(\cos \theta + \sin \theta < \sqrt{2},\ \forall \theta\)
  \[m(-\cos \theta - \sin \theta)m < x + y - a\]
  \[m(\cos \theta + \sin \theta)m > - x - y + a\]
  \[x + y + m\cos \theta + m\sin \theta > a\]
  \[(x+ m\cos \theta) + (y + m\cos \theta) > a\]
  Similarly, it can be shown that
  \[(x+ m\cos \theta) + (y + m\cos \theta) < b\]
  So \(q \in f^{-1}(a,b)\)
  \newline
  So \(B_p \subseteq f^{-1}((a,b))\)
  \newline
  \newline
  By the Union Lemma, \(f^{-1}((a,b)) = \bigcup_{p \in f^{-1}((a,b))}B_p\).
  \newline
  \(f^{-1}((a,b))\) is an arbitrary union of open sets in \(\mathbb{R}^2\), which makes it open.
  \newline
  So \(f^{-1}((a,b))\) is open for all basis elements \((a,b) \in \mathbb{R}\).
  \newline \newline
  Consider an open set \(U \in \mathbb{R}\).
  \newline
  \(U = B_1 \cup B_2 \cup B_3...\) where \(B_i\) are basis elements in \(\mathbb{R}\).
  \newline
  By Theorem 0.22 \(f^{-1}(U) = f^{-1}(B_1) \cup f^{-1}(B_2) \cup f^{-1}(B_3)...\)
  \newline \newline
  \(f^{-1}(U)\) can thus be written as a union of open sets in \(\mathbb{R}^2\).
  \newline
  So \(f^{-1}(U)\) is open in \(\mathbb{R}^2\), and indeed is open for all \(U \in \mathbb{R}\)
  \newline \newline
  \(\therefore\) The addition function, \(f: \mathbb{R}^2 \rightarrow \mathbb{R}\), given by \(f(x,y) = x+y\), is continuous.
  \newline \(\blacksquare\)
  \newpage
  \noindent
  \textbf{EXERCISE 4.15}
  \newline
  Let \(f\) be the multiplicative function, \(f(x,y) = xy\). Complete the proof of continuity of \(f\) that was outlined in Example 4.6, by doing the following:
  \newline
  \textbf{(a)}\ Show that if p and q are both positive, and \(\delta\) is described in the example, then \((p-\delta, p+\delta) \times (q - \delta, q + \delta) \subseteq f^{-1}((a,b))\).
  \newline
  \textbf{(b)}\ Consider the rest of the posibilities for p and q being positive or negative, and show that \((p-\delta, p+\delta) \times (q - \delta, q + \delta) \subseteq f^{-1}((a,b))\).
  \newline
  \newline
  \textbf{(a) \ Proof.}
  Consider an open interval \((a,b) \in \mathbb{R}\).
  \newline
  \(f^{-1}((a,b)) = \{(x,y) \in \mathbb{R}^2|\ a < xy < b\}\)
  \newline
  Now consider some \(t = (p,q)\) where \(t \in f^{-1}((a,b))\).
  \newline
  Let \(m = \textup{min}\{b-pq,pq-a\}\)
  \newline
  Let \(B_t = (p - \delta, p + \delta) \times (q - \delta, q + \delta)\)
  \newline
  where \(\delta > 0\) is chosen such that \(\delta|p|, \delta|q|, \delta^2\) are all less than \(\frac{m}{3}\).
  \newline
  \newline
  Let \((x',y') \in (p-\delta,p+\delta) \times (q-\delta, q+\delta)\)
  \newline
  \[x' < p + \delta,\ y' < q + \delta\]
  \[x'y' < (p + \delta)(q + \delta)\]
  \[x'y' < pq + \delta|p|+ \delta|q| + \delta^2\]
  \[x'y' < pq + m\]
  \[x'y' < b - m + m\]
  \[x'y' < b\]
  \newline
  \[x' > p - \delta,\ y' > q - \delta\]
  \[x'y' > pq -|p|\delta -|q|\delta +\delta^2\]
  \[x'y' > pq - m/3\]
  \[x'y' > a + m - m/3\]
  \[x'y' > a\]
  \(a < x'y' < b\), so \((x',y') \in f^{-1}((a,b))\).
  \newline
  So \(B_t \subseteq f^{-1}((a,b))\).
  \newline
  \newline
  By the Union Lemma, \(f^{-1}((a,b)) = \bigcup_{t \in f^{-1}((a,b))}B_t\).
  \newline
  \(f^{-1}((a,b))\) is an arbitrary union of open sets in \(\mathbb{R}^2\), which makes it open.
  \newline
  So \(f^{-1}((a,b))\) is open for all basis elements \((a,b) \in \mathbb{R}\).
  \newline \newline
  Consider an open set \(U \in \mathbb{R}\).
  \newline
  \(U = B_1 \cup B_2 \cup B_3...\) where \(B_i\) are basis elements in \(\mathbb{R}\).
  \newline
  By Theorem 0.22 \(f^{-1}(U) = f^{-1}(B_1) \cup f^{-1}(B_2) \cup f^{-1}(B_3)...\)
  \newline \newline
  \(f^{-1}(U)\) can thus be written as a union of open sets in \(\mathbb{R}^2\).
  \newline
  So \(f^{-1}(U)\) is open in \(\mathbb{R}^2\), and indeed is open for all \(U \in \mathbb{R}\)
  \newline \newline
  \(\therefore\) For positive p and positive q, the multiplicative function, \(f: \mathbb{R}^2 \rightarrow \mathbb{R}\), given by \(f(x,y) = xy\), is continuous.
  \newline \(\square\) \newline
  \textbf{(b) \ Proof.}
  \newline
  We must consider the rest of the possibilities for p and q being positive or negative.
  \newline \newline
  (1) If both p and q are negative, then the same argument can be given as above since the shape \(f^{-1}((a,b))\) is symmetric about \(y=-x\).
  \newline
  (2) The shape of \(f^{-1}((-a,-b))\) is identical to \(f^{-1}((a,b))\), but flipped about the y-axis and the x-axis, allowing one of p and q to be negative, but still allowing the same argument to be used.
  \newline
  Flipping and rotation are linear transformations, which are continuous, which makes this argument rigorous.
  \newline \newline
  \(\therefore\) in all cases, the multiplicative function, \(f: \mathbb{R}^2 \rightarrow \mathbb{R}\), given by \(f(x,y) = xy\), is continuous.
  \newline \(\blacksquare\)
  \newpage
  \noindent
  \textbf{EXERCISE 4.16}
  \newline
  Use Example 4.6, Exercises 4.13 and 4.14, and Theorem 4.9 to show that the sum and product of a finite number of continuous functions are also continuous functions. That is, assuming that \(f_1,...,f_m: \mathbb{R} \rightarrow \mathbb{R}\) are continuous, prove that \(S: \mathbb{R} \rightarrow \mathbb{R}\) and \(P: \mathbb{R} \rightarrow \mathbb{R}\), defined by \(S(x) = f_1(x)+...+f_m(x)\) and \(P(x) = f_1(x)f_2(x)...f_m(x)\), are continuous.
  \newline \newline
  \textbf{Proof.}
  \newline
  These proofs will use Mathematical Induction.
  \newline
  \textbf{Basis Step:} Let \(S: \mathbb{R} \rightarrow \mathbb{R}\) be defined by \(S(x) = f_1(x)\).
  \newline
  Since \(f_1\) is a continuous function, \(S\) is also a continuous function.
  \newline \(\square\) \newline
  Likewise, like \(P: \mathbb{R} \rightarrow \mathbb{R}\) be defined by \(P(x) = f_1(x)\).
  \newline
  Since \(f_1(x)\) is a continuous function, \(P\) is also a continuous function.
  \newline \(\square\) \newline \newline
  \textbf{Induction Hypothesis:} Assume \(S: \mathbb{R} \rightarrow \mathbb{R}\) defined by \(S(x) = f_1(x) + f_2(x) +...f_k(x)\) is continuous, for some \(k \in \mathbb{Z}_+\)
  \newline
  Likewise, Assume \(P: \mathbb{R} \rightarrow \mathbb{R}\) defined by \(P(x) = f_1(x)f_2(x)...f_k(x)\) is continuous, for some \(k \in \mathbb{Z}_+\)
  \newline \newline
  \textbf{Induction Step:} Consider the function \(S': \mathbb{R} \rightarrow \mathbb{R}\) defined by \(S'(x) = f_1(x) + f_2(x) +...f_k(x) + f_{k+1}(x)\).
  \newline
  \(S'(x) = S(x) + f_{k+1}(x)\)
  \newline
  By our induction hypothesis, S is continuous.
  \newline
  Since S and \(f_{k+1}\) are both continuous, \(A: \mathbb{R} \rightarrow \mathbb{R}^2\) defined by \(A(x) = (S(x),f_{k+1}(x))\) is continuous using our result from Exercise 4.13.
  \newline
  The addition function \(B: \mathbb{R}^2 \rightarrow \mathbb{R}\) is also continuous using Exercise 4.14.
  \newline \newline
  Notice that the function \(S': \mathbb{R} \rightarrow \mathbb{R} = B \circ A\).
  \newline
  \(S'\) is a continuous function by Theorem 4.9.
  \newline
  \(\therefore\) \(S: \mathbb{R} \rightarrow \mathbb{R}\), defined by \(S(x) = f_1(x)+...+f_m(x)\) is continuous for all \(m \in \mathbb{Z}_+\), by Mathematical Induction.
  \newline \(\blacksquare\) \newline
  Consider the function \(P': \mathbb{R} \rightarrow \mathbb{R}\) defined by \(P'(x) = f_1(x)f_2(x)...f_k(x)f_{k+1}(x)\).
  \newline
  \(P'(x) = S(x)f_{k+1}(x)\)
  \newline
  By our induction hypothesis, P is continuous.
  \newline
  Since P and \(f_{k+1}\) are both continuous, \(A: \mathbb{R} \rightarrow \mathbb{R}^2\) defined by \(A(x) = (S(x),f_{k+1}(x))\) is continuous using our result from Exercise 4.13.
  \newline
  The multiplicative function \(B: \mathbb{R}^2 \rightarrow \mathbb{R}\) is continuous using Exercise 4.15.
  \newline \newline
  Notice that the function \(P': \mathbb{R} \rightarrow \mathbb{R} = B \circ A\).
  \newline
  \(P'\) is a continuous function by Theorem 4.9.
  \newline
  \(\therefore\) \(P: \mathbb{R} \rightarrow \mathbb{R}\), defined by \(P(x) = f_1(x)f_2(x)...f_m(x)\) is continuous for all \(m \in \mathbb{Z}_+\), by Mathematical Induction.
  \newline \(\blacksquare\)
  \newpage
  \noindent
  \textbf{EXERCISE 4.17}
  Use Exercise 4.16 to show that every polynomial function \(p: \mathbb{R} \rightarrow \mathbb{R}\), given by \(p(x) = a_nx^n + ... + a_1x + a_0\), is continuous.
  \newline \newline
  \newline
  (1) Constant functions defined by \(c(x) = x_0\) are continuous. (Exm. 4.2)
  \newline
  (2) The identity function defined by \(id(x) = x\) is continuous (Exm. 4.2)
  \newline
  (3) Functions composed of successive addition or multiplication of continuous functions are continuous. (Exr. 4.16)
  \newline \newline
  Consider the function \(p: \mathbb{R} \rightarrow \mathbb{R}\), given by \(p(x) = a_nx^n + ... + a_1x + a_0\).
  \newline
  Now consider a particular term in the series \(a_kx^k\) for some \(0 \leq k \leq n\).
  \newline
  Define a function \(f_k: \mathbb{R} \rightarrow \mathbb{R}\) where \(f_k(x) = a_k(x)\overbrace{id(x)id(x)...id(x)}^k\).
  \newline
  From (1), (2), and (3), \(f_k\) is continuous for all k.
  \newline \newline
  So \(p(x) = f_n(x) + ... + f_1(x) + f_0(x)\), where each \(f_k\) is continuous.
  \newline
  Thus, by (3), p is a continuous function.
  \newline \newline
  \(\therefore\) every polynomial function \(p: \mathbb{R} \rightarrow \mathbb{R}\), given by \(p(x) = a_nx^n + ... + a_1x + a_0\), is continuous.
  \newline \(\blacksquare\)
\end{document}
